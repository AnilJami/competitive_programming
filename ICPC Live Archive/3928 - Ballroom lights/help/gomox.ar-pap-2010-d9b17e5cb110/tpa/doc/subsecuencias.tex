\section{10069 - Distinct subsequences}

\textbf{Problema:} dadas dos strings $x = x_0 \hdots x_n$ y $z = z_0 \hdots z_m$,
determinar la cantidad de ocurrencias distintas de $z$ como subsecuencia de $x$.
(Dos ocurrencias son distintas si les corresponden distintas secuencias
de \'indices).

\subsection{Resoluci�n}

Se resolvi\'o mediante un algoritmo de programaci\'on din\'amica.
Se define la siguiente funci\'on $M(i,j)$ para $i \in [0,n]$, $j \in [0,m]$:
$$M(i,j) = \text{cantidad de ocurrencias de $z[0..j)$ como subsecuencia de $x[0..i)$}$$

\noindent
Definida concretamente por las siguientes ecuaciones:

\vspace{10pt}
\begin{tabular}{ll}
$M(i,0) = 1$ &\vspace{5pt}\\
$M(0,j) = 0$ \text{ (para $j > 0$)}&\vspace{5pt}\\
$M(i,j) = M(i - 1, j) + \begin{cases}
M(i - 1, j - 1) & \text{ si $x_{i - 1}$ = $z_{j - 1}$ }\\
0 & \text{ si no}
\end{cases}$
& (para $i, j > 0$)
\end{tabular}
\vspace{10pt}

La primera ecuaci\'on afirma que la cadena vac\'ia ocurre una vez en cualquier otra cadena
(la secuencia de \'indices de la cadena vac\'ia es una secuencia vac\'ia).
La segunda ecuaci\'on afirma que cualquier cadena no vac\'ia no ocurre en la cadena vac\'ia.
En la tercera ecuaci\'on se afirma que:
\begin{itemize}
\item Todas las ocurrencias de $z[0..j)$ en $x[0..i-1)$
      son ocurrencias de $z[0..j)$ en $x[0..i)$.

\item Adem\'as, si $x_{i - 1} = z_{j - 1}$, cada una de las ocurrencias de
      $z' = z[0..j-1)$ en $x' = x[0..i-1)$ tiene asociada una ocurrencia de
      $z'' = z[0..j)$ en $x'' = x[0..i)$. Esto es porque si $z'$ ocurre en $x'$
      con la secuencia de \'indices $\langle p_1, \hdots, p_{j-1} \rangle$, entonces
      $z''$ ocurre en $x''$ con la secuencia de \'indices $\langle p_1, \hdots, p_{j-1}, i - 1 \rangle$.

\item No puede haber otras ocurrencias de $z[0..j)$ en $x[0..i)$
      (por simple an\'alisis de casos)
      y adem\'as los dos tipos de ocurrencias descriptos son disjuntos
      (porque en el primer caso la secuencia de \'indices nunca termina en $i - 1$).
      Por lo tanto, la cantidad total de ocurrencias es la suma de estos
      dos tipos de ocurrencias.
\end{itemize}

El algoritmo se implementa computando la matriz de la funci\'on $M$,
visitando una vez cada posici\'on de la matriz. S\'olo se necesita el
resultado, $M(n,m) = $ cantidad de ocurrencias de $z$ en $x$. Alcanza
con mantener en memoria s\'olo las dos \'ultimas columnas de $M(i,j)$.

\begin{algorithm}[H]
\begin{algorithmic}
\caption{Cantidad de ocurrencias de $z$ como subsecuencia de $x$}
\PARAMS{$z$: string cuyas ocurrencias se quieren contar, $x$: string donde se buscan las ocurrencias de $z$}
\STATE $M :=$ matriz de 2 filas por $largo(z) + 1$ columnas%%, tal que $m_{i,j}=0$
\STATE $M_{i,0} := 1$ para $i \in \{1,2\}$
\STATE $M_{0,j} := 0$ para $j > 0$
\STATE $fila actual := 1$
\STATE $otra fila := 0$
\FOR{cada $i$ en $[1,largo(x)]$}
	\FOR{cada $j$ en $[1,largo(z)]$}
		\IF{$x_{i-1}$ es igual a  $z_{j-1}$}
			\STATE $M_{filaactual, j} = M_{otrafila, j} + M_{otrafila, j-1}$
		\ELSE
			\STATE	$M_{filaactual, j} = M_{otrafila, j}$
		\ENDIF
	\ENDFOR
	\STATE intercambiar $fila actual$ y $otra fila$
\ENDFOR
\RETURN $M_{otra fila, largo(z)}$
\end{algorithmic}
\end{algorithm}

Como vemos a partir del pseudoc\'odigo, si $largo(x)=n$ y $largo(z)=m$, la
complejidad temporal es $O(n \cdot m)$ en peor caso, y es el costo del llenado
de la matriz. La memoria usada son las dos filas de la matriz, es decir $O(m)$.

\subsubsection{Aritm\'etica de n\'umeros grandes}

El enunciado afirma que $|x| \leq 10000$, $|z| \leq 100$ y que la cantidad de
ocurrencias de $z$ en $x$ es a lo sumo $10^{100}$, que obviamente no cabe en registros
de 64 bits. Adem\'as, la afirmaci\'on no es irrelevante porque {\em pueden} darse
situaciones en las que el n\'umero de ocurrencias sea mucho mayor que $10^{100}$.
Por ejemplo, tomando $x = a^{10000}$ y $z = a^{100}$, hay
$10000 \choose 100$ ocurrencias de $z$ en $x$.

La \'unica operaci\'on requerida por el algoritmo es la suma de n\'umeros
grandes. Se program\'o una clase \texttt{BigInt} que representa
un n\'umero natural con un vector de 16 d\'igitos en base
$10^8$. Esto permite representar enteros en $[0,10^{128})$
e imprimir la salida en base 10 sin recurrir a divisiones.

La suma de n\'umeros grandes es $O(1)$ porque est\'an representados
con un arreglo que siempre tiene exactamente 16 d\'igitos. La
implementaci\'on de la suma con carry es la usual.

\subsection{Implementaci�n}
\noindent
\ttfamily
\shorthandoff{"}\\
\hlstd{}\hlline{\ 1\ }\hldir{\#include\ $<$vector$>$}\\
\hlline{\ 2\ }\hlstd{}\hldir{\#include\ $<$string$>$}\\
\hlline{\ 3\ }\hlstd{}\hldir{\#include\ $<$iostream$>$}\\
\hlline{\ 4\ }\hlstd{}\hldir{\#include\ $<$cassert$>$}\\
\hlline{\ 5\ }\hlstd{}\hldir{\#include\ $<$cstdio$>$}\\
\hlline{\ 6\ }\hlstd{}\\
\hlline{\ 7\ }\hlkwa{using\ namespace\ }\hlstd{std}\hlsym{;}\\
\hlline{\ 8\ }\hlstd{}\\
\hlline{\ 9\ }\hlkwc{typedef\ }\hlstd{}\hlkwb{unsigned\ int\ }\hlstd{uint32}\hlsym{;}\\
\hlline{10\ }\hlstd{}\\
\hlline{11\ }\hldir{\#define\ forsn(i,\ s,\ n)\ for\ (uint32\ i\ =\ (s);\ i\ $<$\ (n);\ i++)}\\
\hlline{12\ }\hlstd{}\hldir{\#define\ forn(i,\ n)\ forsn\ (i,\ 0,\ n)}\\
\hlline{13\ }\hlstd{}\hldir{\#define\ dforn(i,\ n)\ for\ (uint32\ i\ =\ (n);\ i{-}{-}\ $>$\ 0;)}\\
\hlline{14\ }\hlstd{}\\
\hlline{15\ }\hlslc{//\ BigInt}\\
\hlline{16\ }\hlstd{}\hlslc{//\ Represents\ a\ number\ in\ 0\ $<$=\ n\ $<$\ 10\ {*}{*}\ 128}\\
\hlline{17\ }\hlstd{}\\
\hlline{18\ }\hldir{\#define\ BigInt\textunderscore NDIGITS\ 16}\\
\hlline{19\ }\hlstd{}\hldir{\#define\ BigInt\textunderscore BASE\ 100000000}\\
\hlline{20\ }\hlstd{}\hlkwc{class\ }\hlstd{BigInt\ }\hlsym{\{}\\
\hlline{21\ }\hlstd{}\hlkwc{public}\hlstd{}\hlsym{:}\\
\hlline{22\ }\hlstd{\ }\hlkwd{BigInt}\hlstd{}\hlsym{()\ :\ }\hlstd{}\hlkwd{\textunderscore digits}\hlstd{}\hlsym{(}\hlstd{BigInt\textunderscore NDIGITS}\hlsym{,\ }\hlstd{}\hlnum{0}\hlstd{}\hlsym{)\ \{\}}\\
\hlline{23\ }\hlstd{\\
\hlline{24\ }\ }\hlkwd{BigInt}\hlstd{}\hlsym{(}\hlstd{uint32\ n}\hlsym{)\ :\ }\hlstd{}\hlkwd{\textunderscore digits}\hlstd{}\hlsym{(}\hlstd{BigInt\textunderscore NDIGITS}\hlsym{,\ }\hlstd{}\hlnum{0}\hlstd{}\hlsym{)\ \{}\\
\hlline{25\ }\hlstd{}\hlstd{\ \ }\hlstd{}\hlkwd{assert}\hlstd{}\hlsym{(}\hlstd{n\ }\hlsym{$<$\ }\hlstd{BigInt\textunderscore BASE}\hlsym{);}\\
\hlline{26\ }\hlstd{}\hlstd{\ \ }\hlstd{\textunderscore digits}\hlsym{{[}}\hlstd{}\hlnum{0}\hlstd{}\hlsym{{]}\ =\ }\hlstd{n}\hlsym{;}\\
\hlline{27\ }\hlstd{\ }\hlsym{\}}\\
\hlline{28\ }\hlstd{\\
\hlline{29\ }\ }\hlslc{//\ For\ testing}\\
\hlline{30\ }\hlstd{\ }\hlkwd{BigInt}\hlstd{}\hlsym{(}\hlstd{}\hlkwb{const\ }\hlstd{string}\hlsym{\&\ }\hlstd{s}\hlsym{)\ :\ }\hlstd{}\hlkwd{\textunderscore digits}\hlstd{}\hlsym{(}\hlstd{BigInt\textunderscore NDIGITS}\hlsym{,\ }\hlstd{}\hlnum{0}\hlstd{}\hlsym{)\ \{}\\
\hlline{31\ }\hlstd{}\hlstd{\ \ }\hlstd{uint32\ pow\ }\hlsym{=\ }\hlstd{}\hlnum{1}\hlstd{}\hlsym{;}\\
\hlline{32\ }\hlstd{}\hlstd{\ \ }\hlstd{uint32\ digit\ }\hlsym{=\ }\hlstd{}\hlnum{0}\hlstd{}\hlsym{;}\\
\hlline{33\ }\hlstd{}\hlstd{\ \ }\hlstd{uint32\ j\ }\hlsym{=\ }\hlstd{}\hlnum{0}\hlstd{}\hlsym{;}\\
\hlline{34\ }\hlstd{}\hlstd{\ \ }\hlstd{}\hlkwd{dforn\ }\hlstd{}\hlsym{(}\hlstd{i}\hlsym{,\ }\hlstd{s}\hlsym{.}\hlstd{}\hlkwd{size}\hlstd{}\hlsym{())\ \{}\\
\hlline{35\ }\hlstd{}\hlstd{\ \ \ }\hlstd{}\hlkwd{assert}\hlstd{}\hlsym{(}\hlstd{}\hlstr{'0�}\hlstd{\ }\hlsym{$<$=\ }\hlstd{s}\hlsym{{[}}\hlstd{i}\hlsym{{]}\ \&\&\ }\hlstd{s}\hlsym{{[}}\hlstd{i}\hlsym{{]}\ $<$=\ }\hlstd{}\hlstr{`9�}\hlstd{}\hlsym{);}\\
\hlline{36\ }\hlstd{}\hlstd{\ \ \ }\hlstd{digit\ }\hlsym{+=\ }\hlstd{pow\ }\hlsym{{*}\ (}\hlstd{s}\hlsym{{[}}\hlstd{i}\hlsym{{]}\ {-}\ }\hlstd{}\hlstr{'0�}\hlstd{}\hlsym{);}\\
\hlline{37\ }\hlstd{\\
\hlline{38\ }}\hlstd{\ \ \ }\hlstd{pow\ }\hlsym{{*}=\ }\hlstd{}\hlnum{10}\hlstd{}\hlsym{;}\\
\hlline{39\ }\hlstd{}\hlstd{\ \ \ }\hlstd{}\hlkwa{if\ }\hlstd{}\hlsym{(}\hlstd{pow\ }\hlsym{==\ }\hlstd{BigInt\textunderscore BASE}\hlsym{)\ \{}\\
\hlline{40\ }\hlstd{}\hlstd{\ \ \ \ }\hlstd{\textunderscore digits}\hlsym{{[}}\hlstd{j}\hlsym{++{]}\ =\ }\hlstd{digit}\hlsym{;}\\
\hlline{41\ }\hlstd{}\hlstd{\ \ \ \ }\hlstd{digit\ }\hlsym{=\ }\hlstd{}\hlnum{0}\hlstd{}\hlsym{;}\\
\hlline{42\ }\hlstd{}\hlstd{\ \ \ \ }\hlstd{pow\ }\hlsym{=\ }\hlstd{}\hlnum{1}\hlstd{}\hlsym{;}\\
\hlline{43\ }\hlstd{}\hlstd{\ \ \ }\hlstd{}\hlsym{\}}\\
\hlline{44\ }\hlstd{}\hlstd{\ \ }\hlstd{}\hlsym{\}}\\
\hlline{45\ }\hlstd{}\hlstd{\ \ }\hlstd{}\hlkwd{assert}\hlstd{}\hlsym{(}\hlstd{j\ }\hlsym{$<$\ }\hlstd{BigInt\textunderscore NDIGITS\ }\hlsym{\textbar \textbar \ }\hlstd{digit\ }\hlsym{==\ }\hlstd{}\hlnum{0}\hlstd{}\hlsym{);}\\
\hlline{46\ }\hlstd{}\hlstd{\ \ }\hlstd{}\hlkwa{if\ }\hlstd{}\hlsym{(}\hlstd{j\ }\hlsym{$<$\ }\hlstd{BigInt\textunderscore NDIGITS}\hlsym{)\ \{}\\
\hlline{47\ }\hlstd{}\hlstd{\ \ \ }\hlstd{\textunderscore digits}\hlsym{{[}}\hlstd{j}\hlsym{{]}\ =\ }\hlstd{digit}\hlsym{;}\\
\hlline{48\ }\hlstd{}\hlstd{\ \ }\hlstd{}\hlsym{\}}\\
\hlline{49\ }\hlstd{\ }\hlsym{\}}\\
\hlline{50\ }\hlstd{\\
\hlline{51\ }\ BigInt\ }\hlkwc{operator}\hlstd{}\hlsym{+(}\hlstd{}\hlkwb{const\ }\hlstd{BigInt}\hlsym{\&\ }\hlstd{n}\hlsym{)\ \{}\\
\hlline{52\ }\hlstd{}\hlstd{\ \ }\hlstd{BigInt\ res}\hlsym{;}\\
\hlline{53\ }\hlstd{}\hlstd{\ \ }\hlstd{}\hlkwb{bool\ }\hlstd{carry\ }\hlsym{=\ }\hlstd{}\hlnum{0}\hlstd{}\hlsym{;}\\
\hlline{54\ }\hlstd{\\
\hlline{55\ }}\hlstd{\ \ }\hlstd{}\hlkwd{forn\ }\hlstd{}\hlsym{(}\hlstd{i}\hlsym{,\ }\hlstd{BigInt\textunderscore NDIGITS}\hlsym{)\ \{}\\
\hlline{56\ }\hlstd{}\hlstd{\ \ \ }\hlstd{}\hlkwb{const\ }\hlstd{uint32\ r\ }\hlsym{=\ }\hlstd{\textunderscore digits}\hlsym{{[}}\hlstd{i}\hlsym{{]}\ +\ }\hlstd{n}\hlsym{.}\hlstd{\textunderscore digits}\hlsym{{[}}\hlstd{i}\hlsym{{]}\ +\ }\hlstd{carry}\hlsym{;}\\
\hlline{57\ }\hlstd{}\hlstd{\ \ \ }\hlstd{}\hlkwa{if\ }\hlstd{}\hlsym{(}\hlstd{r\ }\hlsym{$>$=\ }\hlstd{BigInt\textunderscore BASE}\hlsym{)\ \{}\\
\hlline{58\ }\hlstd{}\hlstd{\ \ \ \ }\hlstd{res}\hlsym{.}\hlstd{\textunderscore digits}\hlsym{{[}}\hlstd{i}\hlsym{{]}\ =\ }\hlstd{r\ }\hlsym{\%\ }\hlstd{BigInt\textunderscore BASE}\hlsym{;}\\
\hlline{59\ }\hlstd{}\hlstd{\ \ \ \ }\hlstd{carry\ }\hlsym{=\ }\hlstd{}\hlnum{1}\hlstd{}\hlsym{;}\\
\hlline{60\ }\hlstd{}\hlstd{\ \ \ }\hlstd{}\hlsym{\}\ }\hlstd{}\hlkwa{else\ }\hlstd{}\hlsym{\{}\\
\hlline{61\ }\hlstd{}\hlstd{\ \ \ \ }\hlstd{res}\hlsym{.}\hlstd{\textunderscore digits}\hlsym{{[}}\hlstd{i}\hlsym{{]}\ =\ }\hlstd{r}\hlsym{;}\\
\hlline{62\ }\hlstd{}\hlstd{\ \ \ \ }\hlstd{carry\ }\hlsym{=\ }\hlstd{}\hlnum{0}\hlstd{}\hlsym{;}\\
\hlline{63\ }\hlstd{}\hlstd{\ \ \ }\hlstd{}\hlsym{\}}\\
\hlline{64\ }\hlstd{}\hlstd{\ \ }\hlstd{}\hlsym{\}}\\
\hlline{65\ }\hlstd{}\hlstd{\ \ }\hlstd{}\hlkwd{assert}\hlstd{}\hlsym{(}\hlstd{carry\ }\hlsym{==\ }\hlstd{}\hlnum{0}\hlstd{}\hlsym{);}\\
\hlline{66\ }\hlstd{}\hlstd{\ \ }\hlstd{}\hlkwa{return\ }\hlstd{res}\hlsym{;}\\
\hlline{67\ }\hlstd{\ }\hlsym{\}}\\
\hlline{68\ }\hlstd{\\
\hlline{69\ }\ }\hlkwb{void\ }\hlstd{}\hlkwd{print}\hlstd{}\hlsym{()\ \{}\\
\hlline{70\ }\hlstd{}\hlstd{\ \ }\hlstd{}\hlkwb{bool\ }\hlstd{zero\ }\hlsym{=\ }\hlstd{}\hlkwa{true}\hlstd{}\hlsym{;\ }\\
\hlline{71\ }\hlstd{}\hlstd{\ \ }\hlstd{}\hlkwd{dforn\ }\hlstd{}\hlsym{(}\hlstd{i}\hlsym{,\ }\hlstd{BigInt\textunderscore NDIGITS}\hlsym{)\ \{}\\
\hlline{72\ }\hlstd{}\hlstd{\ \ \ }\hlstd{}\hlkwa{if\ }\hlstd{}\hlsym{(}\hlstd{zero}\hlsym{)\ \{}\\
\hlline{73\ }\hlstd{}\hlstd{\ \ \ \ }\hlstd{}\hlkwa{if\ }\hlstd{}\hlsym{(}\hlstd{\textunderscore digits}\hlsym{{[}}\hlstd{i}\hlsym{{]}\ ==\ }\hlstd{}\hlnum{0}\hlstd{}\hlsym{)\ }\hlstd{}\hlkwa{continue}\hlstd{}\hlsym{;}\\
\hlline{74\ }\hlstd{}\hlstd{\ \ \ \ }\hlstd{}\hlkwd{printf}\hlstd{}\hlsym{(}\hlstd{}\hlstr{"\%u"}\hlstd{}\hlsym{,\ }\hlstd{\textunderscore digits}\hlsym{{[}}\hlstd{i}\hlsym{{]});}\\
\hlline{75\ }\hlstd{}\hlstd{\ \ \ \ }\hlstd{zero\ }\hlsym{=\ }\hlstd{}\hlkwa{false}\hlstd{}\hlsym{;}\\
\hlline{76\ }\hlstd{}\hlstd{\ \ \ }\hlstd{}\hlsym{\}\ }\hlstd{}\hlkwa{else\ }\hlstd{}\hlsym{\{}\\
\hlline{77\ }\hlstd{}\hlstd{\ \ \ \ }\hlstd{}\hlkwd{printf}\hlstd{}\hlsym{(}\hlstd{}\hlstr{"\%.8u"}\hlstd{}\hlsym{,\ }\hlstd{\textunderscore digits}\hlsym{{[}}\hlstd{i}\hlsym{{]});}\\
\hlline{78\ }\hlstd{}\hlstd{\ \ \ }\hlstd{}\hlsym{\}}\\
\hlline{79\ }\hlstd{}\hlstd{\ \ }\hlstd{}\hlsym{\}}\\
\hlline{80\ }\hlstd{}\hlstd{\ \ }\hlstd{}\hlkwa{if\ }\hlstd{}\hlsym{(}\hlstd{zero}\hlsym{)\ \{}\\
\hlline{81\ }\hlstd{}\hlstd{\ \ \ }\hlstd{}\hlkwd{printf}\hlstd{}\hlsym{(}\hlstd{}\hlstr{"0"}\hlstd{}\hlsym{);}\\
\hlline{82\ }\hlstd{}\hlstd{\ \ }\hlstd{}\hlsym{\}}\\
\hlline{83\ }\hlstd{\ }\hlsym{\}}\\
\hlline{84\ }\hlstd{}\\
\hlline{85\ }\hlkwc{private}\hlstd{}\hlsym{:}\\
\hlline{86\ }\hlstd{\ }\hlslc{//\ Least\ significant\ digit\ first.}\\
\hlline{87\ }\hlstd{\ }\hlslc{//\ Digits\ are\ in\ 0\ $<$=\ d\ $<$\ BigInt\textunderscore BASE}\\
\hlline{88\ }\hlstd{\ vector}\hlsym{$<$}\hlstd{uint32}\hlsym{$>$\ }\hlstd{\textunderscore digits}\hlsym{;}\\
\hlline{89\ }\hlstd{}\hlsym{\};}\\
\hlline{90\ }\hlstd{}\\
\hlline{91\ }\hldir{\#define\ print\textunderscore bigint(x)\ (x).print()}\\
\hlline{92\ }\hlstd{}\\
\hlline{93\ }\hlkwc{typedef\ }\hlstd{vector}\hlsym{$<$}\hlstd{BigInt}\hlsym{$>$\ }\hlstd{vBigInt}\hlsym{;}\\
\hlline{94\ }\hlstd{}\hlkwc{typedef\ }\hlstd{vector}\hlsym{$<$}\hlstd{vBigInt}\hlsym{$>$\ }\hlstd{vvBigInt}\hlsym{;}\\
\hlline{95\ }\hlstd{BigInt\ }\hlkwd{distinct\textunderscore subsequences}\hlstd{}\hlsym{(}\hlstd{}\hlkwb{const\ }\hlstd{string}\hlsym{\&\ }\hlstd{x}\hlsym{,\ }\hlstd{}\hlkwb{const\ }\hlstd{string}\hlsym{\&\ }\hlstd{z}\hlsym{)\ \{}\\
\hlline{96\ }\hlstd{\ vvBigInt\ }\hlkwd{m}\hlstd{}\hlsym{(}\hlstd{}\hlnum{2}\hlstd{}\hlsym{,\ }\hlstd{}\hlkwd{vBigInt}\hlstd{}\hlsym{(}\hlstd{z}\hlsym{.}\hlstd{}\hlkwd{size}\hlstd{}\hlsym{()\ +\ }\hlstd{}\hlnum{1}\hlstd{}\hlsym{,\ }\hlstd{}\hlkwd{BigInt}\hlstd{}\hlsym{(}\hlstd{}\hlnum{0}\hlstd{}\hlsym{)));}\\
\hlline{97\ }\hlstd{\\
\hlline{98\ }\ m}\hlsym{{[}}\hlstd{}\hlnum{0}\hlstd{}\hlsym{{]}{[}}\hlstd{}\hlnum{0}\hlstd{}\hlsym{{]}\ =\ }\hlstd{}\hlkwd{BigInt}\hlstd{}\hlsym{(}\hlstd{}\hlnum{1}\hlstd{}\hlsym{);}\\
\hlline{99\ }\hlstd{\ m}\hlsym{{[}}\hlstd{}\hlnum{1}\hlstd{}\hlsym{{]}{[}}\hlstd{}\hlnum{0}\hlstd{}\hlsym{{]}\ =\ }\hlstd{}\hlkwd{BigInt}\hlstd{}\hlsym{(}\hlstd{}\hlnum{1}\hlstd{}\hlsym{);}\\
\hlline{100\ }\hlstd{\ }\hlkwb{bool\ }\hlstd{row\ }\hlsym{=\ }\hlstd{}\hlnum{1}\hlstd{}\hlsym{;}\\
\hlline{101\ }\hlstd{\ }\hlkwd{forsn\ }\hlstd{}\hlsym{(}\hlstd{i}\hlsym{,\ }\hlstd{}\hlnum{1}\hlstd{}\hlsym{,\ }\hlstd{x}\hlsym{.}\hlstd{}\hlkwd{size}\hlstd{}\hlsym{()\ +\ }\hlstd{}\hlnum{1}\hlstd{}\hlsym{)\ \{}\\
\hlline{102\ }\hlstd{}\hlstd{\ \ }\hlstd{}\hlkwd{forsn\ }\hlstd{}\hlsym{(}\hlstd{j}\hlsym{,\ }\hlstd{}\hlnum{1}\hlstd{}\hlsym{,\ }\hlstd{z}\hlsym{.}\hlstd{}\hlkwd{size}\hlstd{}\hlsym{()\ +\ }\hlstd{}\hlnum{1}\hlstd{}\hlsym{)\ \{}\\
\hlline{103\ }\hlstd{}\hlstd{\ \ \ }\hlstd{}\hlkwa{if\ }\hlstd{}\hlsym{(}\hlstd{x}\hlsym{{[}}\hlstd{i\ }\hlsym{{-}\ }\hlstd{}\hlnum{1}\hlstd{}\hlsym{{]}\ ==\ }\hlstd{z}\hlsym{{[}}\hlstd{j\ }\hlsym{{-}\ }\hlstd{}\hlnum{1}\hlstd{}\hlsym{{]})\ \{}\\
\hlline{104\ }\hlstd{}\hlstd{\ \ \ \ }\hlstd{m}\hlsym{{[}}\hlstd{row}\hlsym{{]}{[}}\hlstd{j}\hlsym{{]}\ =\ }\hlstd{m}\hlsym{{[}!}\hlstd{row}\hlsym{{]}{[}}\hlstd{j}\hlsym{{]}\ +\ }\hlstd{m}\hlsym{{[}!}\hlstd{row}\hlsym{{]}{[}}\hlstd{j\ }\hlsym{{-}\ }\hlstd{}\hlnum{1}\hlstd{}\hlsym{{]};}\\
\hlline{105\ }\hlstd{}\hlstd{\ \ \ }\hlstd{}\hlsym{\}\ }\hlstd{}\hlkwa{else\ }\hlstd{}\hlsym{\{}\\
\hlline{106\ }\hlstd{}\hlstd{\ \ \ \ }\hlstd{m}\hlsym{{[}}\hlstd{row}\hlsym{{]}{[}}\hlstd{j}\hlsym{{]}\ =\ }\hlstd{m}\hlsym{{[}!}\hlstd{row}\hlsym{{]}{[}}\hlstd{j}\hlsym{{]};}\\
\hlline{107\ }\hlstd{}\hlstd{\ \ \ }\hlstd{}\hlsym{\}}\\
\hlline{108\ }\hlstd{}\hlstd{\ \ }\hlstd{}\hlsym{\}}\\
\hlline{109\ }\hlstd{}\hlstd{\ \ }\hlstd{row\ }\hlsym{=\ !}\hlstd{row}\hlsym{;}\\
\hlline{110\ }\hlstd{\ }\hlsym{\}}\\
\hlline{111\ }\hlstd{\ }\hlkwa{return\ }\hlstd{m}\hlsym{{[}!}\hlstd{row}\hlsym{{]}{[}}\hlstd{z}\hlsym{.}\hlstd{}\hlkwd{size}\hlstd{}\hlsym{(){]};}\\
\hlline{112\ }\hlstd{}\hlsym{\}}\\
\hlline{113\ }\hlstd{}\\
\hlline{114\ }\hlkwb{int\ }\hlstd{}\hlkwd{main}\hlstd{}\hlsym{(}\hlstd{}\hlkwb{int\ }\hlstd{argc}\hlsym{,\ }\hlstd{}\hlkwb{char\ }\hlstd{}\hlsym{{*}{*}}\hlstd{argv}\hlsym{)\ \{}\\
\hlline{115\ }\hlstd{\ uint32\ n}\hlsym{;}\\
\hlline{116\ }\hlstd{\ cin\ }\hlsym{$>$$>$\ }\hlstd{n}\hlsym{;}\\
\hlline{117\ }\hlstd{\ }\hlkwb{char\ }\hlstd{c}\hlsym{;}\\
\hlline{118\ }\hlstd{\ cin}\hlsym{.}\hlstd{}\hlkwd{get}\hlstd{}\hlsym{(}\hlstd{c}\hlsym{);}\\
\hlline{119\ }\hlstd{\ }\hlkwd{forn\ }\hlstd{}\hlsym{(}\hlstd{i}\hlsym{,\ }\hlstd{n}\hlsym{)\ \{}\\
\hlline{120\ }\hlstd{}\hlstd{\ \ }\hlstd{string\ x}\hlsym{,\ }\hlstd{z}\hlsym{;}\\
\hlline{121\ }\hlstd{}\hlstd{\ \ }\hlstd{}\hlkwd{getline}\hlstd{}\hlsym{(}\hlstd{cin}\hlsym{,\ }\hlstd{x}\hlsym{);}\\
\hlline{122\ }\hlstd{}\hlstd{\ \ }\hlstd{}\hlkwd{getline}\hlstd{}\hlsym{(}\hlstd{cin}\hlsym{,\ }\hlstd{z}\hlsym{);}\\
\hlline{123\ }\hlstd{}\hlstd{\ \ }\hlstd{}\hlkwd{print\textunderscore bigint}\hlstd{}\hlsym{(}\hlstd{}\hlkwd{distinct\textunderscore subsequences}\hlstd{}\hlsym{(}\hlstd{x}\hlsym{,\ }\hlstd{z}\hlsym{));}\\
\hlline{124\ }\hlstd{}\hlstd{\ \ }\hlstd{cout\ }\hlsym{$<$$<$\ }\hlstd{endl}\hlsym{;}\\
\hlline{125\ }\hlstd{\ }\hlsym{\}}\\
\hlline{126\ }\hlstd{\ }\hlkwa{return\ }\hlstd{}\hlnum{0}\hlstd{}\hlsym{;}\\
\hlline{127\ }\hlstd{}\hlsym{\}}\\
\hlline{128\ }\hlstd{}\\
\mbox{}
\normalfont
\shorthandon{"}

