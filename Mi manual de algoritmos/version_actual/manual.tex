\documentclass[10pt,letterpaper,twocolumn,twosided]{article}
%\documentclass[10pt,letterpaper]{article}
% ---------------------------------------------------------------
\usepackage[utf8]{inputenc}
\usepackage[spanish]{babel}
\usepackage{listings}
\usepackage[usenames,dvipsnames]{color}
\usepackage{amsmath}
\usepackage{verbatim}
\usepackage{hyperref}
\usepackage{color}
\usepackage{geometry}
%Poner la página en landscape
\geometry{verbose,landscape,letterpaper,tmargin=2cm,bmargin=2cm,lmargin=2cm,rmargin=2cm}

\newcommand{\codigofuente}[1]{
\verbatiminput{#1}
\dotfill
}
\setlength{\columnsep}{0.25in}
\setlength{\columnseprule}{1px}

% ---------------------------------------------------------------
\begin{document}
% ---------------------------------------------------------------
\title{Resumen de algoritmos para torneos de programación}
\author{Andrés Mejía}
\date{\today}
\maketitle
% ---------------------------------------------------------------

% ---------------------------------------------------------------
\tableofcontents
% \lstlistoflistings
\lstloadlanguages{C++}
% ---------------------------------------------------------------
\section{Plantilla}
\codigofuente{./src/template.cpp}
% ---------------------------------------------------------------
\section{Teoría de números}
% ---------------------------------------------------------------
\subsection{Big mod}
%% Generator: GNU source-highlight, by Lorenzo Bettini, http://www.gnu.org/software/src-highlite

{\ttfamily \raggedright {
\noindent
\mbox{}\textit{\textcolor{Brown}{//retorna\ (b\textasciicircum{}p)mod(m)}} \\
\mbox{}\textit{\textcolor{Brown}{//\ 0\ $<$=\ b,p\ $<$=\ 2147483647}} \\
\mbox{}\textit{\textcolor{Brown}{//\ 1\ $<$=\ m\ $<$=\ 46340}} \\
\mbox{}\textcolor{ForestGreen}{long}\ \textbf{\textcolor{Black}{f}}\textcolor{BrickRed}{(}\textcolor{ForestGreen}{long}\ b\textcolor{BrickRed}{,}\ \textcolor{ForestGreen}{long}\ p\textcolor{BrickRed}{,}\ \textcolor{ForestGreen}{long}\ m\textcolor{BrickRed}{)}\textcolor{Red}{\{} \\
\mbox{}\ \ \textcolor{ForestGreen}{long}\ mask\ \textcolor{BrickRed}{=}\ \textcolor{Purple}{1}\textcolor{BrickRed}{;} \\
\mbox{}\ \ \textcolor{ForestGreen}{long}\ pow2\ \textcolor{BrickRed}{=}\ b\ \textcolor{BrickRed}{\%}\ m\textcolor{BrickRed}{;} \\
\mbox{}\ \ \textcolor{ForestGreen}{long}\ r\ \textcolor{BrickRed}{=}\ \textcolor{Purple}{1}\textcolor{BrickRed}{;} \\
\mbox{} \\
\mbox{}\ \ \textbf{\textcolor{Blue}{while}}\ \textcolor{BrickRed}{(}mask\textcolor{BrickRed}{)}\textcolor{Red}{\{} \\
\mbox{}\ \ \ \ \textbf{\textcolor{Blue}{if}}\ \textcolor{BrickRed}{(}p\ \textcolor{BrickRed}{\&}\ mask\textcolor{BrickRed}{)} \\
\mbox{}\ \ \ \ \ \ r\ \textcolor{BrickRed}{=}\ \textcolor{BrickRed}{(}r\ \textcolor{BrickRed}{*}\ pow2\textcolor{BrickRed}{)}\ \textcolor{BrickRed}{\%}\ m\textcolor{BrickRed}{;} \\
\mbox{}\ \ \ \ pow2\ \textcolor{BrickRed}{=}\ \textcolor{BrickRed}{(}pow2\textcolor{BrickRed}{*}pow2\textcolor{BrickRed}{)}\ \textcolor{BrickRed}{\%}\ m\textcolor{BrickRed}{;} \\
\mbox{}\ \ \ \ mask\ \textcolor{BrickRed}{$<$$<$=}\ \textcolor{Purple}{1}\textcolor{BrickRed}{;} \\
\mbox{}\ \ \textcolor{Red}{\}} \\
\mbox{}\ \ \textbf{\textcolor{Blue}{return}}\ r\textcolor{BrickRed}{;} \\
\mbox{}\textcolor{Red}{\}} \\

} \normalfont\normalsize
%.tex
\codigofuente{./src/number_theory/bigmod.cpp}

\subsection{Criba de Eratóstenes}
\small
\textbf{Field-testing:}
\begin{itemize}
\item \emph{SPOJ} -  2912 - Super Primes
\item \emph{Live Archive} - 3639 - Prime Path
\end{itemize}

\normalsize
Marca los números primos en un arreglo. Algunos tiempos de ejecución:
\begin{center}
  \begin{tabular}{c c}
    \hline\hline
    SIZE & Tiempo (s) \\ [0.5ex]
    \hline
    100000 & 0.003 \\
    1000000 & 0.060 \\
    10000000 & 0.620 \\
    100000000 & 7.650 \\ [1ex]
    \hline
  \end{tabular}
\end{center}
\codigofuente{./src/number_theory/criba.cpp}

\subsection{Divisores de un número}
Imprime todos los divisores de un número (en desorden) en O($\sqrt{n}$).
Hasta 4294967295 (máximo \textit{unsigned int}) responde instantáneamente. Se puede
forzar un poco más usando \textit{unsigned long long} pero más allá de $10^{12}$ empieza a
responder muy lento.
\codigofuente{./src/number_theory/divisores.cpp}

\section{Combinatoria}
\subsection{Cuadro resumen}
Fórmulas para combinaciones y permutaciones:
\begin{center}
  \renewcommand{\arraystretch}{2} %Multiplica la altura de cada fila de la tabla por 2
  % Si quiero aumentar el tamaño de una fila en particular insertar \rule{0cm}{1cm} en esa fila.
  \begin{tabular}{| c | c | c |}
    \hline
    \textit{Tipo} & \textit{¿Se permite la repetición?} & \textit{Fórmula} \\ [1.5ex]
    \hline\hline

    $r$-permutaciones & No & $ \displaystyle\frac{n!}{(n-r)!} $ \\ [1.5ex]
    \hline
    $r$-combinaciones & No & $ \displaystyle\frac{n!}{r!(n-r)!} $ \\  [1.5ex]
    \hline
    $r$-permutaciones & Sí & $ \displaystyle n^{r} $ \\
    \hline
    $r$-combinaciones & Sí & $ \displaystyle\frac{(n+r-1)!}{r!(n-1)!} $ \\ [1.5ex]
    \hline
  \end{tabular}
  \renewcommand{\arraystretch}{1}
\end{center}
Tomado de \textit{Matemática discreta y sus aplicaciones}, Kenneth Rosen, 5${}^{\hbox{ta}}$ edición, McGraw-Hill, página 315.

\subsection{Combinaciones, coeficientes binomiales, triángulo de Pascal}
\emph{Complejidad:} $ O(n^2) $ \\
$$ {n \choose k} = \left\{
  \begin{array}{c l}
    1 & k = 0\\
    1 & n = k\\
    \displaystyle {n - 1 \choose k - 1} + {n - 1 \choose k} & \mbox{en otro caso}\\
  \end{array}
\right.
$$

\codigofuente{./src/combinatoria/pascal_triangle.cpp}

\bigskip
\textbf{Nota:} $ \displaystyle {n \choose k }  $ está indefinido en el código anterior si $ n > k$. ¡La tabla puede estar llena con cualquier basura del compilador!

\subsection{Permutaciones con elementos indistinguibles}
El número de permutaciones \underline{diferentes} de $n$ objetos, donde hay $n_{1}$ objetos indistinguibles de tipo 1,
$n_{2}$ objetos indistinguibles de tipo 2, ..., y $n_{k}$ objetos indistinguibles de tipo $k$, es
$$
\frac{n!}{n_{1}!n_{2}! \cdots n_{k}!}
$$
\textbf{Ejemplo:} Con las letras de la palabra \texttt{PROGRAMAR} se pueden formar $ \displaystyle \frac{9!}{2! \cdot 3!} =
30240 $ permutaciones \underline{diferentes}.
\subsection{Desordenes, desarreglos o permutaciones completas}

Un desarreglo es una permutación donde ningún elemento $i$ está en la
posición $i$-ésima. Por ejemplo, \textit{4213} es un desarreglo de 4 elementos pero
\textit{3241} no lo es porque el 2 aparece en la posición 2.

Sea $D_{n}$ el número de desarreglos de $n$ elementos, entonces:
$$ {D_{n}} = \left\{
  \begin{array}{c l}
    1 & n = 0\\
    0 & n = 1\\
    (n-1)(D_{n-1} + D_{n-2}) & n \geq 2\\
  \end{array}
\right.
$$
Usando el principio de inclusión-exclusión, también se puede encontrar la fórmula
$$
D_{n} = n!\left [ 1 - \frac{1}{1!} + \frac{1}{2!} - \frac{1}{3!} + \cdots + (-1)^{n}\frac{1}{n!} \right ]
= n! \sum_{i=0}^{n} \frac{(-1)^{i}}{i!}
$$

\section{Grafos}
\subsection{Algoritmo de Dijkstra}
El peso de todas las aristas debe ser no negativo.
\\
\codigofuente{./src/grafos/dijkstra.cpp}

\subsection{Minimum spanning tree: Algoritmo de Prim}

\codigofuente{./src/grafos/prim.cpp}

\subsection{Minimum spanning tree: Algoritmo de Kruskal + Union-Find}
\codigofuente{./src/grafos/kruskal.cpp}

\subsection{Algoritmo de Floyd-Warshall}
\codigofuente{./src/grafos/floyd.cpp}

\subsection{Algoritmo de Bellman-Ford}
Si no hay ciclos de coste negativo, encuentra la distancia más corta
entre un nodo y todos los demás. Si sí hay, permite saberlo. \\ El
coste de las aristas \underline{sí} puede ser negativo
(\emph{Debería}, si no es así se puede usar Dijsktra o Floyd).
\codigofuente{./src/grafos/bellman.cpp}

\subsection{Puntos de articulación}
\codigofuente{./src/grafos/puntos_articulacion.cpp}

\subsection{Máximo flujo: Método de Ford-Fulkerson, algoritmo de Edmonds-Karp}
El algoritmo de Edmonds-Karp es una modificación al método de Ford-Fulkerson. Este último
utiliza DFS para hallar un camino de aumentación, pero la sugerencia de Edmonds-Karp
es utilizar BFS que lo hace más eficiente en algunos grafos.
\medskip

\codigofuente{./src/grafos/ford_fulkerson.cpp}

\subsection{Máximo flujo para grafos dispersos usando Ford-Fulkerson}
\codigofuente{./src/grafos/ford_fulkerson_sparse.cpp}

\subsection{Componentes fuertemente conexas: Algoritmo de Tarjan}
\label{tarjan}
\codigofuente{./src/grafos/tarjan.cpp}

\subsection{2-Satisfiability}
Dada una ecuación lógica de conjunciones de disyunciones de 2 términos, se pretente decidir si existen valores de verdad que puedan asignarse a las variables para hacer cierta la ecuación. \\
Por ejemplo, $(b_1 \vee \neg b_2) \wedge (b_2 \vee b_3) \wedge (\neg b_1 \vee \neg b_2) $ es verdadero cuando $b_1$ y $b_3$ son verdaderos y $b_2$ es falso. \\
\textbf{Solución:} Se sabe que $(p \rightarrow q) \leftrightarrow (\neg p \vee q)$. Entonces se traduce cada disyunción en una implicación y se crea un grafo donde los nodos son cada variable y su negación. Cada implicación es una arista en este grafo. Existe solución si nunca se cumple que una variable y su negación están en la misma componenete fuertemente conexa (Se usa el algoritmo de Tarjan, \ref{tarjan}).

\section{Programación dinámica}
\subsection{Longest common subsequence}
\codigofuente{./src/dp/lcs.cpp}
\subsection{Partición de troncos}
Este problema es similar al problema de \textit{Matrix Chain Multiplication}. Se tiene
un tronco de longitud $n$, y $m$ puntos de corte en el tronco. Se puede hacer un corte a la vez,
cuyo costo es igual a la longitud del tronco. ¿Cuál es el mínimo costo para partir todo el tronco
en pedacitos individuales?
\\
\medskip
\textbf{Ejemplo:} Se tiene un tronco de longitud 10. Los puntos de corte son 2, 4, y 7. El mínimo
costo para partirlo es 20, y se obtiene así:
\begin{itemize}
\item Partir el tronco (0, 10) por 4. Vale 10 y quedan los troncos (0, 4) y (4, 10).
\item Partir el tronco (0, 4) por 2. Vale 4 y quedan los troncos (0, 2), (2, 4) y (4, 10).
\item No hay que partir el tronco (0, 2).
\item No hay que partir el tronco (2, 4).
\item Partir el tronco (4, 10) por 7. Vale 6 y quedan los troncos (4, 7) y (7, 10).
\item No hay que partir el tronco (4, 7).
\item No hay que partir el tronco (7, 10).
\item El costo total es $10+4+6 = 20$.
\end{itemize}

\medskip
El algoritmo es $O(n^3)$, pero optimizable a $O(n^2)$ con una tabla adicional:
\codigofuente{./src/dp/particion_troncos.cpp}

\section{Geometría}
\subsection{Área de un polígono}
Si P es un polígono simple (no se intersecta a sí mismo) su área está dada por: \\

$ A(P) = \frac{1}{2} \displaystyle\sum_{i=0}^{n-1} (x_{i} \cdot y_{i+1} - x_{i+1} \cdot y_{i}) $ \\
\bigskip
\codigofuente{./src/geometria/polygon_area.cpp}

\subsection{Centro de masa de un polígono}
Si P es un polígono simple (no se intersecta a sí mismo) su centro de masa está dado por: \\

$ \displaystyle\bar{C}_{x} = \frac{ \displaystyle\iint_{R} x \, dA }{M} = \frac{1}{6M}\sum_{i=1}^{n} (y_{i+1} - y_{i}) (x_{i+1}^2 + x_{i+1} \cdot x_{i} + x_{i}^2) $

\medskip

$\displaystyle\bar{C}_{y} = \frac{ \displaystyle\iint_{R} y \, dA }{M} = \frac{1}{6M} \sum_{i=1}^{n} (x_{i} - x_{i+1}) (y_{i+1}^2 + y_{i+1} \cdot y_{i} + y_{i}^2)$

\medskip

Donde $ M $ es el área del polígono. \\

Otra posible fórmula equivalente:

$ \displaystyle\bar{C}_{x} = \frac{1}{6A} \sum_{i=0}^{n-1} (x_{i} + x_{i+1}) (x_{i} \cdot y_{i+1} - x_{i+1} \cdot y_{i}) $

\medskip

$ \displaystyle\bar{C}_{y} = \frac{1}{6A} \sum_{i=0}^{n-1} (y_{i} + y_{i+1}) (x_{i} \cdot y_{i+1} - x_{i+1} \cdot y_{i}) $


\subsection{Convex hull: Graham Scan}
\emph{Complejidad:} $ O(n \log_{2}{n}) $
\codigofuente{./src/geometria/grahamscan.cpp}

\subsection{Convex hull: Andrew's monotone chain}
\emph{Complejidad:} $ O(n \log_{2}{n}) $
\codigofuente{./src/geometria/monotonechain.cpp}

\subsection{Mínima distancia entre un punto y un segmento}
\codigofuente{./src/geometria/distance_point_to_segment.cpp}

\subsection{Mínima distancia entre un punto y una recta}
\codigofuente{./src/geometria/distance_point_to_line.cpp}

\subsection{Determinar si un polígono es convexo}
\codigofuente{./src/geometria/is_convex_polygon.cpp}

\subsection{Determinar si un punto está dentro de un polígono convexo}
\codigofuente{./src/geometria/is_inside_convex_polygon.cpp}

\subsection{Determinar si un punto está dentro de un polígono cualquiera}
\small
\textbf{Field-testing:}
\begin{itemize}
\item \emph{TopCoder} -  SRM 187 - Division 2 Hard - PointInPolygon
\end{itemize}
\codigofuente{./src/geometria/is_inside_concave_polygon.cpp}

\subsection{Intersección de dos rectas}
\codigofuente{./src/geometria/line_line_intersection.cpp}

\subsection{Intersección de dos segmentos de recta}
\codigofuente{./src/geometria/segment_segment_intersection.cpp}
% ---------------------------------------------------------------
\section{Estructuras de datos}
\subsection{Árboles de Fenwick ó Binary indexed trees}

Se tiene un arreglo $\{a_0, a_1, \cdots, a_{n-1}\}$. Los árboles
de Fenwick permiten encontrar $ \displaystyle \sum_{k=i}^{j} a_k $ en orden $O(\log_{2}{n})$ para parejas de $(i, j)$ con $i \leq j$. De la misma manera, permiten sumarle una cantidad a un $a_i$ también en tiempo $O(log_{2}{n})$.
\medskip
\codigofuente{./src/estructuras/fenwick.cpp}

\subsection{Segment tree}
\codigofuente{./src/estructuras/segment_tree.cpp}

% ---------------------------------------------------------------
\section{Misceláneo}
\subsection{El \textit{parser} más rápido del mundo}
\begin{itemize}
\item Cada no-terminal: un método
\item Cada lado derecho:
  \begin{itemize}
  \item invocar los métodos de los no-terminales o
  \item Cada terminal: invocar proceso \textit{match}
  \end{itemize}
\item Alternativas en una producción: se hace un \textit{if}
\end{itemize}
\medskip
No funciona con gramáticas recursivas por izquierda ó en las que en algún momento haya
varias posibles escogencias que empiezan por el mismo caracter (En ambos casos la gramática se puede factorizar).

\medskip
\textbf{Ejemplo:} Para la gramática:
$$
A \longrightarrow (A)A
$$ $$
A \longrightarrow \epsilon
$$

\codigofuente{./src/misc/parser_recursivo_desc.cpp}

\subsection{Checklist para corregir un Wrong Answer}
Consideraciones que podrían ser causa de un Wrong Answer:
\begin{itemize}
  \begin{item}
    Overflow.
  \end{item}

  \begin{item}
    El programa termina anticipadamente por la condición en el ciclo de lectura.
    Por ejemplo, se tiene \verb_while (cin >> n >> k && n && k)_ y un caso válido de entrada
    es n = 1 y k = 0.
  \end{item}

  \begin{item}
    El grafo no es conexo.
  \end{item}

  \begin{item}
    Puede haber varias aristas entre el mismo par de nodos.
  \end{item}

  \begin{item}
    Las aristas pueden tener costos negativos.
  \end{item}

  \begin{item}
    El grafo tiene un sólo nodo.
  \end{item}

  \begin{item}
    La cadena puede ser vacía.
  \end{item}

  \begin{item}
    Las líneas pueden tener espacios en blanco al principio o al final (Cuidado al usar \texttt{getline} o \texttt{fgets}).
  \end{item}

  \begin{item}
    El arreglo no se limpia entre caso y caso.
  \end{item}

  \begin{item}
    Estás imprimiendo una línea en blanco con un espacio (\verb_printf(" \n")_ en vez de \verb_printf("\n")_  ó \verb_puts(" ")_  en vez de \verb_puts("")_).
  \end{item}

  \begin{item}
    Hay pérdida de precisión al leer variables como \verb_double_ y converterirlas a enteros. Por ejemplo, en C++ pasa que \verb_floor(0.29 * 100) == 28_.

  \begin{item}
    La rana se puede quedar quieta.
  \end{item}


\end{itemize}


\section{Java}
\subsection{Entrada desde entrada estándar}
Este primer método es muy fácil pero es mucho más ineficiente porque utiliza Scanner en vez de BufferedReader:
\codigofuente{./src/java/io_estandar_easy.java}

\bigskip

Este segundo es más rápido:
\codigofuente{./src/java/io_estandar.java}
\subsection{Entrada desde archivo}
\codigofuente{./src/java/io_file.java}

\subsection{Mapas y sets}
Programa de ejemplo:
\codigofuente{./src/java/maps_sets.java} \\
La salida de este programa es: \\
\bigskip
\ttfamily
\fbox{\parbox{2.0in}{
    Maps\\
    m.size() = 1\\
    465\\
    null\\
    \\
    Sets\\
    54 presente.\\
    s.size() = 3\\
    54\\
    3576\\
    1000000007\\
    s.size() = 0\\
  }
}
\\ \normalfont\normalsize
\bigskip

Si quiere usarse una clase propia como llave del mapa o como elemento del set, la clase debe implementar
algunos métodos especiales: Si va a usarse un TreeMap ó TreeSet hay que implementar los métodos \texttt{compareTo} y
\texttt{equals} de la interfaz \texttt{Comparable} como en la sección \ref{colas_de_prioridad_java}. Si va a usarse
un HashMap ó HashSet hay más complicaciones.\\
\smallskip
\textbf{Sugerencia:} Inventar una manera de codificar y decodificar la clase en una String o un Integer y meter esa representación en el mapa o set: esas clases ya tienen los métodos implementados.

\subsection{Colas de prioridad}
\label{colas_de_prioridad_java}
Hay que implementar unos métodos. Veamos un ejemplo: \\
\codigofuente{./src/java/priority_queue.java}\\
La salida de este programa es: \\

\ttfamily
\fbox{\parbox{2.0in}{
    peso = 0, destino = 12\\
    peso = 0, destino = 8\\
    peso = 0, destino = 13\\
    peso = 3, destino = 7\\
    peso = 1876, destino = 4\\
  }
}
 \normalfont\normalsize

Vemos que la función de comparación que definimos no tiene en cuenta \texttt{destino},
por eso no desempata cuando dos \texttt{Items} tienen el mismo \texttt{peso} si no que escoge
cualquiera de manera arbitraria.

\section{C++}
\subsection{Entrada desde archivo}
\codigofuente{./src/c++/io_file.cpp}

\subsection{Strings con caractéres especiales}
\codigofuente{./src/c++/unicode.cpp}

\emph{Nota}: Como alternativa a la función getline, se pueden utilizar las funciones fgetws y fputws, y más adelante swscanf y wprintf:
\codigofuente{./src/c++/fgetws.cpp}

\end{document}