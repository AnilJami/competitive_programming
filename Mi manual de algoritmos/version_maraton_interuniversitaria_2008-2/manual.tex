\documentclass[10pt,letterpaper]{article}

%---------------------------------------------------------------
\usepackage[utf8]{inputenc}
\usepackage[spanish]{babel}
\usepackage{listings}
\usepackage[usenames,dvipsnames]{color}
\usepackage{amsmath}
%\usepackage{color}
%---------------------------------------------------------------

\setlength{\topmargin}{-1.0in}
\setlength{\textheight}{9.5in} 
\setlength{\evensidemargin}{0.0in}
\setlength{\oddsidemargin}{0.0in}
\setlength{\textwidth}{6.5in} 

\begin{document}

%---------------------------------------------------------------
\title{Resumen de algoritmos para torneos de programación}
\author{Andrés Mejía}
\date{\today}
%\maketitle
%---------------------------------------------------------------

%---------------------------------------------------------------
\tableofcontents
%\lstlistoflistings
\lstloadlanguages{C++}
%---------------------------------------------------------------
%---------------------------------------------------------------
\section{Teoría de números}
%---------------------------------------------------------------
\subsection{Big mod}

% Generator: GNU source-highlight, by Lorenzo Bettini, http://www.gnu.org/software/src-highlite

{\ttfamily \raggedright {
\noindent
\mbox{}\textit{\textcolor{Brown}{//retorna\ (b\textasciicircum{}p)mod(m)}} \\
\mbox{}\textit{\textcolor{Brown}{//\ 0\ $<$=\ b,p\ $<$=\ 2147483647}} \\
\mbox{}\textit{\textcolor{Brown}{//\ 1\ $<$=\ m\ $<$=\ 46340}} \\
\mbox{}\textcolor{ForestGreen}{long}\ \textbf{\textcolor{Black}{f}}\textcolor{BrickRed}{(}\textcolor{ForestGreen}{long}\ b\textcolor{BrickRed}{,}\ \textcolor{ForestGreen}{long}\ p\textcolor{BrickRed}{,}\ \textcolor{ForestGreen}{long}\ m\textcolor{BrickRed}{)}\textcolor{Red}{\{} \\
\mbox{}\ \ \textcolor{ForestGreen}{long}\ mask\ \textcolor{BrickRed}{=}\ \textcolor{Purple}{1}\textcolor{BrickRed}{;} \\
\mbox{}\ \ \textcolor{ForestGreen}{long}\ pow2\ \textcolor{BrickRed}{=}\ b\ \textcolor{BrickRed}{\%}\ m\textcolor{BrickRed}{;} \\
\mbox{}\ \ \textcolor{ForestGreen}{long}\ r\ \textcolor{BrickRed}{=}\ \textcolor{Purple}{1}\textcolor{BrickRed}{;} \\
\mbox{} \\
\mbox{}\ \ \textbf{\textcolor{Blue}{while}}\ \textcolor{BrickRed}{(}mask\textcolor{BrickRed}{)}\textcolor{Red}{\{} \\
\mbox{}\ \ \ \ \textbf{\textcolor{Blue}{if}}\ \textcolor{BrickRed}{(}p\ \textcolor{BrickRed}{\&}\ mask\textcolor{BrickRed}{)} \\
\mbox{}\ \ \ \ \ \ r\ \textcolor{BrickRed}{=}\ \textcolor{BrickRed}{(}r\ \textcolor{BrickRed}{*}\ pow2\textcolor{BrickRed}{)}\ \textcolor{BrickRed}{\%}\ m\textcolor{BrickRed}{;} \\
\mbox{}\ \ \ \ pow2\ \textcolor{BrickRed}{=}\ \textcolor{BrickRed}{(}pow2\textcolor{BrickRed}{*}pow2\textcolor{BrickRed}{)}\ \textcolor{BrickRed}{\%}\ m\textcolor{BrickRed}{;} \\
\mbox{}\ \ \ \ mask\ \textcolor{BrickRed}{$<$$<$=}\ \textcolor{Purple}{1}\textcolor{BrickRed}{;} \\
\mbox{}\ \ \textcolor{Red}{\}} \\
\mbox{}\ \ \textbf{\textcolor{Blue}{return}}\ r\textcolor{BrickRed}{;} \\
\mbox{}\textcolor{Red}{\}} \\

} \normalfont\normalsize
%.tex

\subsection{Criba de Eratóstenes}
Marca los números primos en un arreglo. Algunos tiempos de ejecución:
\begin{center}
\begin{tabular}{c c}
\hline\hline
SIZE & Tiempo (s) \\ [0.5ex]
\hline
100000 & 0.004 \\
1000000 & 0.078 \\
10000000 & 1.550 \\
100000000 & 14.319 \\ [1ex]
\hline
\end{tabular}
\end{center}

% Generator: GNU source-highlight, by Lorenzo Bettini, http://www.gnu.org/software/src-highlite
\noindent
\mbox{}\textbf{\textcolor{RoyalBlue}{\#include}}\ \texttt{\textcolor{Red}{$<$iostream$>$}} \\
\mbox{} \\
\mbox{}\textbf{\textcolor{Blue}{const}}\ \textcolor{ForestGreen}{int}\ SIZE\ \textcolor{BrickRed}{=}\ \textcolor{Purple}{1000000}\textcolor{BrickRed}{;} \\
\mbox{} \\
\mbox{}\textit{\textcolor{Brown}{//criba[i]\ =\ false\ si\ i\ es\ primo}} \\
\mbox{}\textcolor{ForestGreen}{bool}\ criba\textcolor{BrickRed}{[}SIZE\textcolor{BrickRed}{+}\textcolor{Purple}{1}\textcolor{BrickRed}{];} \\
\mbox{} \\
\mbox{}\textcolor{ForestGreen}{void}\ \textbf{\textcolor{Black}{buildCriba}}\textcolor{BrickRed}{()}\textcolor{Red}{\{} \\
\mbox{}\ \ \textbf{\textcolor{Black}{memset}}\textcolor{BrickRed}{(}criba\textcolor{BrickRed}{,}\ \textbf{\textcolor{Blue}{false}}\textcolor{BrickRed}{,}\ \textbf{\textcolor{Blue}{sizeof}}\textcolor{BrickRed}{(}criba\textcolor{BrickRed}{));} \\
\mbox{} \\
\mbox{}\ \ criba\textcolor{BrickRed}{[}\textcolor{Purple}{0}\textcolor{BrickRed}{]}\ \textcolor{BrickRed}{=}\ criba\textcolor{BrickRed}{[}\textcolor{Purple}{1}\textcolor{BrickRed}{]}\ \textcolor{BrickRed}{=}\ \textbf{\textcolor{Blue}{true}}\textcolor{BrickRed}{;} \\
\mbox{}\ \ \textbf{\textcolor{Blue}{for}}\ \textcolor{BrickRed}{(}\textcolor{ForestGreen}{int}\ i\textcolor{BrickRed}{=}\textcolor{Purple}{2}\textcolor{BrickRed}{;}\ i\textcolor{BrickRed}{$<$=}SIZE\textcolor{BrickRed}{;}\ i\ \textcolor{BrickRed}{+=}\ \textcolor{Purple}{2}\textcolor{BrickRed}{)}\textcolor{Red}{\{} \\
\mbox{}\ \ \ \ criba\textcolor{BrickRed}{[}i\textcolor{BrickRed}{]}\ \textcolor{BrickRed}{=}\ \textbf{\textcolor{Blue}{true}}\textcolor{BrickRed}{;} \\
\mbox{}\ \ \textcolor{Red}{\}} \\
\mbox{}\ \  \\
\mbox{}\ \ \textbf{\textcolor{Blue}{for}}\ \textcolor{BrickRed}{(}\textcolor{ForestGreen}{int}\ i\textcolor{BrickRed}{=}\textcolor{Purple}{3}\textcolor{BrickRed}{;}\ i\textcolor{BrickRed}{$<$=}SIZE\textcolor{BrickRed}{;}\ i\ \textcolor{BrickRed}{+=}\ \textcolor{Purple}{2}\textcolor{BrickRed}{)}\textcolor{Red}{\{} \\
\mbox{}\ \ \ \ \textbf{\textcolor{Blue}{if}}\ \textcolor{BrickRed}{(!}criba\textcolor{BrickRed}{[}i\textcolor{BrickRed}{])}\textcolor{Red}{\{} \\
\mbox{}\ \ \ \ \ \ \textbf{\textcolor{Blue}{for}}\ \textcolor{BrickRed}{(}\textcolor{ForestGreen}{int}\ j\textcolor{BrickRed}{=}i\textcolor{BrickRed}{+}i\textcolor{BrickRed}{;}\ j\textcolor{BrickRed}{$<$=}SIZE\textcolor{BrickRed}{;}\ j\ \textcolor{BrickRed}{+=}\ i\textcolor{BrickRed}{)}\textcolor{Red}{\{} \\
\mbox{}\ \ \ \ \ \ \ \ criba\textcolor{BrickRed}{[}j\textcolor{BrickRed}{]}\ \textcolor{BrickRed}{=}\ \textbf{\textcolor{Blue}{true}}\textcolor{BrickRed}{;} \\
\mbox{}\ \ \ \ \ \ \textcolor{Red}{\}} \\
\mbox{}\ \ \ \ \textcolor{Red}{\}} \\
\mbox{}\ \ \textcolor{Red}{\}} \\
\mbox{}\textcolor{Red}{\}} \\

%.tex

\subsection{Divisores de un número}
Este algoritmo imprime todos los divisores de un número (en desorden) en O($\sqrt{n}$).
Hasta 4294967295 (máximo \textit{unsigned long}) responde instantaneamente. Se puede
forzar un poco más usando \textit{unsigned long long} pero más allá de $10^{12}$ empieza a
responder muy lento.

\bigskip

% Generator: GNU source-highlight, by Lorenzo Bettini, http://www.gnu.org/software/src-highlite

{\ttfamily \raggedright {
\noindent
\mbox{}\textbf{\textcolor{Blue}{for}}\ \textcolor{BrickRed}{(}\textcolor{ForestGreen}{int}\ i\textcolor{BrickRed}{=}\textcolor{Purple}{1}\textcolor{BrickRed}{;}\ i\textcolor{BrickRed}{*}i\textcolor{BrickRed}{$<$=}n\textcolor{BrickRed}{;}\ i\textcolor{BrickRed}{++)}\ \textcolor{Red}{\{} \\
\mbox{}\ \ \textbf{\textcolor{Blue}{if}}\ \textcolor{BrickRed}{(}n\textcolor{BrickRed}{\%}i\ \textcolor{BrickRed}{==}\ \textcolor{Purple}{0}\textcolor{BrickRed}{)}\ \textcolor{Red}{\{} \\
\mbox{}\ \ \ \ cout\ \textcolor{BrickRed}{$<$$<$}\ i\ \textcolor{BrickRed}{$<$$<$}\ endl\textcolor{BrickRed}{;} \\
\mbox{}\ \ \ \ \textbf{\textcolor{Blue}{if}}\ \textcolor{BrickRed}{(}i\textcolor{BrickRed}{*}i\textcolor{BrickRed}{$<$}n\textcolor{BrickRed}{)}\ cout\ \textcolor{BrickRed}{$<$$<$}\ \textcolor{BrickRed}{(}n\textcolor{BrickRed}{/}i\textcolor{BrickRed}{)}\ \textcolor{BrickRed}{$<$$<$}\ endl\textcolor{BrickRed}{;} \\
\mbox{}\ \ \textcolor{Red}{\}} \\
\mbox{}\textcolor{Red}{\}}\  \\

} \normalfont\normalsize
%.tex


\section{Grafos}
\subsection{Algoritmo de Dijkstra}
El peso de todas las aristas debe ser no negativo.
\\
% Generator: GNU source-highlight, by Lorenzo Bettini, http://www.gnu.org/software/src-highlite

{\ttfamily \raggedright {
\noindent
\mbox{}\textbf{\textcolor{RoyalBlue}{\#include}}\ \texttt{\textcolor{Red}{$<$iostream$>$}} \\
\mbox{}\textbf{\textcolor{RoyalBlue}{\#include}}\ \texttt{\textcolor{Red}{$<$algorithm$>$}} \\
\mbox{}\textbf{\textcolor{RoyalBlue}{\#include}}\ \texttt{\textcolor{Red}{$<$queue$>$}} \\
\mbox{} \\
\mbox{}\textbf{\textcolor{Blue}{using}}\ \textbf{\textcolor{Blue}{namespace}}\ std\textcolor{BrickRed}{;} \\
\mbox{} \\
\mbox{}\textbf{\textcolor{Blue}{struct}}\ edge\textcolor{Red}{\{} \\
\mbox{}\ \ \textcolor{ForestGreen}{int}\ to\textcolor{BrickRed}{,}\ weight\textcolor{BrickRed}{;} \\
\mbox{}\ \ \textbf{\textcolor{Black}{edge}}\textcolor{BrickRed}{()}\ \textcolor{Red}{\{\}} \\
\mbox{}\ \ \textbf{\textcolor{Black}{edge}}\textcolor{BrickRed}{(}\textcolor{ForestGreen}{int}\ t\textcolor{BrickRed}{,}\ \textcolor{ForestGreen}{int}\ w\textcolor{BrickRed}{)}\ \textcolor{BrickRed}{:}\ \textbf{\textcolor{Black}{to}}\textcolor{BrickRed}{(}t\textcolor{BrickRed}{),}\ \textbf{\textcolor{Black}{weight}}\textcolor{BrickRed}{(}w\textcolor{BrickRed}{)}\ \textcolor{Red}{\{\}} \\
\mbox{}\ \ \textcolor{ForestGreen}{bool}\ \textbf{\textcolor{Blue}{operator}}\ \textcolor{BrickRed}{$<$}\ \textcolor{BrickRed}{(}\textbf{\textcolor{Blue}{const}}\ edge\ \textcolor{BrickRed}{\&}that\textcolor{BrickRed}{)}\ \textbf{\textcolor{Blue}{const}}\ \textcolor{Red}{\{} \\
\mbox{}\ \ \ \ \textbf{\textcolor{Blue}{return}}\ weight\ \textcolor{BrickRed}{$>$}\ that\textcolor{BrickRed}{.}weight\textcolor{BrickRed}{;} \\
\mbox{}\ \ \textcolor{Red}{\}} \\
\mbox{}\textcolor{Red}{\}}\textcolor{BrickRed}{;} \\
\mbox{} \\
\mbox{}\textcolor{ForestGreen}{int}\ \textbf{\textcolor{Black}{main}}\textcolor{BrickRed}{()}\textcolor{Red}{\{} \\
\mbox{}\ \ \textcolor{ForestGreen}{int}\ N\textcolor{BrickRed}{,}\ C\textcolor{BrickRed}{=}\textcolor{Purple}{0}\textcolor{BrickRed}{;} \\
\mbox{}\ \ \textbf{\textcolor{Black}{scanf}}\textcolor{BrickRed}{(}\texttt{\textcolor{Red}{"{}\%d"{}}}\textcolor{BrickRed}{,}\ \textcolor{BrickRed}{\&}N\textcolor{BrickRed}{);} \\
\mbox{}\ \ \textbf{\textcolor{Blue}{while}}\ \textcolor{BrickRed}{(}N\textcolor{BrickRed}{-\/-}\ \textcolor{BrickRed}{\&\&}\ \textcolor{BrickRed}{++}C\textcolor{BrickRed}{)}\textcolor{Red}{\{} \\
\mbox{}\ \ \ \ \textcolor{ForestGreen}{int}\ n\textcolor{BrickRed}{,}\ m\textcolor{BrickRed}{,}\ s\textcolor{BrickRed}{,}\ t\textcolor{BrickRed}{;} \\
\mbox{}\ \ \ \ \textbf{\textcolor{Black}{scanf}}\textcolor{BrickRed}{(}\texttt{\textcolor{Red}{"{}\%d\ \%d\ \%d\ \%d"{}}}\textcolor{BrickRed}{,}\ \textcolor{BrickRed}{\&}n\textcolor{BrickRed}{,}\ \textcolor{BrickRed}{\&}m\textcolor{BrickRed}{,}\ \textcolor{BrickRed}{\&}s\textcolor{BrickRed}{,}\ \textcolor{BrickRed}{\&}t\textcolor{BrickRed}{);} \\
\mbox{}\ \ \ \ vector\textcolor{BrickRed}{$<$}edge\textcolor{BrickRed}{$>$}\ g\textcolor{BrickRed}{[}n\textcolor{BrickRed}{];} \\
\mbox{}\ \ \ \ \textbf{\textcolor{Blue}{while}}\ \textcolor{BrickRed}{(}m\textcolor{BrickRed}{-\/-)}\textcolor{Red}{\{} \\
\mbox{}\ \ \ \ \ \ \textcolor{ForestGreen}{int}\ u\textcolor{BrickRed}{,}\ v\textcolor{BrickRed}{,}\ w\textcolor{BrickRed}{;} \\
\mbox{}\ \ \ \ \ \ \textbf{\textcolor{Black}{scanf}}\textcolor{BrickRed}{(}\texttt{\textcolor{Red}{"{}\%d\ \%d\ \%d"{}}}\textcolor{BrickRed}{,}\ \textcolor{BrickRed}{\&}u\textcolor{BrickRed}{,}\ \textcolor{BrickRed}{\&}v\textcolor{BrickRed}{,}\ \textcolor{BrickRed}{\&}w\textcolor{BrickRed}{);} \\
\mbox{}\ \ \ \ \ \ g\textcolor{BrickRed}{[}u\textcolor{BrickRed}{].}\textbf{\textcolor{Black}{push$\_$back}}\textcolor{BrickRed}{(}\textbf{\textcolor{Black}{edge}}\textcolor{BrickRed}{(}v\textcolor{BrickRed}{,}\ w\textcolor{BrickRed}{));} \\
\mbox{}\ \ \ \ \ \ g\textcolor{BrickRed}{[}v\textcolor{BrickRed}{].}\textbf{\textcolor{Black}{push$\_$back}}\textcolor{BrickRed}{(}\textbf{\textcolor{Black}{edge}}\textcolor{BrickRed}{(}u\textcolor{BrickRed}{,}\ w\textcolor{BrickRed}{));} \\
\mbox{}\ \ \ \ \textcolor{Red}{\}} \\
\mbox{} \\
\mbox{}\ \ \ \ \textcolor{ForestGreen}{int}\ d\textcolor{BrickRed}{[}n\textcolor{BrickRed}{];} \\
\mbox{}\ \ \ \ \textbf{\textcolor{Blue}{for}}\ \textcolor{BrickRed}{(}\textcolor{ForestGreen}{int}\ i\textcolor{BrickRed}{=}\textcolor{Purple}{0}\textcolor{BrickRed}{;}\ i\textcolor{BrickRed}{$<$}n\textcolor{BrickRed}{;}\ \textcolor{BrickRed}{++}i\textcolor{BrickRed}{)}\ d\textcolor{BrickRed}{[}i\textcolor{BrickRed}{]}\ \textcolor{BrickRed}{=}\ INT$\_$MAX\textcolor{BrickRed}{;} \\
\mbox{}\ \ \ \ d\textcolor{BrickRed}{[}s\textcolor{BrickRed}{]}\ \textcolor{BrickRed}{=}\ \textcolor{Purple}{0}\textcolor{BrickRed}{;} \\
\mbox{}\ \ \ \ priority$\_$queue\textcolor{BrickRed}{$<$}edge\textcolor{BrickRed}{$>$}\ q\textcolor{BrickRed}{;} \\
\mbox{}\ \ \ \ q\textcolor{BrickRed}{.}\textbf{\textcolor{Black}{push}}\textcolor{BrickRed}{(}\textbf{\textcolor{Black}{edge}}\textcolor{BrickRed}{(}s\textcolor{BrickRed}{,}\ \textcolor{Purple}{0}\textcolor{BrickRed}{));} \\
\mbox{}\ \ \ \ \textbf{\textcolor{Blue}{while}}\ \textcolor{BrickRed}{(}q\textcolor{BrickRed}{.}\textbf{\textcolor{Black}{empty}}\textcolor{BrickRed}{()}\ \textcolor{BrickRed}{==}\ \textbf{\textcolor{Blue}{false}}\textcolor{BrickRed}{)}\textcolor{Red}{\{} \\
\mbox{}\ \ \ \ \ \ \textcolor{ForestGreen}{int}\ node\ \textcolor{BrickRed}{=}\ q\textcolor{BrickRed}{.}\textbf{\textcolor{Black}{top}}\textcolor{BrickRed}{().}to\textcolor{BrickRed}{;} \\
\mbox{}\ \ \ \ \ \ \textcolor{ForestGreen}{int}\ dist\ \textcolor{BrickRed}{=}\ q\textcolor{BrickRed}{.}\textbf{\textcolor{Black}{top}}\textcolor{BrickRed}{().}weight\textcolor{BrickRed}{;} \\
\mbox{}\ \ \ \ \ \ q\textcolor{BrickRed}{.}\textbf{\textcolor{Black}{pop}}\textcolor{BrickRed}{();} \\
\mbox{} \\
\mbox{}\ \ \ \ \ \ \textbf{\textcolor{Blue}{if}}\ \textcolor{BrickRed}{(}dist\ \textcolor{BrickRed}{$>$}\ d\textcolor{BrickRed}{[}node\textcolor{BrickRed}{])}\ \textbf{\textcolor{Blue}{continue}}\textcolor{BrickRed}{;} \\
\mbox{}\ \ \ \ \ \ \textbf{\textcolor{Blue}{if}}\ \textcolor{BrickRed}{(}node\ \textcolor{BrickRed}{==}\ t\textcolor{BrickRed}{)}\ \textbf{\textcolor{Blue}{break}}\textcolor{BrickRed}{;} \\
\mbox{} \\
\mbox{}\ \ \ \ \ \ \textbf{\textcolor{Blue}{for}}\ \textcolor{BrickRed}{(}\textcolor{ForestGreen}{int}\ i\textcolor{BrickRed}{=}\textcolor{Purple}{0}\textcolor{BrickRed}{;}\ i\textcolor{BrickRed}{$<$}g\textcolor{BrickRed}{[}node\textcolor{BrickRed}{].}\textbf{\textcolor{Black}{size}}\textcolor{BrickRed}{();}\ \textcolor{BrickRed}{++}i\textcolor{BrickRed}{)}\textcolor{Red}{\{} \\
\mbox{}\ \ \ \ \ \ \ \ \textcolor{ForestGreen}{int}\ to\ \textcolor{BrickRed}{=}\ g\textcolor{BrickRed}{[}node\textcolor{BrickRed}{][}i\textcolor{BrickRed}{].}to\textcolor{BrickRed}{;} \\
\mbox{}\ \ \ \ \ \ \ \ \textcolor{ForestGreen}{int}\ w$\_$extra\ \textcolor{BrickRed}{=}\ g\textcolor{BrickRed}{[}node\textcolor{BrickRed}{][}i\textcolor{BrickRed}{].}weight\textcolor{BrickRed}{;} \\
\mbox{} \\
\mbox{}\ \ \ \ \ \ \ \ \textbf{\textcolor{Blue}{if}}\ \textcolor{BrickRed}{(}dist\ \textcolor{BrickRed}{+}\ w$\_$extra\ \textcolor{BrickRed}{$<$}\ d\textcolor{BrickRed}{[}to\textcolor{BrickRed}{])}\textcolor{Red}{\{} \\
\mbox{}\ \ \ \ \ \ \ \ \ \ d\textcolor{BrickRed}{[}to\textcolor{BrickRed}{]}\ \textcolor{BrickRed}{=}\ dist\ \textcolor{BrickRed}{+}\ w$\_$extra\textcolor{BrickRed}{;} \\
\mbox{}\ \ \ \ \ \ \ \ \ \ q\textcolor{BrickRed}{.}\textbf{\textcolor{Black}{push}}\textcolor{BrickRed}{(}\textbf{\textcolor{Black}{edge}}\textcolor{BrickRed}{(}to\textcolor{BrickRed}{,}\ d\textcolor{BrickRed}{[}to\textcolor{BrickRed}{]));} \\
\mbox{}\ \ \ \ \ \ \ \ \textcolor{Red}{\}} \\
\mbox{}\ \ \ \ \ \ \textcolor{Red}{\}} \\
\mbox{}\ \ \ \ \textcolor{Red}{\}} \\
\mbox{}\ \ \ \ \textbf{\textcolor{Black}{printf}}\textcolor{BrickRed}{(}\texttt{\textcolor{Red}{"{}Case\ \#\%d:\ "{}}}\textcolor{BrickRed}{,}\ C\textcolor{BrickRed}{);} \\
\mbox{}\ \ \ \ \textbf{\textcolor{Blue}{if}}\ \textcolor{BrickRed}{(}d\textcolor{BrickRed}{[}t\textcolor{BrickRed}{]}\ \textcolor{BrickRed}{$<$}\ INT$\_$MAX\textcolor{BrickRed}{)}\ \textbf{\textcolor{Black}{printf}}\textcolor{BrickRed}{(}\texttt{\textcolor{Red}{"{}\%d}}\texttt{\textcolor{CarnationPink}{\textbackslash{}n}}\texttt{\textcolor{Red}{"{}}}\textcolor{BrickRed}{,}\ d\textcolor{BrickRed}{[}t\textcolor{BrickRed}{]);} \\
\mbox{}\ \ \ \ \textbf{\textcolor{Blue}{else}}\ \textbf{\textcolor{Black}{printf}}\textcolor{BrickRed}{(}\texttt{\textcolor{Red}{"{}unreachable}}\texttt{\textcolor{CarnationPink}{\textbackslash{}n}}\texttt{\textcolor{Red}{"{}}}\textcolor{BrickRed}{);} \\
\mbox{}\ \ \textcolor{Red}{\}} \\
\mbox{}\ \ \textbf{\textcolor{Blue}{return}}\ \textcolor{Purple}{0}\textcolor{BrickRed}{;} \\
\mbox{}\textcolor{Red}{\}} \\

} \normalfont\normalsize
%.tex

\subsection{Minimum spanning tree: Algoritmo de Prim}

% Generator: GNU source-highlight, by Lorenzo Bettini, http://www.gnu.org/software/src-highlite
\noindent
\mbox{}\textbf{\textcolor{RoyalBlue}{\#include}}\ \texttt{\textcolor{Red}{$<$iostream$>$}} \\
\mbox{}\textbf{\textcolor{RoyalBlue}{\#include}}\ \texttt{\textcolor{Red}{$<$cmath$>$}} \\
\mbox{}\textbf{\textcolor{RoyalBlue}{\#include}}\ \texttt{\textcolor{Red}{$<$map$>$}} \\
\mbox{}\textbf{\textcolor{RoyalBlue}{\#include}}\ \texttt{\textcolor{Red}{$<$queue$>$}} \\
\mbox{}\textbf{\textcolor{RoyalBlue}{\#include}}\ \texttt{\textcolor{Red}{$<$set$>$}} \\
\mbox{} \\
\mbox{}\textbf{\textcolor{Blue}{using}}\ \textbf{\textcolor{Blue}{namespace}}\ std\textcolor{BrickRed}{;} \\
\mbox{} \\
\mbox{}\textbf{\textcolor{Blue}{typedef}}\ pair\textcolor{BrickRed}{$<$}\textcolor{ForestGreen}{double}\textcolor{BrickRed}{,}\ \textcolor{ForestGreen}{double}\textcolor{BrickRed}{$>$}\ point\textcolor{BrickRed}{;} \\
\mbox{}\textit{\textcolor{Brown}{//Gives\ a\ vector\ of\ adjacent\ nodes\ to\ a\ point}} \\
\mbox{}\textbf{\textcolor{Blue}{typedef}}\ map\textcolor{BrickRed}{$<$}\ point\textcolor{BrickRed}{,}\ vector\textcolor{BrickRed}{$<$}point\textcolor{BrickRed}{$>$}\ \textcolor{BrickRed}{$>$}\ graph\textcolor{BrickRed}{;} \\
\mbox{}\textit{\textcolor{Brown}{//Edge\ of\ length\ "{}first"{}\ that\ arrives\ to\ point\ "{}second"{}}} \\
\mbox{}\textbf{\textcolor{Blue}{typedef}}\ pair\textcolor{BrickRed}{$<$}\textcolor{ForestGreen}{double}\textcolor{BrickRed}{,}\ point\textcolor{BrickRed}{$>$}\ edge\textcolor{BrickRed}{;}\  \\
\mbox{} \\
\mbox{}\textcolor{ForestGreen}{double}\ \textbf{\textcolor{Black}{euclidean}}\textcolor{BrickRed}{(}\textbf{\textcolor{Blue}{const}}\ point\ \textcolor{BrickRed}{\&}a\textcolor{BrickRed}{,}\ \textbf{\textcolor{Blue}{const}}\ point\ \textcolor{BrickRed}{\&}b\textcolor{BrickRed}{)}\textcolor{Red}{\{}\ \textbf{\textcolor{Blue}{return}}\ \textbf{\textcolor{Black}{hypot}}\textcolor{BrickRed}{(}a\textcolor{BrickRed}{.}first\textcolor{BrickRed}{-}b\textcolor{BrickRed}{.}first\textcolor{BrickRed}{,}\ a\textcolor{BrickRed}{.}second\textcolor{BrickRed}{-}b\textcolor{BrickRed}{.}second\textcolor{BrickRed}{);}\textcolor{Red}{\}} \\
\mbox{} \\
\mbox{} \\
\mbox{}\textcolor{ForestGreen}{int}\ \textbf{\textcolor{Black}{main}}\textcolor{BrickRed}{()}\textcolor{Red}{\{} \\
\mbox{}\ \ \textcolor{ForestGreen}{int}\ casos\textcolor{BrickRed}{;} \\
\mbox{}\ \ cin\ \textcolor{BrickRed}{$>$$>$}\ casos\textcolor{BrickRed}{;} \\
\mbox{}\ \ \textbf{\textcolor{Blue}{while}}\ \textcolor{BrickRed}{(}casos\textcolor{BrickRed}{-\/-)}\textcolor{Red}{\{} \\
\mbox{}\ \ \ \ graph\ g\textcolor{BrickRed}{;} \\
\mbox{}\ \ \ \ \textcolor{ForestGreen}{int}\ n\textcolor{BrickRed}{;} \\
\mbox{}\ \ \ \ cin\ \textcolor{BrickRed}{$>$$>$}\ n\textcolor{BrickRed}{;} \\
\mbox{}\ \ \ \ \textbf{\textcolor{Blue}{while}}\ \textcolor{BrickRed}{(}n\textcolor{BrickRed}{-\/-)}\textcolor{Red}{\{} \\
\mbox{}\ \ \ \ \ \ \textcolor{ForestGreen}{double}\ x\textcolor{BrickRed}{,}y\textcolor{BrickRed}{;} \\
\mbox{}\ \ \ \ \ \ cin\ \textcolor{BrickRed}{$>$$>$}\ x\ \textcolor{BrickRed}{$>$$>$}\ y\textcolor{BrickRed}{;} \\
\mbox{}\ \ \ \ \ \ point\ \textbf{\textcolor{Black}{p}}\textcolor{BrickRed}{(}x\textcolor{BrickRed}{,}y\textcolor{BrickRed}{);} \\
\mbox{}\ \ \ \ \ \ \textbf{\textcolor{Blue}{if}}\ \textcolor{BrickRed}{(}g\textcolor{BrickRed}{.}\textbf{\textcolor{Black}{count}}\textcolor{BrickRed}{(}p\textcolor{BrickRed}{)}\ \textcolor{BrickRed}{==}\ \textcolor{Purple}{0}\textcolor{BrickRed}{)}\textcolor{Red}{\{}\ \textit{\textcolor{Brown}{//Si\ no\ está\ todavía}} \\
\mbox{}\ \ \ \ \ \ \ \ vector\textcolor{BrickRed}{$<$}point\textcolor{BrickRed}{$>$}\ v\textcolor{BrickRed}{;} \\
\mbox{}\ \ \ \ \ \ \ \ g\textcolor{BrickRed}{[}p\textcolor{BrickRed}{]}\ \textcolor{BrickRed}{=}\ v\textcolor{BrickRed}{;} \\
\mbox{}\ \ \ \ \ \ \ \ \textbf{\textcolor{Blue}{for}}\ \textcolor{BrickRed}{(}graph\textcolor{BrickRed}{::}iterator\ i\ \textcolor{BrickRed}{=}\ g\textcolor{BrickRed}{.}\textbf{\textcolor{Black}{begin}}\textcolor{BrickRed}{();}\ i\ \textcolor{BrickRed}{!=}\ g\textcolor{BrickRed}{.}\textbf{\textcolor{Black}{end}}\textcolor{BrickRed}{();}\ \textcolor{BrickRed}{++}i\textcolor{BrickRed}{)}\textcolor{Red}{\{} \\
\mbox{}\ \ \ \ \ \ \ \ \ \ \textbf{\textcolor{Blue}{if}}\ \textcolor{BrickRed}{((*}i\textcolor{BrickRed}{).}first\ \textcolor{BrickRed}{!=}\ p\textcolor{BrickRed}{)}\textcolor{Red}{\{} \\
\mbox{}\ \ \ \ \ \ \ \ \ \ \ \ \textcolor{BrickRed}{(*}i\textcolor{BrickRed}{).}second\textcolor{BrickRed}{.}\textbf{\textcolor{Black}{push$\_$back}}\textcolor{BrickRed}{(}p\textcolor{BrickRed}{);} \\
\mbox{}\ \ \ \ \ \ \ \ \ \ \ \ g\textcolor{BrickRed}{[}p\textcolor{BrickRed}{].}\textbf{\textcolor{Black}{push$\_$back}}\textcolor{BrickRed}{((*}i\textcolor{BrickRed}{).}first\textcolor{BrickRed}{);} \\
\mbox{}\ \ \ \ \ \ \ \ \ \ \textcolor{Red}{\}} \\
\mbox{}\ \ \ \ \ \ \ \ \textcolor{Red}{\}} \\
\mbox{}\ \ \ \ \ \ \textcolor{Red}{\}} \\
\mbox{}\ \ \ \ \textcolor{Red}{\}} \\
\mbox{} \\
\mbox{}\ \ \ \ set\textcolor{BrickRed}{$<$}point\textcolor{BrickRed}{$>$}\ visited\textcolor{BrickRed}{;} \\
\mbox{}\ \ \ \ priority$\_$queue\textcolor{BrickRed}{$<$}edge\textcolor{BrickRed}{,}\ vector\textcolor{BrickRed}{$<$}edge\textcolor{BrickRed}{$>$,}\ greater\textcolor{BrickRed}{$<$}edge\textcolor{BrickRed}{$>$}\ \textcolor{BrickRed}{$>$}\ q\textcolor{BrickRed}{;} \\
\mbox{}\ \ \ \ \textit{\textcolor{Brown}{//Each\ edge\ in\ q\ has\ got\ a\ length\ "{}first"{}\ and\ a\ point\ "{}second"{}.}} \\
\mbox{}\ \ \ \ \textit{\textcolor{Brown}{//It\ means\ I\ can\ reach\ point\ "{}second"{}\ which\ is\ "{}first"{}\ meters\ away.}} \\
\mbox{}\ \ \ \ \textit{\textcolor{Brown}{//q\ has\ the\ closest\ reachable\ node\ on\ top\ (I\ may\ have\ already\ visited\ it!)}} \\
\mbox{}\ \ \ \ q\textcolor{BrickRed}{.}\textbf{\textcolor{Black}{push}}\textcolor{BrickRed}{(}\textbf{\textcolor{Black}{edge}}\textcolor{BrickRed}{(}\textcolor{Purple}{0.0}\textcolor{BrickRed}{,}\ \textcolor{BrickRed}{(*}g\textcolor{BrickRed}{.}\textbf{\textcolor{Black}{begin}}\textcolor{BrickRed}{()).}first\textcolor{BrickRed}{));} \\
\mbox{}\ \ \ \ \textcolor{ForestGreen}{double}\ totalDistance\ \textcolor{BrickRed}{=}\ \textcolor{Purple}{0.0}\textcolor{BrickRed}{;} \\
\mbox{}\ \ \ \ \textbf{\textcolor{Blue}{while}}\ \textcolor{BrickRed}{(!}q\textcolor{BrickRed}{.}\textbf{\textcolor{Black}{empty}}\textcolor{BrickRed}{())}\textcolor{Red}{\{} \\
\mbox{}\ \ \ \ \ \ edge\ nearest\ \textcolor{BrickRed}{=}\ q\textcolor{BrickRed}{.}\textbf{\textcolor{Black}{top}}\textcolor{BrickRed}{();} \\
\mbox{}\ \ \ \ \ \ q\textcolor{BrickRed}{.}\textbf{\textcolor{Black}{pop}}\textcolor{BrickRed}{();} \\
\mbox{}\ \ \ \ \ \ point\ actualNode\ \textcolor{BrickRed}{=}\ nearest\textcolor{BrickRed}{.}second\textcolor{BrickRed}{;} \\
\mbox{}\ \ \ \ \ \ \textbf{\textcolor{Blue}{if}}\ \textcolor{BrickRed}{(}visited\textcolor{BrickRed}{.}\textbf{\textcolor{Black}{count}}\textcolor{BrickRed}{(}actualNode\textcolor{BrickRed}{)}\ \textcolor{BrickRed}{==}\ \textcolor{Purple}{1}\textcolor{BrickRed}{)}\ \textbf{\textcolor{Blue}{continue}}\textcolor{BrickRed}{;}\ \textit{\textcolor{Brown}{//Ya\ habia\ visitado\ este}} \\
\mbox{}\ \ \ \ \ \ totalDistance\ \textcolor{BrickRed}{+=}\ nearest\textcolor{BrickRed}{.}first\textcolor{BrickRed}{;} \\
\mbox{}\ \ \ \ \ \ visited\textcolor{BrickRed}{.}\textbf{\textcolor{Black}{insert}}\textcolor{BrickRed}{(}actualNode\textcolor{BrickRed}{);} \\
\mbox{}\ \ \ \ \ \ vector\textcolor{BrickRed}{$<$}point\textcolor{BrickRed}{$>$}\ neighbors\ \textcolor{BrickRed}{=}\ g\textcolor{BrickRed}{[}actualNode\textcolor{BrickRed}{];} \\
\mbox{}\ \ \ \ \ \ \textbf{\textcolor{Blue}{for}}\ \textcolor{BrickRed}{(}\textcolor{ForestGreen}{int}\ i\textcolor{BrickRed}{=}\textcolor{Purple}{0}\textcolor{BrickRed}{;}\ i\textcolor{BrickRed}{$<$}neighbors\textcolor{BrickRed}{.}\textbf{\textcolor{Black}{size}}\textcolor{BrickRed}{();}\ \textcolor{BrickRed}{++}i\textcolor{BrickRed}{)}\textcolor{Red}{\{} \\
\mbox{}\ \ \ \ \ \ \ \ point\ t\ \textcolor{BrickRed}{=}\ neighbors\textcolor{BrickRed}{[}i\textcolor{BrickRed}{];} \\
\mbox{}\ \ \ \ \ \ \ \ \textcolor{ForestGreen}{double}\ dist\ \textcolor{BrickRed}{=}\ \textbf{\textcolor{Black}{euclidean}}\textcolor{BrickRed}{(}actualNode\textcolor{BrickRed}{,}\ t\textcolor{BrickRed}{);} \\
\mbox{}\ \ \ \ \ \ \ \ q\textcolor{BrickRed}{.}\textbf{\textcolor{Black}{push}}\textcolor{BrickRed}{(}\textbf{\textcolor{Black}{edge}}\textcolor{BrickRed}{(}dist\textcolor{BrickRed}{,}\ t\textcolor{BrickRed}{));} \\
\mbox{}\ \ \ \ \ \ \textcolor{Red}{\}} \\
\mbox{}\ \ \ \ \textcolor{Red}{\}} \\
\mbox{}\ \ \ \ \textbf{\textcolor{Black}{printf}}\textcolor{BrickRed}{(}\texttt{\textcolor{Red}{"{}\%.2f}}\texttt{\textcolor{CarnationPink}{\textbackslash{}n}}\texttt{\textcolor{Red}{"{}}}\textcolor{BrickRed}{,}\ totalDistance\textcolor{BrickRed}{);} \\
\mbox{}\ \ \ \ \textbf{\textcolor{Blue}{if}}\ \textcolor{BrickRed}{(}casos\ \textcolor{BrickRed}{$>$}\ \textcolor{Purple}{0}\textcolor{BrickRed}{)}\ cout\ \textcolor{BrickRed}{$<$$<$}\ endl\textcolor{BrickRed}{;}\ \textit{\textcolor{Brown}{//Endl\ between\ cases}} \\
\mbox{}\ \ \textcolor{Red}{\}} \\
\mbox{}\textcolor{Red}{\}} \\

%.tex

\subsection{Minimum spanning tree: Algoritmo de Kruskal + Union-Find}
% Generator: GNU source-highlight, by Lorenzo Bettini, http://www.gnu.org/software/src-highlite

{\ttfamily \raggedright {
\noindent
\mbox{}\textbf{\textcolor{RoyalBlue}{\#include}}\ \texttt{\textcolor{Red}{$<$iostream$>$}} \\
\mbox{}\textbf{\textcolor{RoyalBlue}{\#include}}\ \texttt{\textcolor{Red}{$<$vector$>$}} \\
\mbox{}\textbf{\textcolor{RoyalBlue}{\#include}}\ \texttt{\textcolor{Red}{$<$algorithm$>$}} \\
\mbox{} \\
\mbox{}\textbf{\textcolor{Blue}{using}}\ \textbf{\textcolor{Blue}{namespace}}\ std\textcolor{BrickRed}{;} \\
\mbox{} \\
\mbox{}\textit{\textcolor{Brown}{/*}} \\
\mbox{}\textit{\textcolor{Brown}{Algoritmo\ de\ Kruskal,\ para\ encontrar\ el\ árbol\ de\ recubrimiento\ de\ mínima\ suma.}} \\
\mbox{}\textit{\textcolor{Brown}{*/}} \\
\mbox{} \\
\mbox{}\textbf{\textcolor{Blue}{struct}}\ edge\textcolor{Red}{\{} \\
\mbox{}\ \ \textcolor{ForestGreen}{int}\ start\textcolor{BrickRed}{,}\ end\textcolor{BrickRed}{,}\ weight\textcolor{BrickRed}{;} \\
\mbox{}\ \ \textcolor{ForestGreen}{bool}\ \textbf{\textcolor{Blue}{operator}}\ \textcolor{BrickRed}{$<$}\ \textcolor{BrickRed}{(}\textbf{\textcolor{Blue}{const}}\ edge\ \textcolor{BrickRed}{\&}that\textcolor{BrickRed}{)}\ \textbf{\textcolor{Blue}{const}}\ \textcolor{Red}{\{} \\
\mbox{}\ \ \ \ \textit{\textcolor{Brown}{//Si\ se\ desea\ encontrar\ el\ árbol\ de\ recubrimiento\ de\ máxima\ suma,\ cambiar\ el\ $<$\ por\ un\ $>$}} \\
\mbox{}\ \ \ \ \textbf{\textcolor{Blue}{return}}\ weight\ \textcolor{BrickRed}{$<$}\ that\textcolor{BrickRed}{.}weight\textcolor{BrickRed}{;} \\
\mbox{}\ \ \textcolor{Red}{\}} \\
\mbox{}\textcolor{Red}{\}}\textcolor{BrickRed}{;} \\
\mbox{} \\
\mbox{} \\
\mbox{}\textit{\textcolor{Brown}{/*\ Union\ find\ */}} \\
\mbox{}\textcolor{ForestGreen}{int}\ p\textcolor{BrickRed}{[}\textcolor{Purple}{10001}\textcolor{BrickRed}{],}\ rank\textcolor{BrickRed}{[}\textcolor{Purple}{10001}\textcolor{BrickRed}{];} \\
\mbox{}\textcolor{ForestGreen}{void}\ \textbf{\textcolor{Black}{make$\_$set}}\textcolor{BrickRed}{(}\textcolor{ForestGreen}{int}\ x\textcolor{BrickRed}{)}\textcolor{Red}{\{}\ p\textcolor{BrickRed}{[}x\textcolor{BrickRed}{]}\ \textcolor{BrickRed}{=}\ x\textcolor{BrickRed}{,}\ rank\textcolor{BrickRed}{[}x\textcolor{BrickRed}{]}\ \textcolor{BrickRed}{=}\ \textcolor{Purple}{0}\textcolor{BrickRed}{;}\ \textcolor{Red}{\}} \\
\mbox{}\textcolor{ForestGreen}{void}\ \textbf{\textcolor{Black}{link}}\textcolor{BrickRed}{(}\textcolor{ForestGreen}{int}\ x\textcolor{BrickRed}{,}\ \textcolor{ForestGreen}{int}\ y\textcolor{BrickRed}{)}\textcolor{Red}{\{}\ rank\textcolor{BrickRed}{[}x\textcolor{BrickRed}{]}\ \textcolor{BrickRed}{$>$}\ rank\textcolor{BrickRed}{[}y\textcolor{BrickRed}{]}\ \textcolor{BrickRed}{?}\ p\textcolor{BrickRed}{[}y\textcolor{BrickRed}{]}\ \textcolor{BrickRed}{=}\ x\ \textcolor{BrickRed}{:}\ p\textcolor{BrickRed}{[}x\textcolor{BrickRed}{]}\ \textcolor{BrickRed}{=}\ y\textcolor{BrickRed}{,}\ rank\textcolor{BrickRed}{[}x\textcolor{BrickRed}{]}\ \textcolor{BrickRed}{==}\ rank\textcolor{BrickRed}{[}y\textcolor{BrickRed}{]}\ \textcolor{BrickRed}{?}\ rank\textcolor{BrickRed}{[}y\textcolor{BrickRed}{]++:}\ \textcolor{Purple}{0}\textcolor{BrickRed}{;}\ \textcolor{Red}{\}} \\
\mbox{}\textcolor{ForestGreen}{int}\ \textbf{\textcolor{Black}{find$\_$set}}\textcolor{BrickRed}{(}\textcolor{ForestGreen}{int}\ x\textcolor{BrickRed}{)}\textcolor{Red}{\{}\ \textbf{\textcolor{Blue}{return}}\ x\ \textcolor{BrickRed}{!=}\ p\textcolor{BrickRed}{[}x\textcolor{BrickRed}{]}\ \textcolor{BrickRed}{?}\ p\textcolor{BrickRed}{[}x\textcolor{BrickRed}{]}\ \textcolor{BrickRed}{=}\ \textbf{\textcolor{Black}{find$\_$set}}\textcolor{BrickRed}{(}p\textcolor{BrickRed}{[}x\textcolor{BrickRed}{])}\ \textcolor{BrickRed}{:}\ p\textcolor{BrickRed}{[}x\textcolor{BrickRed}{];}\ \textcolor{Red}{\}} \\
\mbox{}\textcolor{ForestGreen}{void}\ \textbf{\textcolor{Black}{merge}}\textcolor{BrickRed}{(}\textcolor{ForestGreen}{int}\ x\textcolor{BrickRed}{,}\ \textcolor{ForestGreen}{int}\ y\textcolor{BrickRed}{)}\textcolor{Red}{\{}\ \textbf{\textcolor{Black}{link}}\textcolor{BrickRed}{(}\textbf{\textcolor{Black}{find$\_$set}}\textcolor{BrickRed}{(}x\textcolor{BrickRed}{),}\ \textbf{\textcolor{Black}{find$\_$set}}\textcolor{BrickRed}{(}y\textcolor{BrickRed}{));}\ \textcolor{Red}{\}} \\
\mbox{}\textit{\textcolor{Brown}{/*\ End\ union\ find\ */}} \\
\mbox{} \\
\mbox{} \\
\mbox{}\textcolor{ForestGreen}{int}\ \textbf{\textcolor{Black}{main}}\textcolor{BrickRed}{()}\textcolor{Red}{\{} \\
\mbox{}\ \ \textcolor{ForestGreen}{int}\ c\textcolor{BrickRed}{;} \\
\mbox{}\ \ cin\ \textcolor{BrickRed}{$>$$>$}\ c\textcolor{BrickRed}{;} \\
\mbox{}\ \ \textbf{\textcolor{Blue}{while}}\ \textcolor{BrickRed}{(}c\textcolor{BrickRed}{-\/-)}\textcolor{Red}{\{} \\
\mbox{}\ \ \ \ \textcolor{ForestGreen}{int}\ n\textcolor{BrickRed}{,}\ m\textcolor{BrickRed}{;} \\
\mbox{}\ \ \ \ cin\ \textcolor{BrickRed}{$>$$>$}\ n\ \textcolor{BrickRed}{$>$$>$}\ m\textcolor{BrickRed}{;} \\
\mbox{}\ \ \ \ vector\textcolor{BrickRed}{$<$}edge\textcolor{BrickRed}{$>$}\ e\textcolor{BrickRed}{;} \\
\mbox{}\ \ \ \ \textcolor{ForestGreen}{long}\ \textcolor{ForestGreen}{long}\ total\ \textcolor{BrickRed}{=}\ \textcolor{Purple}{0}\textcolor{BrickRed}{;} \\
\mbox{}\ \ \ \ \textbf{\textcolor{Blue}{while}}\ \textcolor{BrickRed}{(}m\textcolor{BrickRed}{-\/-)}\textcolor{Red}{\{} \\
\mbox{}\ \ \ \ \ \ edge\ t\textcolor{BrickRed}{;} \\
\mbox{}\ \ \ \ \ \ cin\ \textcolor{BrickRed}{$>$$>$}\ t\textcolor{BrickRed}{.}start\ \textcolor{BrickRed}{$>$$>$}\ t\textcolor{BrickRed}{.}end\ \textcolor{BrickRed}{$>$$>$}\ t\textcolor{BrickRed}{.}weight\textcolor{BrickRed}{;} \\
\mbox{}\ \ \ \ \ \ e\textcolor{BrickRed}{.}\textbf{\textcolor{Black}{push$\_$back}}\textcolor{BrickRed}{(}t\textcolor{BrickRed}{);} \\
\mbox{}\ \ \ \ \ \ total\ \textcolor{BrickRed}{+=}\ t\textcolor{BrickRed}{.}weight\textcolor{BrickRed}{;} \\
\mbox{}\ \ \ \ \textcolor{Red}{\}} \\
\mbox{}\ \ \ \ \textbf{\textcolor{Black}{sort}}\textcolor{BrickRed}{(}e\textcolor{BrickRed}{.}\textbf{\textcolor{Black}{begin}}\textcolor{BrickRed}{(),}\ e\textcolor{BrickRed}{.}\textbf{\textcolor{Black}{end}}\textcolor{BrickRed}{());} \\
\mbox{}\ \ \ \ \textbf{\textcolor{Blue}{for}}\ \textcolor{BrickRed}{(}\textcolor{ForestGreen}{int}\ i\textcolor{BrickRed}{=}\textcolor{Purple}{0}\textcolor{BrickRed}{;}\ i\textcolor{BrickRed}{$<$=}n\textcolor{BrickRed}{;}\ \textcolor{BrickRed}{++}i\textcolor{BrickRed}{)}\textcolor{Red}{\{} \\
\mbox{}\ \ \ \ \ \ \textbf{\textcolor{Black}{make$\_$set}}\textcolor{BrickRed}{(}i\textcolor{BrickRed}{);} \\
\mbox{}\ \ \ \ \textcolor{Red}{\}} \\
\mbox{}\ \ \ \ \textbf{\textcolor{Blue}{for}}\ \textcolor{BrickRed}{(}\textcolor{ForestGreen}{int}\ i\textcolor{BrickRed}{=}\textcolor{Purple}{0}\textcolor{BrickRed}{;}\ i\textcolor{BrickRed}{$<$}e\textcolor{BrickRed}{.}\textbf{\textcolor{Black}{size}}\textcolor{BrickRed}{();}\ \textcolor{BrickRed}{++}i\textcolor{BrickRed}{)}\textcolor{Red}{\{} \\
\mbox{}\ \ \ \ \ \ \textcolor{ForestGreen}{int}\ u\ \textcolor{BrickRed}{=}\ e\textcolor{BrickRed}{[}i\textcolor{BrickRed}{].}start\textcolor{BrickRed}{,}\ v\ \textcolor{BrickRed}{=}\ e\textcolor{BrickRed}{[}i\textcolor{BrickRed}{].}end\textcolor{BrickRed}{,}\ w\ \textcolor{BrickRed}{=}\ e\textcolor{BrickRed}{[}i\textcolor{BrickRed}{].}weight\textcolor{BrickRed}{;} \\
\mbox{}\ \ \ \ \ \ \textbf{\textcolor{Blue}{if}}\ \textcolor{BrickRed}{(}\textbf{\textcolor{Black}{find$\_$set}}\textcolor{BrickRed}{(}u\textcolor{BrickRed}{)}\ \textcolor{BrickRed}{!=}\ \textbf{\textcolor{Black}{find$\_$set}}\textcolor{BrickRed}{(}v\textcolor{BrickRed}{))}\textcolor{Red}{\{} \\
\mbox{}\ \ \ \ \ \ \ \ \textit{\textcolor{Brown}{//printf("{}Joining\ \%d\ with\ \%d\ using\ weight\ \%d\textbackslash{}n"{},\ u,\ v,\ w);}} \\
\mbox{}\ \ \ \ \ \ \ \ total\ \textcolor{BrickRed}{-=}\ w\textcolor{BrickRed}{;} \\
\mbox{}\ \ \ \ \ \ \ \ \textbf{\textcolor{Black}{merge}}\textcolor{BrickRed}{(}u\textcolor{BrickRed}{,}\ v\textcolor{BrickRed}{);} \\
\mbox{}\ \ \ \ \ \ \textcolor{Red}{\}} \\
\mbox{}\ \ \ \ \textcolor{Red}{\}} \\
\mbox{}\ \ \ \ cout\ \textcolor{BrickRed}{$<$$<$}\ total\ \textcolor{BrickRed}{$<$$<$}\ endl\textcolor{BrickRed}{;} \\
\mbox{} \\
\mbox{}\ \ \textcolor{Red}{\}} \\
\mbox{}\ \ \textbf{\textcolor{Blue}{return}}\ \textcolor{Purple}{0}\textcolor{BrickRed}{;} \\
\mbox{}\textcolor{Red}{\}} \\

} \normalfont\normalsize
%.tex

\subsection{Algoritmo de Floyd}
% Generator: GNU source-highlight, by Lorenzo Bettini, http://www.gnu.org/software/src-highlite

{\ttfamily \raggedright {
\noindent
\mbox{}\textbf{\textcolor{RoyalBlue}{\#include}}\ \texttt{\textcolor{Red}{$<$iostream$>$}} \\
\mbox{}\textbf{\textcolor{RoyalBlue}{\#include}}\ \texttt{\textcolor{Red}{$<$climits$>$}} \\
\mbox{}\textbf{\textcolor{RoyalBlue}{\#include}}\ \texttt{\textcolor{Red}{$<$algorithm$>$}} \\
\mbox{} \\
\mbox{}\textbf{\textcolor{Blue}{using}}\ \textbf{\textcolor{Blue}{namespace}}\ std\textcolor{BrickRed}{;} \\
\mbox{} \\
\mbox{}\textcolor{ForestGreen}{unsigned}\ \textcolor{ForestGreen}{long}\ \textcolor{ForestGreen}{long}\ g\textcolor{BrickRed}{[}\textcolor{Purple}{101}\textcolor{BrickRed}{][}\textcolor{Purple}{101}\textcolor{BrickRed}{];} \\
\mbox{} \\
\mbox{}\textcolor{ForestGreen}{int}\ \textbf{\textcolor{Black}{main}}\textcolor{BrickRed}{()}\textcolor{Red}{\{} \\
\mbox{}\ \ \textcolor{ForestGreen}{int}\ casos\textcolor{BrickRed}{;} \\
\mbox{}\ \ cin\ \textcolor{BrickRed}{$>$$>$}\ casos\textcolor{BrickRed}{;} \\
\mbox{}\ \ \textcolor{ForestGreen}{bool}\ first\ \textcolor{BrickRed}{=}\ \textbf{\textcolor{Blue}{true}}\textcolor{BrickRed}{;} \\
\mbox{}\ \ \textbf{\textcolor{Blue}{while}}\ \textcolor{BrickRed}{(}casos\textcolor{BrickRed}{-\/-)}\textcolor{Red}{\{} \\
\mbox{}\ \ \ \ \textbf{\textcolor{Blue}{if}}\ \textcolor{BrickRed}{(!}first\textcolor{BrickRed}{)}\ cout\ \textcolor{BrickRed}{$<$$<$}\ endl\textcolor{BrickRed}{;} \\
\mbox{}\ \ \ \ first\ \textcolor{BrickRed}{=}\ \textbf{\textcolor{Blue}{false}}\textcolor{BrickRed}{;} \\
\mbox{} \\
\mbox{}\ \ \ \ \textcolor{ForestGreen}{int}\ n\textcolor{BrickRed}{,}\ e\textcolor{BrickRed}{,}\ t\textcolor{BrickRed}{;} \\
\mbox{}\ \ \ \ cin\ \textcolor{BrickRed}{$>$$>$}\ n\ \textcolor{BrickRed}{$>$$>$}\ e\ \textcolor{BrickRed}{$>$$>$}\ t\textcolor{BrickRed}{;} \\
\mbox{}\ \ \ \ \textbf{\textcolor{Blue}{for}}\ \textcolor{BrickRed}{(}\textcolor{ForestGreen}{int}\ i\textcolor{BrickRed}{=}\textcolor{Purple}{0}\textcolor{BrickRed}{;}\ i\textcolor{BrickRed}{$<$}n\textcolor{BrickRed}{;}\ \textcolor{BrickRed}{++}i\textcolor{BrickRed}{)}\textcolor{Red}{\{} \\
\mbox{}\ \ \ \ \ \ \textbf{\textcolor{Blue}{for}}\ \textcolor{BrickRed}{(}\textcolor{ForestGreen}{int}\ j\textcolor{BrickRed}{=}\textcolor{Purple}{0}\textcolor{BrickRed}{;}\ j\textcolor{BrickRed}{$<$}n\textcolor{BrickRed}{;}\ \textcolor{BrickRed}{++}j\textcolor{BrickRed}{)}\textcolor{Red}{\{} \\
\mbox{}\ \ \ \ \ \ \ \ g\textcolor{BrickRed}{[}i\textcolor{BrickRed}{][}j\textcolor{BrickRed}{]}\ \textcolor{BrickRed}{=}\ INT$\_$MAX\textcolor{BrickRed}{;} \\
\mbox{}\ \ \ \ \ \ \textcolor{Red}{\}} \\
\mbox{}\ \ \ \ \ \ g\textcolor{BrickRed}{[}i\textcolor{BrickRed}{][}i\textcolor{BrickRed}{]}\ \textcolor{BrickRed}{=}\ \textcolor{Purple}{0}\textcolor{BrickRed}{;} \\
\mbox{}\ \ \ \ \textcolor{Red}{\}} \\
\mbox{} \\
\mbox{}\ \ \ \ \textcolor{ForestGreen}{int}\ m\textcolor{BrickRed}{;} \\
\mbox{}\ \ \ \ cin\ \textcolor{BrickRed}{$>$$>$}\ m\textcolor{BrickRed}{;} \\
\mbox{}\ \ \ \ \textbf{\textcolor{Blue}{while}}\ \textcolor{BrickRed}{(}m\textcolor{BrickRed}{-\/-)}\textcolor{Red}{\{} \\
\mbox{}\ \ \ \ \ \ \textcolor{ForestGreen}{int}\ i\textcolor{BrickRed}{,}\ j\textcolor{BrickRed}{,}\ k\textcolor{BrickRed}{;} \\
\mbox{}\ \ \ \ \ \ cin\ \textcolor{BrickRed}{$>$$>$}\ i\ \textcolor{BrickRed}{$>$$>$}\ j\ \textcolor{BrickRed}{$>$$>$}\ k\textcolor{BrickRed}{;} \\
\mbox{}\ \ \ \ \ \ g\textcolor{BrickRed}{[}i\textcolor{BrickRed}{-}\textcolor{Purple}{1}\textcolor{BrickRed}{][}j\textcolor{BrickRed}{-}\textcolor{Purple}{1}\textcolor{BrickRed}{]}\ \textcolor{BrickRed}{=}\ k\textcolor{BrickRed}{;} \\
\mbox{}\ \ \ \ \textcolor{Red}{\}} \\
\mbox{} \\
\mbox{}\ \ \ \ \textbf{\textcolor{Blue}{for}}\ \textcolor{BrickRed}{(}\textcolor{ForestGreen}{int}\ k\textcolor{BrickRed}{=}\textcolor{Purple}{0}\textcolor{BrickRed}{;}\ k\textcolor{BrickRed}{$<$}n\textcolor{BrickRed}{;}\ \textcolor{BrickRed}{++}k\textcolor{BrickRed}{)}\textcolor{Red}{\{} \\
\mbox{}\ \ \ \ \ \ \textbf{\textcolor{Blue}{for}}\ \textcolor{BrickRed}{(}\textcolor{ForestGreen}{int}\ i\textcolor{BrickRed}{=}\textcolor{Purple}{0}\textcolor{BrickRed}{;}\ i\textcolor{BrickRed}{$<$}n\textcolor{BrickRed}{;}\ \textcolor{BrickRed}{++}i\textcolor{BrickRed}{)}\textcolor{Red}{\{} \\
\mbox{}\ \ \ \ \ \ \ \ \textbf{\textcolor{Blue}{for}}\ \textcolor{BrickRed}{(}\textcolor{ForestGreen}{int}\ j\textcolor{BrickRed}{=}\textcolor{Purple}{0}\textcolor{BrickRed}{;}\ j\textcolor{BrickRed}{$<$}n\textcolor{BrickRed}{;}\ \textcolor{BrickRed}{++}j\textcolor{BrickRed}{)}\textcolor{Red}{\{} \\
\mbox{}\ \ \ \ \ \ \ \ \ \ g\textcolor{BrickRed}{[}i\textcolor{BrickRed}{][}j\textcolor{BrickRed}{]}\ \textcolor{BrickRed}{=}\ \textbf{\textcolor{Black}{min}}\textcolor{BrickRed}{(}g\textcolor{BrickRed}{[}i\textcolor{BrickRed}{][}j\textcolor{BrickRed}{],}\ g\textcolor{BrickRed}{[}i\textcolor{BrickRed}{][}k\textcolor{BrickRed}{]}\ \textcolor{BrickRed}{+}\ g\textcolor{BrickRed}{[}k\textcolor{BrickRed}{][}j\textcolor{BrickRed}{]);} \\
\mbox{}\ \ \ \ \ \ \ \ \textcolor{Red}{\}} \\
\mbox{}\ \ \ \ \ \ \textcolor{Red}{\}} \\
\mbox{}\ \ \ \ \textcolor{Red}{\}} \\
\mbox{}\ \ \ \  \\
\mbox{}\ \ \ \ \textcolor{ForestGreen}{int}\ r\textcolor{BrickRed}{=}\textcolor{Purple}{0}\textcolor{BrickRed}{;} \\
\mbox{}\ \ \ \ e\ \textcolor{BrickRed}{-=}\ \textcolor{Purple}{1}\textcolor{BrickRed}{;} \\
\mbox{}\ \ \ \ \textbf{\textcolor{Blue}{for}}\ \textcolor{BrickRed}{(}\textcolor{ForestGreen}{int}\ i\textcolor{BrickRed}{=}\textcolor{Purple}{0}\textcolor{BrickRed}{;}\ i\textcolor{BrickRed}{$<$}n\textcolor{BrickRed}{;}\ \textcolor{BrickRed}{++}i\textcolor{BrickRed}{)}\textcolor{Red}{\{} \\
\mbox{}\ \ \ \ \ \ r\ \textcolor{BrickRed}{+=}\ \textcolor{BrickRed}{((}g\textcolor{BrickRed}{[}i\textcolor{BrickRed}{][}e\textcolor{BrickRed}{]}\ \textcolor{BrickRed}{$<$=}\ t\textcolor{BrickRed}{)}\ \textcolor{BrickRed}{?}\ \textcolor{Purple}{1}\ \textcolor{BrickRed}{:}\ \textcolor{Purple}{0}\textcolor{BrickRed}{);}\ \ \ \ \ \  \\
\mbox{}\ \ \ \ \textcolor{Red}{\}} \\
\mbox{} \\
\mbox{}\ \ \ \ cout\ \textcolor{BrickRed}{$<$$<$}\ r\ \textcolor{BrickRed}{$<$$<$}\ endl\textcolor{BrickRed}{;} \\
\mbox{}\ \ \textcolor{Red}{\}} \\
\mbox{}\ \ \textbf{\textcolor{Blue}{return}}\ \textcolor{Purple}{0}\textcolor{BrickRed}{;} \\
\mbox{}\textcolor{Red}{\}} \\

} \normalfont\normalsize
%.tex

\subsection{Puntos de articulación}
% Generator: GNU source-highlight, by Lorenzo Bettini, http://www.gnu.org/software/src-highlite

{\ttfamily \raggedright {
\noindent
\mbox{}\textbf{\textcolor{RoyalBlue}{\#include}}\ \texttt{\textcolor{Red}{$<$vector$>$}} \\
\mbox{}\textbf{\textcolor{RoyalBlue}{\#include}}\ \texttt{\textcolor{Red}{$<$set$>$}} \\
\mbox{}\textbf{\textcolor{RoyalBlue}{\#include}}\ \texttt{\textcolor{Red}{$<$map$>$}} \\
\mbox{}\textbf{\textcolor{RoyalBlue}{\#include}}\ \texttt{\textcolor{Red}{$<$algorithm$>$}} \\
\mbox{}\textbf{\textcolor{RoyalBlue}{\#include}}\ \texttt{\textcolor{Red}{$<$iostream$>$}} \\
\mbox{}\textbf{\textcolor{RoyalBlue}{\#include}}\ \texttt{\textcolor{Red}{$<$iterator$>$}} \\
\mbox{} \\
\mbox{}\textbf{\textcolor{Blue}{using}}\ \textbf{\textcolor{Blue}{namespace}}\ std\textcolor{BrickRed}{;} \\
\mbox{} \\
\mbox{}\textbf{\textcolor{Blue}{typedef}}\ string\ node\textcolor{BrickRed}{;} \\
\mbox{}\textbf{\textcolor{Blue}{typedef}}\ map\textcolor{BrickRed}{$<$}node\textcolor{BrickRed}{,}\ vector\textcolor{BrickRed}{$<$}node\textcolor{BrickRed}{$>$}\ \textcolor{BrickRed}{$>$}\ graph\textcolor{BrickRed}{;} \\
\mbox{}\textbf{\textcolor{Blue}{typedef}}\ \textcolor{ForestGreen}{char}\ color\textcolor{BrickRed}{;} \\
\mbox{} \\
\mbox{}\textbf{\textcolor{Blue}{const}}\ color\ WHITE\ \textcolor{BrickRed}{=}\ \textcolor{Purple}{0}\textcolor{BrickRed}{,}\ GRAY\ \textcolor{BrickRed}{=}\ \textcolor{Purple}{1}\textcolor{BrickRed}{,}\ BLACK\ \textcolor{BrickRed}{=}\ \textcolor{Purple}{2}\textcolor{BrickRed}{;} \\
\mbox{} \\
\mbox{}graph\ g\textcolor{BrickRed}{;} \\
\mbox{}map\textcolor{BrickRed}{$<$}node\textcolor{BrickRed}{,}\ color\textcolor{BrickRed}{$>$}\ colors\textcolor{BrickRed}{;} \\
\mbox{}map\textcolor{BrickRed}{$<$}node\textcolor{BrickRed}{,}\ \textcolor{ForestGreen}{int}\textcolor{BrickRed}{$>$}\ d\textcolor{BrickRed}{,}\ low\textcolor{BrickRed}{;} \\
\mbox{} \\
\mbox{}set\textcolor{BrickRed}{$<$}node\textcolor{BrickRed}{$>$}\ cameras\textcolor{BrickRed}{;} \\
\mbox{} \\
\mbox{}\textcolor{ForestGreen}{int}\ timeCount\textcolor{BrickRed}{;} \\
\mbox{} \\
\mbox{}\textcolor{ForestGreen}{void}\ \textbf{\textcolor{Black}{dfs}}\textcolor{BrickRed}{(}node\ v\textcolor{BrickRed}{,}\ \textcolor{ForestGreen}{bool}\ isRoot\ \textcolor{BrickRed}{=}\ \textbf{\textcolor{Blue}{true}}\textcolor{BrickRed}{)}\textcolor{Red}{\{} \\
\mbox{}\ \ colors\textcolor{BrickRed}{[}v\textcolor{BrickRed}{]}\ \textcolor{BrickRed}{=}\ GRAY\textcolor{BrickRed}{;} \\
\mbox{}\ \ d\textcolor{BrickRed}{[}v\textcolor{BrickRed}{]}\ \textcolor{BrickRed}{=}\ low\textcolor{BrickRed}{[}v\textcolor{BrickRed}{]}\ \textcolor{BrickRed}{=}\ \textcolor{BrickRed}{++}timeCount\textcolor{BrickRed}{;} \\
\mbox{}\ \ vector\textcolor{BrickRed}{$<$}node\textcolor{BrickRed}{$>$}\ neighbors\ \textcolor{BrickRed}{=}\ g\textcolor{BrickRed}{[}v\textcolor{BrickRed}{];} \\
\mbox{}\ \ \textcolor{ForestGreen}{int}\ count\ \textcolor{BrickRed}{=}\ \textcolor{Purple}{0}\textcolor{BrickRed}{;} \\
\mbox{}\ \ \textbf{\textcolor{Blue}{for}}\ \textcolor{BrickRed}{(}\textcolor{ForestGreen}{int}\ i\textcolor{BrickRed}{=}\textcolor{Purple}{0}\textcolor{BrickRed}{;}\ i\textcolor{BrickRed}{$<$}neighbors\textcolor{BrickRed}{.}\textbf{\textcolor{Black}{size}}\textcolor{BrickRed}{();}\ \textcolor{BrickRed}{++}i\textcolor{BrickRed}{)}\textcolor{Red}{\{} \\
\mbox{}\ \ \ \ \textbf{\textcolor{Blue}{if}}\ \textcolor{BrickRed}{(}colors\textcolor{BrickRed}{[}neighbors\textcolor{BrickRed}{[}i\textcolor{BrickRed}{]]}\ \textcolor{BrickRed}{==}\ WHITE\textcolor{BrickRed}{)}\textcolor{Red}{\{}\ \textit{\textcolor{Brown}{//\ \ (v,\ neighbors[i])\ is\ a\ tree\ edge}} \\
\mbox{}\ \ \ \ \ \ \textbf{\textcolor{Black}{dfs}}\textcolor{BrickRed}{(}neighbors\textcolor{BrickRed}{[}i\textcolor{BrickRed}{],}\ \textbf{\textcolor{Blue}{false}}\textcolor{BrickRed}{);} \\
\mbox{}\ \ \ \ \ \ \textbf{\textcolor{Blue}{if}}\ \textcolor{BrickRed}{(!}isRoot\ \textcolor{BrickRed}{\&\&}\ low\textcolor{BrickRed}{[}neighbors\textcolor{BrickRed}{[}i\textcolor{BrickRed}{]]}\ \textcolor{BrickRed}{$>$=}\ d\textcolor{BrickRed}{[}v\textcolor{BrickRed}{])}\textcolor{Red}{\{} \\
\mbox{}\ \ \ \ \ \ \ \ cameras\textcolor{BrickRed}{.}\textbf{\textcolor{Black}{insert}}\textcolor{BrickRed}{(}v\textcolor{BrickRed}{);} \\
\mbox{}\ \ \ \ \ \ \textcolor{Red}{\}} \\
\mbox{}\ \ \ \ \ \ low\textcolor{BrickRed}{[}v\textcolor{BrickRed}{]}\ \textcolor{BrickRed}{=}\ \textbf{\textcolor{Black}{min}}\textcolor{BrickRed}{(}low\textcolor{BrickRed}{[}v\textcolor{BrickRed}{],}\ low\textcolor{BrickRed}{[}neighbors\textcolor{BrickRed}{[}i\textcolor{BrickRed}{]]);} \\
\mbox{}\ \ \ \ \ \ \textcolor{BrickRed}{++}count\textcolor{BrickRed}{;} \\
\mbox{}\ \ \ \ \textcolor{Red}{\}}\textbf{\textcolor{Blue}{else}}\textcolor{Red}{\{}\ \textit{\textcolor{Brown}{//\ (v,\ neighbors[i])\ is\ a\ back\ edge}} \\
\mbox{}\ \ \ \ \ \ low\textcolor{BrickRed}{[}v\textcolor{BrickRed}{]}\ \textcolor{BrickRed}{=}\ \textbf{\textcolor{Black}{min}}\textcolor{BrickRed}{(}low\textcolor{BrickRed}{[}v\textcolor{BrickRed}{],}\ d\textcolor{BrickRed}{[}neighbors\textcolor{BrickRed}{[}i\textcolor{BrickRed}{]]);} \\
\mbox{}\ \ \ \ \textcolor{Red}{\}} \\
\mbox{}\ \ \textcolor{Red}{\}} \\
\mbox{}\ \ \textbf{\textcolor{Blue}{if}}\ \textcolor{BrickRed}{(}isRoot\ \textcolor{BrickRed}{\&\&}\ count\ \textcolor{BrickRed}{$>$}\ \textcolor{Purple}{1}\textcolor{BrickRed}{)}\textcolor{Red}{\{}\ \textit{\textcolor{Brown}{//Is\ root\ and\ has\ two\ neighbors\ in\ the\ DFS-tree}} \\
\mbox{}\ \ \ \ cameras\textcolor{BrickRed}{.}\textbf{\textcolor{Black}{insert}}\textcolor{BrickRed}{(}v\textcolor{BrickRed}{);} \\
\mbox{}\ \ \textcolor{Red}{\}} \\
\mbox{}\ \ colors\textcolor{BrickRed}{[}v\textcolor{BrickRed}{]}\ \textcolor{BrickRed}{=}\ BLACK\textcolor{BrickRed}{;} \\
\mbox{}\textcolor{Red}{\}} \\
\mbox{} \\
\mbox{}\textcolor{ForestGreen}{int}\ \textbf{\textcolor{Black}{main}}\textcolor{BrickRed}{()}\textcolor{Red}{\{} \\
\mbox{}\ \ \textcolor{ForestGreen}{int}\ n\textcolor{BrickRed}{;} \\
\mbox{}\ \ \textcolor{ForestGreen}{int}\ map\ \textcolor{BrickRed}{=}\ \textcolor{Purple}{1}\textcolor{BrickRed}{;} \\
\mbox{}\ \ \textbf{\textcolor{Blue}{while}}\ \textcolor{BrickRed}{(}cin\ \textcolor{BrickRed}{$>$$>$}\ n\ \textcolor{BrickRed}{\&\&}\ n\ \textcolor{BrickRed}{$>$}\ \textcolor{Purple}{0}\textcolor{BrickRed}{)}\textcolor{Red}{\{} \\
\mbox{}\ \ \ \ \textbf{\textcolor{Blue}{if}}\ \textcolor{BrickRed}{(}map\ \textcolor{BrickRed}{$>$}\ \textcolor{Purple}{1}\textcolor{BrickRed}{)}\ cout\ \textcolor{BrickRed}{$<$$<$}\ endl\textcolor{BrickRed}{;} \\
\mbox{}\ \ \ \ g\textcolor{BrickRed}{.}\textbf{\textcolor{Black}{clear}}\textcolor{BrickRed}{();} \\
\mbox{}\ \ \ \ colors\textcolor{BrickRed}{.}\textbf{\textcolor{Black}{clear}}\textcolor{BrickRed}{();} \\
\mbox{}\ \ \ \ d\textcolor{BrickRed}{.}\textbf{\textcolor{Black}{clear}}\textcolor{BrickRed}{();} \\
\mbox{}\ \ \ \ low\textcolor{BrickRed}{.}\textbf{\textcolor{Black}{clear}}\textcolor{BrickRed}{();} \\
\mbox{}\ \ \ \ timeCount\ \textcolor{BrickRed}{=}\ \textcolor{Purple}{0}\textcolor{BrickRed}{;} \\
\mbox{}\ \ \ \ \textbf{\textcolor{Blue}{while}}\ \textcolor{BrickRed}{(}n\textcolor{BrickRed}{-\/-)}\textcolor{Red}{\{} \\
\mbox{}\ \ \ \ \ \ node\ v\textcolor{BrickRed}{;} \\
\mbox{}\ \ \ \ \ \ cin\ \textcolor{BrickRed}{$>$$>$}\ v\textcolor{BrickRed}{;} \\
\mbox{}\ \ \ \ \ \ colors\textcolor{BrickRed}{[}v\textcolor{BrickRed}{]}\ \textcolor{BrickRed}{=}\ WHITE\textcolor{BrickRed}{;} \\
\mbox{}\ \ \ \ \ \ g\textcolor{BrickRed}{[}v\textcolor{BrickRed}{]}\ \textcolor{BrickRed}{=}\ vector\textcolor{BrickRed}{$<$}node\textcolor{BrickRed}{$>$();} \\
\mbox{}\ \ \ \ \textcolor{Red}{\}} \\
\mbox{}\ \ \ \  \\
\mbox{}\ \ \ \ cin\ \textcolor{BrickRed}{$>$$>$}\ n\textcolor{BrickRed}{;} \\
\mbox{}\ \ \ \ \textbf{\textcolor{Blue}{while}}\ \textcolor{BrickRed}{(}n\textcolor{BrickRed}{-\/-)}\textcolor{Red}{\{} \\
\mbox{}\ \ \ \ \ \ node\ v\textcolor{BrickRed}{,}u\textcolor{BrickRed}{;} \\
\mbox{}\ \ \ \ \ \ cin\ \textcolor{BrickRed}{$>$$>$}\ v\ \textcolor{BrickRed}{$>$$>$}\ u\textcolor{BrickRed}{;} \\
\mbox{}\ \ \ \ \ \ g\textcolor{BrickRed}{[}v\textcolor{BrickRed}{].}\textbf{\textcolor{Black}{push$\_$back}}\textcolor{BrickRed}{(}u\textcolor{BrickRed}{);} \\
\mbox{}\ \ \ \ \ \ g\textcolor{BrickRed}{[}u\textcolor{BrickRed}{].}\textbf{\textcolor{Black}{push$\_$back}}\textcolor{BrickRed}{(}v\textcolor{BrickRed}{);} \\
\mbox{}\ \ \ \ \textcolor{Red}{\}} \\
\mbox{}\ \ \ \  \\
\mbox{}\ \ \ \ cameras\textcolor{BrickRed}{.}\textbf{\textcolor{Black}{clear}}\textcolor{BrickRed}{();} \\
\mbox{}\ \ \ \  \\
\mbox{}\ \ \ \ \textbf{\textcolor{Blue}{for}}\ \textcolor{BrickRed}{(}graph\textcolor{BrickRed}{::}iterator\ i\ \textcolor{BrickRed}{=}\ g\textcolor{BrickRed}{.}\textbf{\textcolor{Black}{begin}}\textcolor{BrickRed}{();}\ i\ \textcolor{BrickRed}{!=}\ g\textcolor{BrickRed}{.}\textbf{\textcolor{Black}{end}}\textcolor{BrickRed}{();}\ \textcolor{BrickRed}{++}i\textcolor{BrickRed}{)}\textcolor{Red}{\{} \\
\mbox{}\ \ \ \ \ \ \textbf{\textcolor{Blue}{if}}\ \textcolor{BrickRed}{(}colors\textcolor{BrickRed}{[(*}i\textcolor{BrickRed}{).}first\textcolor{BrickRed}{]}\ \textcolor{BrickRed}{==}\ WHITE\textcolor{BrickRed}{)}\textcolor{Red}{\{} \\
\mbox{}\ \ \ \ \ \ \ \ \textbf{\textcolor{Black}{dfs}}\textcolor{BrickRed}{((*}i\textcolor{BrickRed}{).}first\textcolor{BrickRed}{);} \\
\mbox{}\ \ \ \ \ \ \textcolor{Red}{\}} \\
\mbox{}\ \ \ \ \textcolor{Red}{\}} \\
\mbox{}\ \ \ \ \ \  \\
\mbox{}\ \ \ \ cout\ \textcolor{BrickRed}{$<$$<$}\ \texttt{\textcolor{Red}{"{}City\ map\ \#"{}}}\textcolor{BrickRed}{$<$$<$}map\textcolor{BrickRed}{$<$$<$}\texttt{\textcolor{Red}{"{}:\ "{}}}\ \textcolor{BrickRed}{$<$$<$}\ cameras\textcolor{BrickRed}{.}\textbf{\textcolor{Black}{size}}\textcolor{BrickRed}{()}\ \textcolor{BrickRed}{$<$$<$}\ \texttt{\textcolor{Red}{"{}\ camera(s)\ found"{}}}\ \textcolor{BrickRed}{$<$$<$}\ endl\textcolor{BrickRed}{;} \\
\mbox{}\ \ \ \ \textbf{\textcolor{Black}{copy}}\textcolor{BrickRed}{(}cameras\textcolor{BrickRed}{.}\textbf{\textcolor{Black}{begin}}\textcolor{BrickRed}{(),}\ cameras\textcolor{BrickRed}{.}\textbf{\textcolor{Black}{end}}\textcolor{BrickRed}{(),}\ ostream$\_$iterator\textcolor{BrickRed}{$<$}node\textcolor{BrickRed}{$>$(}cout\textcolor{BrickRed}{,}\texttt{\textcolor{Red}{"{}}}\texttt{\textcolor{CarnationPink}{\textbackslash{}n}}\texttt{\textcolor{Red}{"{}}}\textcolor{BrickRed}{));} \\
\mbox{}\ \ \ \ \textcolor{BrickRed}{++}map\textcolor{BrickRed}{;} \\
\mbox{}\ \ \textcolor{Red}{\}} \\
\mbox{}\ \ \textbf{\textcolor{Blue}{return}}\ \textcolor{Purple}{0}\textcolor{BrickRed}{;} \\
\mbox{}\textcolor{Red}{\}} \\

} \normalfont\normalsize
%.tex

\subsection{Máximo flujo: Método de Ford-Fulkerson, algoritmo de Edmonds-Karp}
El algoritmo de Edmonds-Karp es una modificación al método de Ford-Fulkerson. Este último
utiliza DFS para hallar un camino de aumentación, pero la sugerencia de Edmonds-Karp
es utilizar BFS que lo hace más eficiente en algunos grafos.
% Generator: GNU source-highlight, by Lorenzo Bettini, http://www.gnu.org/software/src-highlite

{\ttfamily \raggedright {
\noindent
\mbox{}\textcolor{ForestGreen}{int}\ cap\textcolor{BrickRed}{[}MAXN\textcolor{BrickRed}{+}\textcolor{Purple}{1}\textcolor{BrickRed}{][}MAXN\textcolor{BrickRed}{+}\textcolor{Purple}{1}\textcolor{BrickRed}{],}\ flow\textcolor{BrickRed}{[}MAXN\textcolor{BrickRed}{+}\textcolor{Purple}{1}\textcolor{BrickRed}{][}MAXN\textcolor{BrickRed}{+}\textcolor{Purple}{1}\textcolor{BrickRed}{],}\ prev\textcolor{BrickRed}{[}MAXN\textcolor{BrickRed}{+}\textcolor{Purple}{1}\textcolor{BrickRed}{];} \\
\mbox{} \\
\mbox{}\textit{\textcolor{Brown}{/*}} \\
\mbox{}\textit{\textcolor{Brown}{\ \ cap[i][j]\ =\ Capacidad\ de\ la\ arista\ (i,\ j).}} \\
\mbox{}\textit{\textcolor{Brown}{\ \ flow[i][j]\ =\ Flujo\ actual\ de\ i\ a\ j.}} \\
\mbox{}\textit{\textcolor{Brown}{\ \ prev[i]\ =\ Predecesor\ del\ nodo\ i\ en\ un\ camino\ de\ aumentación.}} \\
\mbox{}\textit{\textcolor{Brown}{\ */}} \\
\mbox{} \\
\mbox{}\textcolor{ForestGreen}{int}\ \textbf{\textcolor{Black}{fordFulkerson}}\textcolor{BrickRed}{(}\textcolor{ForestGreen}{int}\ n\textcolor{BrickRed}{,}\ \textcolor{ForestGreen}{int}\ s\textcolor{BrickRed}{,}\ \textcolor{ForestGreen}{int}\ t\textcolor{BrickRed}{)}\textcolor{Red}{\{} \\
\mbox{}\ \ \textcolor{ForestGreen}{int}\ ans\ \textcolor{BrickRed}{=}\ \textcolor{Purple}{0}\textcolor{BrickRed}{;} \\
\mbox{}\ \ \textbf{\textcolor{Blue}{for}}\ \textcolor{BrickRed}{(}\textcolor{ForestGreen}{int}\ i\textcolor{BrickRed}{=}\textcolor{Purple}{0}\textcolor{BrickRed}{;}\ i\textcolor{BrickRed}{$<$}n\textcolor{BrickRed}{;}\ \textcolor{BrickRed}{++}i\textcolor{BrickRed}{)}\ \textbf{\textcolor{Black}{fill}}\textcolor{BrickRed}{(}flow\textcolor{BrickRed}{[}i\textcolor{BrickRed}{],}\ flow\textcolor{BrickRed}{[}i\textcolor{BrickRed}{]+}n\textcolor{BrickRed}{,}\ \textcolor{Purple}{0}\textcolor{BrickRed}{);} \\
\mbox{}\ \ \textbf{\textcolor{Blue}{while}}\ \textcolor{BrickRed}{(}\textbf{\textcolor{Blue}{true}}\textcolor{BrickRed}{)}\textcolor{Red}{\{} \\
\mbox{}\ \ \ \ \textbf{\textcolor{Black}{fill}}\textcolor{BrickRed}{(}prev\textcolor{BrickRed}{,}\ prev\textcolor{BrickRed}{+}n\textcolor{BrickRed}{,}\ \textcolor{BrickRed}{-}\textcolor{Purple}{1}\textcolor{BrickRed}{);} \\
\mbox{}\ \ \ \ queue\textcolor{BrickRed}{$<$}\textcolor{ForestGreen}{int}\textcolor{BrickRed}{$>$}\ q\textcolor{BrickRed}{;} \\
\mbox{}\ \ \ \ q\textcolor{BrickRed}{.}\textbf{\textcolor{Black}{push}}\textcolor{BrickRed}{(}s\textcolor{BrickRed}{);} \\
\mbox{}\ \ \ \ \textbf{\textcolor{Blue}{while}}\ \textcolor{BrickRed}{(}q\textcolor{BrickRed}{.}\textbf{\textcolor{Black}{size}}\textcolor{BrickRed}{()}\ \textcolor{BrickRed}{\&\&}\ prev\textcolor{BrickRed}{[}t\textcolor{BrickRed}{]}\ \textcolor{BrickRed}{==}\ \textcolor{BrickRed}{-}\textcolor{Purple}{1}\textcolor{BrickRed}{)}\textcolor{Red}{\{} \\
\mbox{}\ \ \ \ \ \ \textcolor{ForestGreen}{int}\ u\ \textcolor{BrickRed}{=}\ q\textcolor{BrickRed}{.}\textbf{\textcolor{Black}{front}}\textcolor{BrickRed}{();} \\
\mbox{}\ \ \ \ \ \ q\textcolor{BrickRed}{.}\textbf{\textcolor{Black}{pop}}\textcolor{BrickRed}{();} \\
\mbox{}\ \ \ \ \ \ \textbf{\textcolor{Blue}{for}}\ \textcolor{BrickRed}{(}\textcolor{ForestGreen}{int}\ v\ \textcolor{BrickRed}{=}\ \textcolor{Purple}{0}\textcolor{BrickRed}{;}\ v\textcolor{BrickRed}{$<$}n\textcolor{BrickRed}{;}\ \textcolor{BrickRed}{++}v\textcolor{BrickRed}{)} \\
\mbox{}\ \ \ \ \ \ \ \ \textbf{\textcolor{Blue}{if}}\ \textcolor{BrickRed}{(}\ v\ \textcolor{BrickRed}{!=}\ s\ \textcolor{BrickRed}{\&\&}\ prev\textcolor{BrickRed}{[}v\textcolor{BrickRed}{]}\ \textcolor{BrickRed}{==}\ \textcolor{BrickRed}{-}\textcolor{Purple}{1}\ \textcolor{BrickRed}{\&\&}\ cap\textcolor{BrickRed}{[}u\textcolor{BrickRed}{][}v\textcolor{BrickRed}{]}\ \textcolor{BrickRed}{$>$}\ flow\textcolor{BrickRed}{[}u\textcolor{BrickRed}{][}v\textcolor{BrickRed}{]}\ \textcolor{BrickRed}{)} \\
\mbox{}\ \ \ \ \ \ \ \ \ \ prev\textcolor{BrickRed}{[}v\textcolor{BrickRed}{]}\ \textcolor{BrickRed}{=}\ u\textcolor{BrickRed}{,}\ q\textcolor{BrickRed}{.}\textbf{\textcolor{Black}{push}}\textcolor{BrickRed}{(}v\textcolor{BrickRed}{);} \\
\mbox{}\ \ \ \ \textcolor{Red}{\}} \\
\mbox{} \\
\mbox{}\ \ \ \ \textbf{\textcolor{Blue}{if}}\ \textcolor{BrickRed}{(}prev\textcolor{BrickRed}{[}t\textcolor{BrickRed}{]}\ \textcolor{BrickRed}{==}\ \textcolor{BrickRed}{-}\textcolor{Purple}{1}\textcolor{BrickRed}{)}\ \textbf{\textcolor{Blue}{break}}\textcolor{BrickRed}{;} \\
\mbox{} \\
\mbox{}\ \ \ \ \textcolor{ForestGreen}{int}\ bottleneck\ \textcolor{BrickRed}{=}\ INT$\_$MAX\textcolor{BrickRed}{;} \\
\mbox{}\ \ \ \ \textbf{\textcolor{Blue}{for}}\ \textcolor{BrickRed}{(}\textcolor{ForestGreen}{int}\ v\ \textcolor{BrickRed}{=}\ t\textcolor{BrickRed}{,}\ u\ \textcolor{BrickRed}{=}\ prev\textcolor{BrickRed}{[}v\textcolor{BrickRed}{];}\ u\ \textcolor{BrickRed}{!=}\ \textcolor{BrickRed}{-}\textcolor{Purple}{1}\textcolor{BrickRed}{;}\ v\ \textcolor{BrickRed}{=}\ u\textcolor{BrickRed}{,}\ u\ \textcolor{BrickRed}{=}\ prev\textcolor{BrickRed}{[}v\textcolor{BrickRed}{])}\textcolor{Red}{\{} \\
\mbox{}\ \ \ \ \ \ bottleneck\ \textcolor{BrickRed}{=}\ \textbf{\textcolor{Black}{min}}\textcolor{BrickRed}{(}bottleneck\textcolor{BrickRed}{,}\ cap\textcolor{BrickRed}{[}u\textcolor{BrickRed}{][}v\textcolor{BrickRed}{]}\ \textcolor{BrickRed}{-}\ flow\textcolor{BrickRed}{[}u\textcolor{BrickRed}{][}v\textcolor{BrickRed}{]);} \\
\mbox{}\ \ \ \ \textcolor{Red}{\}} \\
\mbox{}\ \ \ \ \textbf{\textcolor{Blue}{for}}\ \textcolor{BrickRed}{(}\textcolor{ForestGreen}{int}\ v\ \textcolor{BrickRed}{=}\ t\textcolor{BrickRed}{,}\ u\ \textcolor{BrickRed}{=}\ prev\textcolor{BrickRed}{[}v\textcolor{BrickRed}{];}\ u\ \textcolor{BrickRed}{!=}\ \textcolor{BrickRed}{-}\textcolor{Purple}{1}\textcolor{BrickRed}{;}\ v\ \textcolor{BrickRed}{=}\ u\textcolor{BrickRed}{,}\ u\ \textcolor{BrickRed}{=}\ prev\textcolor{BrickRed}{[}v\textcolor{BrickRed}{])}\textcolor{Red}{\{} \\
\mbox{}\ \ \ \ \ \ flow\textcolor{BrickRed}{[}u\textcolor{BrickRed}{][}v\textcolor{BrickRed}{]}\ \textcolor{BrickRed}{+=}\ bottleneck\textcolor{BrickRed}{;} \\
\mbox{}\ \ \ \ \ \ flow\textcolor{BrickRed}{[}v\textcolor{BrickRed}{][}u\textcolor{BrickRed}{]}\ \textcolor{BrickRed}{=}\ \textcolor{BrickRed}{-}flow\textcolor{BrickRed}{[}u\textcolor{BrickRed}{][}v\textcolor{BrickRed}{];} \\
\mbox{}\ \ \ \ \textcolor{Red}{\}} \\
\mbox{}\ \ \ \ ans\ \textcolor{BrickRed}{+=}\ bottleneck\textcolor{BrickRed}{;} \\
\mbox{}\ \ \textcolor{Red}{\}} \\
\mbox{}\ \ \textbf{\textcolor{Blue}{return}}\ ans\textcolor{BrickRed}{;} \\
\mbox{}\textcolor{Red}{\}} \\

} \normalfont\normalsize
%.tex

\section{Programación dinámica}
\subsection{Longest common subsequence}
% Generator: GNU source-highlight, by Lorenzo Bettini, http://www.gnu.org/software/src-highlite

{\ttfamily \raggedright {
\noindent
\mbox{}\textbf{\textcolor{RoyalBlue}{\#define}}\ \textbf{\textcolor{Black}{MAX}}\textcolor{BrickRed}{(}a\textcolor{BrickRed}{,}b\textcolor{BrickRed}{)}\ \textcolor{BrickRed}{((}a\textcolor{BrickRed}{$>$}b\textcolor{BrickRed}{)?(}a\textcolor{BrickRed}{):(}b\textcolor{BrickRed}{))} \\
\mbox{}\textcolor{ForestGreen}{int}\ dp\textcolor{BrickRed}{[}\textcolor{Purple}{1001}\textcolor{BrickRed}{][}\textcolor{Purple}{1001}\textcolor{BrickRed}{];} \\
\mbox{} \\
\mbox{}\textcolor{ForestGreen}{int}\ \textbf{\textcolor{Black}{lcs}}\textcolor{BrickRed}{(}\textbf{\textcolor{Blue}{const}}\ string\ \textcolor{BrickRed}{\&}s\textcolor{BrickRed}{,}\ \textbf{\textcolor{Blue}{const}}\ string\ \textcolor{BrickRed}{\&}t\textcolor{BrickRed}{)}\textcolor{Red}{\{} \\
\mbox{}\ \ \textcolor{ForestGreen}{int}\ m\ \textcolor{BrickRed}{=}\ s\textcolor{BrickRed}{.}\textbf{\textcolor{Black}{size}}\textcolor{BrickRed}{(),}\ n\ \textcolor{BrickRed}{=}\ t\textcolor{BrickRed}{.}\textbf{\textcolor{Black}{size}}\textcolor{BrickRed}{();} \\
\mbox{}\ \ \textbf{\textcolor{Blue}{if}}\ \textcolor{BrickRed}{(}m\ \textcolor{BrickRed}{==}\ \textcolor{Purple}{0}\ \textcolor{BrickRed}{$|$$|$}\ n\ \textcolor{BrickRed}{==}\ \textcolor{Purple}{0}\textcolor{BrickRed}{)}\ \textbf{\textcolor{Blue}{return}}\ \textcolor{Purple}{0}\textcolor{BrickRed}{;} \\
\mbox{}\ \ \textbf{\textcolor{Blue}{for}}\ \textcolor{BrickRed}{(}\textcolor{ForestGreen}{int}\ i\textcolor{BrickRed}{=}\textcolor{Purple}{0}\textcolor{BrickRed}{;}\ i\textcolor{BrickRed}{$<$=}m\textcolor{BrickRed}{;}\ \textcolor{BrickRed}{++}i\textcolor{BrickRed}{)} \\
\mbox{}\ \ \ \ dp\textcolor{BrickRed}{[}i\textcolor{BrickRed}{][}\textcolor{Purple}{0}\textcolor{BrickRed}{]}\ \textcolor{BrickRed}{=}\ \textcolor{Purple}{0}\textcolor{BrickRed}{;} \\
\mbox{}\ \ \textbf{\textcolor{Blue}{for}}\ \textcolor{BrickRed}{(}\textcolor{ForestGreen}{int}\ j\textcolor{BrickRed}{=}\textcolor{Purple}{1}\textcolor{BrickRed}{;}\ j\textcolor{BrickRed}{$<$=}n\textcolor{BrickRed}{;}\ \textcolor{BrickRed}{++}j\textcolor{BrickRed}{)} \\
\mbox{}\ \ \ \ dp\textcolor{BrickRed}{[}\textcolor{Purple}{0}\textcolor{BrickRed}{][}j\textcolor{BrickRed}{]}\ \textcolor{BrickRed}{=}\ \textcolor{Purple}{0}\textcolor{BrickRed}{;} \\
\mbox{}\ \ \textbf{\textcolor{Blue}{for}}\ \textcolor{BrickRed}{(}\textcolor{ForestGreen}{int}\ i\textcolor{BrickRed}{=}\textcolor{Purple}{0}\textcolor{BrickRed}{;}\ i\textcolor{BrickRed}{$<$}m\textcolor{BrickRed}{;}\ \textcolor{BrickRed}{++}i\textcolor{BrickRed}{)} \\
\mbox{}\ \ \ \ \textbf{\textcolor{Blue}{for}}\ \textcolor{BrickRed}{(}\textcolor{ForestGreen}{int}\ j\textcolor{BrickRed}{=}\textcolor{Purple}{0}\textcolor{BrickRed}{;}\ j\textcolor{BrickRed}{$<$}n\textcolor{BrickRed}{;}\ \textcolor{BrickRed}{++}j\textcolor{BrickRed}{)} \\
\mbox{}\ \ \ \ \ \ \textbf{\textcolor{Blue}{if}}\ \textcolor{BrickRed}{(}s\textcolor{BrickRed}{[}i\textcolor{BrickRed}{]}\ \textcolor{BrickRed}{==}\ t\textcolor{BrickRed}{[}j\textcolor{BrickRed}{])} \\
\mbox{}\ \ \ \ \ \ \ \ dp\textcolor{BrickRed}{[}i\textcolor{BrickRed}{+}\textcolor{Purple}{1}\textcolor{BrickRed}{][}j\textcolor{BrickRed}{+}\textcolor{Purple}{1}\textcolor{BrickRed}{]}\ \textcolor{BrickRed}{=}\ dp\textcolor{BrickRed}{[}i\textcolor{BrickRed}{][}j\textcolor{BrickRed}{]+}\textcolor{Purple}{1}\textcolor{BrickRed}{;} \\
\mbox{}\ \ \ \ \ \ \textbf{\textcolor{Blue}{else}} \\
\mbox{}\ \ \ \ \ \ \ \ dp\textcolor{BrickRed}{[}i\textcolor{BrickRed}{+}\textcolor{Purple}{1}\textcolor{BrickRed}{][}j\textcolor{BrickRed}{+}\textcolor{Purple}{1}\textcolor{BrickRed}{]}\ \textcolor{BrickRed}{=}\ \textbf{\textcolor{Black}{MAX}}\textcolor{BrickRed}{(}dp\textcolor{BrickRed}{[}i\textcolor{BrickRed}{+}\textcolor{Purple}{1}\textcolor{BrickRed}{][}j\textcolor{BrickRed}{],}\ dp\textcolor{BrickRed}{[}i\textcolor{BrickRed}{][}j\textcolor{BrickRed}{+}\textcolor{Purple}{1}\textcolor{BrickRed}{]);} \\
\mbox{}\ \ \textbf{\textcolor{Blue}{return}}\ dp\textcolor{BrickRed}{[}m\textcolor{BrickRed}{][}n\textcolor{BrickRed}{];} \\
\mbox{}\textcolor{Red}{\}} \\

} \normalfont\normalsize
%.tex

\section{Geometría}
\subsection{Área de un polígono}
Si P es un polígono simple (no se intersecta a sí mismo) su área está dada por: \\

$ A(P) = \frac{1}{2} \displaystyle\sum_{i=0}^{n-1} (x_{i} \cdot y_{i+1} - x_{i+1} \cdot y_{i}) $ \\

\subsection{Centro de masa de un polígono}
Si P es un polígono simple (no se intersecta a sí mismo) su centro de masa está dado por: \\

$ \displaystyle\bar{C}_{x} = \frac{ \displaystyle\iint_{R} x \, dA }{M} = \frac{1}{6M}\sum_{i=1}^{n} (y_{i+1} - y_{i}) (x_{i+1}^2 + x_{i+1} \cdot x_{i} + x_{i}^2) $

\medskip

$\displaystyle\bar{C}_{y} = \frac{ \displaystyle\iint_{R} y \, dA }{M} = \frac{1}{6M} \sum_{i=1}^{n} (x_{i} - x_{i+1}) (y_{i+1}^2 + y_{i+1} \cdot y_{i} + y_{i}^2)$

\medskip

Donde $ M $ es el área del polígono. \\

Otra posible fórmula equivalente:

$ \displaystyle\bar{C}_{x} = \frac{1}{6A} \sum_{i=0}^{n-1} (x_{i} + x_{i+1}) (x_{i} \cdot y_{i+1} - x_{i+1} \cdot y_{i}) $

\medskip

$ \displaystyle\bar{C}_{y} = \frac{1}{6A} \sum_{i=0}^{n-1} (y_{i} + y_{i+1}) (x_{i} \cdot y_{i+1} - x_{i+1} \cdot y_{i}) $


\subsection{Convex hull: Graham Scan}
\emph{Complejidad:} $ O(n \log_{2}{n}) $
% Generator: GNU source-highlight, by Lorenzo Bettini, http://www.gnu.org/software/src-highlite

{\ttfamily \raggedright {
\noindent
\mbox{}\textbf{\textcolor{RoyalBlue}{\#include}}\ \texttt{\textcolor{Red}{$<$iostream$>$}} \\
\mbox{}\textbf{\textcolor{RoyalBlue}{\#include}}\ \texttt{\textcolor{Red}{$<$vector$>$}} \\
\mbox{}\textbf{\textcolor{RoyalBlue}{\#include}}\ \texttt{\textcolor{Red}{$<$algorithm$>$}} \\
\mbox{}\textbf{\textcolor{RoyalBlue}{\#include}}\ \texttt{\textcolor{Red}{$<$iterator$>$}} \\
\mbox{}\textbf{\textcolor{RoyalBlue}{\#include}}\ \texttt{\textcolor{Red}{$<$cmath$>$}} \\
\mbox{} \\
\mbox{}\textbf{\textcolor{Blue}{using}}\ \textbf{\textcolor{Blue}{namespace}}\ std\textcolor{BrickRed}{;} \\
\mbox{} \\
\mbox{}\textbf{\textcolor{Blue}{struct}}\ point\textcolor{Red}{\{} \\
\mbox{}\ \ \textcolor{ForestGreen}{int}\ x\textcolor{BrickRed}{,}y\textcolor{BrickRed}{;} \\
\mbox{}\ \ \textbf{\textcolor{Black}{point}}\textcolor{BrickRed}{()}\ \textcolor{Red}{\{\}} \\
\mbox{}\ \ \textbf{\textcolor{Black}{point}}\textcolor{BrickRed}{(}\textcolor{ForestGreen}{int}\ X\textcolor{BrickRed}{,}\ \textcolor{ForestGreen}{int}\ Y\textcolor{BrickRed}{)}\ \textcolor{BrickRed}{:}\ \textbf{\textcolor{Black}{x}}\textcolor{BrickRed}{(}X\textcolor{BrickRed}{),}\ \textbf{\textcolor{Black}{y}}\textcolor{BrickRed}{(}Y\textcolor{BrickRed}{)}\ \textcolor{Red}{\{\}} \\
\mbox{}\textcolor{Red}{\}}\textcolor{BrickRed}{;} \\
\mbox{} \\
\mbox{}point\ pivot\textcolor{BrickRed}{;} \\
\mbox{} \\
\mbox{}ostream\textcolor{BrickRed}{\&}\ \textbf{\textcolor{Blue}{operator}}\textcolor{BrickRed}{$<$$<$}\ \textcolor{BrickRed}{(}ostream\textcolor{BrickRed}{\&}\ out\textcolor{BrickRed}{,}\ \textbf{\textcolor{Blue}{const}}\ point\textcolor{BrickRed}{\&}\ c\textcolor{BrickRed}{)} \\
\mbox{}\textcolor{Red}{\{} \\
\mbox{}\ \ out\ \textcolor{BrickRed}{$<$$<$}\ \texttt{\textcolor{Red}{"{}("{}}}\ \textcolor{BrickRed}{$<$$<$}\ c\textcolor{BrickRed}{.}x\ \textcolor{BrickRed}{$<$$<$}\ \texttt{\textcolor{Red}{"{},"{}}}\ \textcolor{BrickRed}{$<$$<$}\ c\textcolor{BrickRed}{.}y\ \textcolor{BrickRed}{$<$$<$}\ \texttt{\textcolor{Red}{"{})"{}}}\textcolor{BrickRed}{;} \\
\mbox{}\ \ \textbf{\textcolor{Blue}{return}}\ out\textcolor{BrickRed}{;} \\
\mbox{}\textcolor{Red}{\}} \\
\mbox{} \\
\mbox{}\textit{\textcolor{Brown}{//P\ es\ un\ polígono\ ordenado\ anticlockwise.}} \\
\mbox{}\textit{\textcolor{Brown}{//Si\ es\ clockwise,\ retorna\ el\ area\ negativa.}} \\
\mbox{}\textit{\textcolor{Brown}{//Si\ no\ esta\ ordenado\ retorna\ pura\ mierda}} \\
\mbox{}\textcolor{ForestGreen}{double}\ \textbf{\textcolor{Black}{area}}\textcolor{BrickRed}{(}\textbf{\textcolor{Blue}{const}}\ vector\textcolor{BrickRed}{$<$}point\textcolor{BrickRed}{$>$}\ \textcolor{BrickRed}{\&}p\textcolor{BrickRed}{)}\textcolor{Red}{\{} \\
\mbox{}\ \ \textcolor{ForestGreen}{double}\ r\ \textcolor{BrickRed}{=}\ \textcolor{Purple}{0.0}\textcolor{BrickRed}{;} \\
\mbox{}\ \ \textbf{\textcolor{Blue}{for}}\ \textcolor{BrickRed}{(}\textcolor{ForestGreen}{int}\ i\textcolor{BrickRed}{=}\textcolor{Purple}{0}\textcolor{BrickRed}{;}\ i\textcolor{BrickRed}{$<$}p\textcolor{BrickRed}{.}\textbf{\textcolor{Black}{size}}\textcolor{BrickRed}{();}\ \textcolor{BrickRed}{++}i\textcolor{BrickRed}{)}\textcolor{Red}{\{} \\
\mbox{}\ \ \ \ \textcolor{ForestGreen}{int}\ j\ \textcolor{BrickRed}{=}\ \textcolor{BrickRed}{(}i\textcolor{BrickRed}{+}\textcolor{Purple}{1}\textcolor{BrickRed}{)}\ \textcolor{BrickRed}{\%}\ p\textcolor{BrickRed}{.}\textbf{\textcolor{Black}{size}}\textcolor{BrickRed}{();} \\
\mbox{}\ \ \ \ r\ \textcolor{BrickRed}{+=}\ p\textcolor{BrickRed}{[}i\textcolor{BrickRed}{].}x\textcolor{BrickRed}{*}p\textcolor{BrickRed}{[}j\textcolor{BrickRed}{].}y\ \textcolor{BrickRed}{-}\ p\textcolor{BrickRed}{[}j\textcolor{BrickRed}{].}x\textcolor{BrickRed}{*}p\textcolor{BrickRed}{[}i\textcolor{BrickRed}{].}y\textcolor{BrickRed}{;} \\
\mbox{}\ \ \textcolor{Red}{\}} \\
\mbox{}\ \ \textbf{\textcolor{Blue}{return}}\ r\textcolor{BrickRed}{/}\textcolor{Purple}{2.0}\textcolor{BrickRed}{;} \\
\mbox{}\textcolor{Red}{\}} \\
\mbox{} \\
\mbox{}\textit{\textcolor{Brown}{//retorna\ si\ c\ esta\ a\ la\ izquierda\ de\ el\ segmento\ AB}} \\
\mbox{}\textbf{\textcolor{Blue}{inline}}\ \textcolor{ForestGreen}{int}\ \textbf{\textcolor{Black}{cross}}\textcolor{BrickRed}{(}\textbf{\textcolor{Blue}{const}}\ point\ \textcolor{BrickRed}{\&}a\textcolor{BrickRed}{,}\ \textbf{\textcolor{Blue}{const}}\ point\ \textcolor{BrickRed}{\&}b\textcolor{BrickRed}{,}\ \textbf{\textcolor{Blue}{const}}\ point\ \textcolor{BrickRed}{\&}c\textcolor{BrickRed}{)}\textcolor{Red}{\{} \\
\mbox{}\ \ \textbf{\textcolor{Blue}{return}}\ \textcolor{BrickRed}{(}b\textcolor{BrickRed}{.}x\textcolor{BrickRed}{-}a\textcolor{BrickRed}{.}x\textcolor{BrickRed}{)*(}c\textcolor{BrickRed}{.}y\textcolor{BrickRed}{-}a\textcolor{BrickRed}{.}y\textcolor{BrickRed}{)}\ \textcolor{BrickRed}{-}\ \textcolor{BrickRed}{(}c\textcolor{BrickRed}{.}x\textcolor{BrickRed}{-}a\textcolor{BrickRed}{.}x\textcolor{BrickRed}{)*(}b\textcolor{BrickRed}{.}y\textcolor{BrickRed}{-}a\textcolor{BrickRed}{.}y\textcolor{BrickRed}{);} \\
\mbox{}\textcolor{Red}{\}} \\
\mbox{} \\
\mbox{}\textit{\textcolor{Brown}{//Self\ $<$\ that\ si\ esta\ a\ la\ derecha\ del\ segmento\ Pivot-That}} \\
\mbox{}\textcolor{ForestGreen}{bool}\ \textbf{\textcolor{Black}{angleCmp}}\textcolor{BrickRed}{(}\textbf{\textcolor{Blue}{const}}\ point\ \textcolor{BrickRed}{\&}self\textcolor{BrickRed}{,}\ \textbf{\textcolor{Blue}{const}}\ point\ \textcolor{BrickRed}{\&}that\textcolor{BrickRed}{)}\textcolor{Red}{\{} \\
\mbox{}\ \ \textbf{\textcolor{Blue}{return}}\textcolor{BrickRed}{(}\ \textbf{\textcolor{Black}{cross}}\textcolor{BrickRed}{(}pivot\textcolor{BrickRed}{,}\ that\textcolor{BrickRed}{,}\ self\textcolor{BrickRed}{)}\ \textcolor{BrickRed}{$<$}\ \textcolor{Purple}{0}\ \textcolor{BrickRed}{);} \\
\mbox{}\textcolor{Red}{\}} \\
\mbox{} \\
\mbox{}\textbf{\textcolor{Blue}{inline}}\ \textcolor{ForestGreen}{int}\ \textbf{\textcolor{Black}{distsqr}}\textcolor{BrickRed}{(}\textbf{\textcolor{Blue}{const}}\ point\ \textcolor{BrickRed}{\&}a\textcolor{BrickRed}{,}\ \textbf{\textcolor{Blue}{const}}\ point\ \textcolor{BrickRed}{\&}b\textcolor{BrickRed}{)}\textcolor{Red}{\{} \\
\mbox{}\ \ \textbf{\textcolor{Blue}{return}}\ \textcolor{BrickRed}{(}a\textcolor{BrickRed}{.}x\ \textcolor{BrickRed}{-}\ b\textcolor{BrickRed}{.}x\textcolor{BrickRed}{)*(}a\textcolor{BrickRed}{.}x\ \textcolor{BrickRed}{-}\ b\textcolor{BrickRed}{.}x\textcolor{BrickRed}{)}\ \textcolor{BrickRed}{+}\ \textcolor{BrickRed}{(}a\textcolor{BrickRed}{.}y\ \textcolor{BrickRed}{-}\ b\textcolor{BrickRed}{.}y\textcolor{BrickRed}{)*(}a\textcolor{BrickRed}{.}y\ \textcolor{BrickRed}{-}\ b\textcolor{BrickRed}{.}y\textcolor{BrickRed}{);} \\
\mbox{}\textcolor{Red}{\}} \\
\mbox{} \\
\mbox{}\textit{\textcolor{Brown}{//vector\ p\ tiene\ los\ puntos\ ordenados\ anticlockwise}} \\
\mbox{}vector\textcolor{BrickRed}{$<$}point\textcolor{BrickRed}{$>$}\ \textbf{\textcolor{Black}{graham}}\textcolor{BrickRed}{(}vector\textcolor{BrickRed}{$<$}point\textcolor{BrickRed}{$>$}\ p\textcolor{BrickRed}{)}\textcolor{Red}{\{} \\
\mbox{}\ \ pivot\ \textcolor{BrickRed}{=}\ p\textcolor{BrickRed}{[}\textcolor{Purple}{0}\textcolor{BrickRed}{];} \\
\mbox{}\ \ \textbf{\textcolor{Black}{sort}}\textcolor{BrickRed}{(}p\textcolor{BrickRed}{.}\textbf{\textcolor{Black}{begin}}\textcolor{BrickRed}{(),}\ p\textcolor{BrickRed}{.}\textbf{\textcolor{Black}{end}}\textcolor{BrickRed}{(),}\ angleCmp\textcolor{BrickRed}{);} \\
\mbox{}\ \ \textit{\textcolor{Brown}{//Ordenar\ por\ ángulo\ y\ eliminar\ repetidos.}} \\
\mbox{}\ \ \textit{\textcolor{Brown}{//Si\ varios\ puntos\ tienen\ el\ mismo\ angulo\ se\ borran\ todos\ excepto\ el\ que\ esté\ más\ lejos}} \\
\mbox{}\ \ \textbf{\textcolor{Blue}{for}}\ \textcolor{BrickRed}{(}\textcolor{ForestGreen}{int}\ i\textcolor{BrickRed}{=}\textcolor{Purple}{1}\textcolor{BrickRed}{;}\ i\textcolor{BrickRed}{$<$}p\textcolor{BrickRed}{.}\textbf{\textcolor{Black}{size}}\textcolor{BrickRed}{()-}\textcolor{Purple}{1}\textcolor{BrickRed}{;}\ \textcolor{BrickRed}{++}i\textcolor{BrickRed}{)}\textcolor{Red}{\{}\ \ \ \  \\
\mbox{}\ \ \ \ \textbf{\textcolor{Blue}{if}}\ \textcolor{BrickRed}{(}\textbf{\textcolor{Black}{cross}}\textcolor{BrickRed}{(}p\textcolor{BrickRed}{[}\textcolor{Purple}{0}\textcolor{BrickRed}{],}\ p\textcolor{BrickRed}{[}i\textcolor{BrickRed}{],}\ p\textcolor{BrickRed}{[}i\textcolor{BrickRed}{+}\textcolor{Purple}{1}\textcolor{BrickRed}{])}\ \textcolor{BrickRed}{==}\ \textcolor{Purple}{0}\textcolor{BrickRed}{)}\textcolor{Red}{\{}\ \textit{\textcolor{Brown}{//Si\ son\ colineales...}} \\
\mbox{}\ \ \ \ \ \ \textbf{\textcolor{Blue}{if}}\ \textcolor{BrickRed}{(}\textbf{\textcolor{Black}{distsqr}}\textcolor{BrickRed}{(}p\textcolor{BrickRed}{[}\textcolor{Purple}{0}\textcolor{BrickRed}{],}\ p\textcolor{BrickRed}{[}i\textcolor{BrickRed}{])}\ \textcolor{BrickRed}{$<$}\ \textbf{\textcolor{Black}{distsqr}}\textcolor{BrickRed}{(}p\textcolor{BrickRed}{[}\textcolor{Purple}{0}\textcolor{BrickRed}{],}\ p\textcolor{BrickRed}{[}i\textcolor{BrickRed}{+}\textcolor{Purple}{1}\textcolor{BrickRed}{]))}\textcolor{Red}{\{}\ \textit{\textcolor{Brown}{//Borrar\ el\ mas\ cercano}} \\
\mbox{}\ \ \ \ \ \ \ \ p\textcolor{BrickRed}{.}\textbf{\textcolor{Black}{erase}}\textcolor{BrickRed}{(}p\textcolor{BrickRed}{.}\textbf{\textcolor{Black}{begin}}\textcolor{BrickRed}{()}\ \textcolor{BrickRed}{+}\ i\textcolor{BrickRed}{);} \\
\mbox{}\ \ \ \ \ \ \textcolor{Red}{\}}\textbf{\textcolor{Blue}{else}}\textcolor{Red}{\{} \\
\mbox{}\ \ \ \ \ \ \ \ p\textcolor{BrickRed}{.}\textbf{\textcolor{Black}{erase}}\textcolor{BrickRed}{(}p\textcolor{BrickRed}{.}\textbf{\textcolor{Black}{begin}}\textcolor{BrickRed}{()}\ \textcolor{BrickRed}{+}\ i\ \textcolor{BrickRed}{+}\ \textcolor{Purple}{1}\textcolor{BrickRed}{);} \\
\mbox{}\ \ \ \ \ \ \textcolor{Red}{\}} \\
\mbox{}\ \ \ \ \ \ i\textcolor{BrickRed}{-\/-;} \\
\mbox{}\ \ \ \ \textcolor{Red}{\}} \\
\mbox{}\ \ \textcolor{Red}{\}} \\
\mbox{}\ \  \\
\mbox{}\ \ vector\textcolor{BrickRed}{$<$}point\textcolor{BrickRed}{$>$}\ \textbf{\textcolor{Black}{chull}}\textcolor{BrickRed}{(}p\textcolor{BrickRed}{.}\textbf{\textcolor{Black}{begin}}\textcolor{BrickRed}{(),}\ p\textcolor{BrickRed}{.}\textbf{\textcolor{Black}{begin}}\textcolor{BrickRed}{()+}\textcolor{Purple}{3}\textcolor{BrickRed}{);} \\
\mbox{} \\
\mbox{}\ \ \textit{\textcolor{Brown}{//Ahora\ sí!!!}} \\
\mbox{}\ \ \textbf{\textcolor{Blue}{for}}\ \textcolor{BrickRed}{(}\textcolor{ForestGreen}{int}\ i\textcolor{BrickRed}{=}\textcolor{Purple}{3}\textcolor{BrickRed}{;}\ i\textcolor{BrickRed}{$<$}p\textcolor{BrickRed}{.}\textbf{\textcolor{Black}{size}}\textcolor{BrickRed}{();}\ \textcolor{BrickRed}{++}i\textcolor{BrickRed}{)}\textcolor{Red}{\{} \\
\mbox{}\ \ \ \ \textbf{\textcolor{Blue}{while}}\ \textcolor{BrickRed}{(}\ chull\textcolor{BrickRed}{.}\textbf{\textcolor{Black}{size}}\textcolor{BrickRed}{()}\ \textcolor{BrickRed}{$>$=}\ \textcolor{Purple}{2}\ \textcolor{BrickRed}{\&\&}\ \textbf{\textcolor{Black}{cross}}\textcolor{BrickRed}{(}chull\textcolor{BrickRed}{[}chull\textcolor{BrickRed}{.}\textbf{\textcolor{Black}{size}}\textcolor{BrickRed}{()-}\textcolor{Purple}{2}\textcolor{BrickRed}{],}\ chull\textcolor{BrickRed}{[}chull\textcolor{BrickRed}{.}\textbf{\textcolor{Black}{size}}\textcolor{BrickRed}{()-}\textcolor{Purple}{1}\textcolor{BrickRed}{],}\ p\textcolor{BrickRed}{[}i\textcolor{BrickRed}{])}\ \textcolor{BrickRed}{$<$}\ \textcolor{Purple}{0}\textcolor{BrickRed}{)}\textcolor{Red}{\{} \\
\mbox{}\ \ \ \ \ \ chull\textcolor{BrickRed}{.}\textbf{\textcolor{Black}{erase}}\textcolor{BrickRed}{(}chull\textcolor{BrickRed}{.}\textbf{\textcolor{Black}{end}}\textcolor{BrickRed}{()}\ \textcolor{BrickRed}{-}\ \textcolor{Purple}{1}\textcolor{BrickRed}{);} \\
\mbox{}\ \ \ \ \textcolor{Red}{\}} \\
\mbox{}\ \ \ \ chull\textcolor{BrickRed}{.}\textbf{\textcolor{Black}{push$\_$back}}\textcolor{BrickRed}{(}p\textcolor{BrickRed}{[}i\textcolor{BrickRed}{]);} \\
\mbox{}\ \ \textcolor{Red}{\}} \\
\mbox{} \\
\mbox{}\ \ \textbf{\textcolor{Blue}{return}}\ chull\textcolor{BrickRed}{;} \\
\mbox{}\textcolor{Red}{\}} \\
\mbox{} \\
\mbox{}\textcolor{ForestGreen}{int}\ \textbf{\textcolor{Black}{main}}\textcolor{BrickRed}{()}\textcolor{Red}{\{} \\
\mbox{}\ \ \textcolor{ForestGreen}{int}\ n\textcolor{BrickRed}{;} \\
\mbox{}\ \ \textcolor{ForestGreen}{int}\ tileNo\ \textcolor{BrickRed}{=}\ \textcolor{Purple}{1}\textcolor{BrickRed}{;} \\
\mbox{}\ \ \textbf{\textcolor{Blue}{while}}\ \textcolor{BrickRed}{(}cin\ \textcolor{BrickRed}{$>$$>$}\ n\ \textcolor{BrickRed}{\&\&}\ n\textcolor{BrickRed}{)}\textcolor{Red}{\{} \\
\mbox{}\ \ \ \ vector\textcolor{BrickRed}{$<$}point\textcolor{BrickRed}{$>$}\ p\textcolor{BrickRed}{;} \\
\mbox{}\ \ \ \ point\ \textbf{\textcolor{Black}{min}}\textcolor{BrickRed}{(}\textcolor{Purple}{10000}\textcolor{BrickRed}{,}\ \textcolor{Purple}{10000}\textcolor{BrickRed}{);} \\
\mbox{}\ \ \ \ \textcolor{ForestGreen}{int}\ minPos\textcolor{BrickRed}{;} \\
\mbox{}\ \ \ \ \textbf{\textcolor{Blue}{for}}\ \textcolor{BrickRed}{(}\textcolor{ForestGreen}{int}\ i\textcolor{BrickRed}{=}\textcolor{Purple}{0}\textcolor{BrickRed}{;}\ i\textcolor{BrickRed}{$<$}n\textcolor{BrickRed}{;}\ \textcolor{BrickRed}{++}i\textcolor{BrickRed}{)}\textcolor{Red}{\{} \\
\mbox{}\ \ \ \ \ \ \textcolor{ForestGreen}{int}\ x\textcolor{BrickRed}{,}\ y\textcolor{BrickRed}{;} \\
\mbox{}\ \ \ \ \ \ cin\ \textcolor{BrickRed}{$>$$>$}\ x\ \textcolor{BrickRed}{$>$$>$}\ y\textcolor{BrickRed}{;} \\
\mbox{}\ \ \ \ \ \ p\textcolor{BrickRed}{.}\textbf{\textcolor{Black}{push$\_$back}}\textcolor{BrickRed}{(}\textbf{\textcolor{Black}{point}}\textcolor{BrickRed}{(}x\textcolor{BrickRed}{,}y\textcolor{BrickRed}{));}\ \ \ \ \ \  \\
\mbox{}\ \ \ \ \ \ \textbf{\textcolor{Blue}{if}}\ \textcolor{BrickRed}{(}y\ \textcolor{BrickRed}{$<$}\ min\textcolor{BrickRed}{.}y\ \textcolor{BrickRed}{$|$$|$}\ \textcolor{BrickRed}{(}y\ \textcolor{BrickRed}{==}\ min\textcolor{BrickRed}{.}y\ \textcolor{BrickRed}{\&\&}\ x\ \textcolor{BrickRed}{$<$}\ min\textcolor{BrickRed}{.}x\ \textcolor{BrickRed}{))}\textcolor{Red}{\{} \\
\mbox{}\ \ \ \ \ \ \ \ min\ \textcolor{BrickRed}{=}\ \textbf{\textcolor{Black}{point}}\textcolor{BrickRed}{(}x\textcolor{BrickRed}{,}y\textcolor{BrickRed}{);} \\
\mbox{}\ \ \ \ \ \ \ \ minPos\ \textcolor{BrickRed}{=}\ i\textcolor{BrickRed}{;} \\
\mbox{}\ \ \ \ \ \ \textcolor{Red}{\}} \\
\mbox{}\ \ \ \ \textcolor{Red}{\}} \\
\mbox{}\ \ \ \ \textcolor{ForestGreen}{double}\ tileArea\ \textcolor{BrickRed}{=}\ \textbf{\textcolor{Black}{fabs}}\textcolor{BrickRed}{(}\textbf{\textcolor{Black}{area}}\textcolor{BrickRed}{(}p\textcolor{BrickRed}{));} \\
\mbox{} \\
\mbox{}\ \ \ \ \textit{\textcolor{Brown}{//Destruye\ el\ orden\ cw$|$ccw\ poligono,\ pero\ hay\ que\ hacerlo\ por\ que\ Graham\ necesita\ el\ pivote\ en\ p[0]}} \\
\mbox{}\ \ \ \ \textbf{\textcolor{Black}{swap}}\textcolor{BrickRed}{(}p\textcolor{BrickRed}{[}\textcolor{Purple}{0}\textcolor{BrickRed}{],}\ p\textcolor{BrickRed}{[}minPos\textcolor{BrickRed}{]);} \\
\mbox{}\ \ \ \ pivot\ \textcolor{BrickRed}{=}\ p\textcolor{BrickRed}{[}\textcolor{Purple}{0}\textcolor{BrickRed}{];} \\
\mbox{}\ \ \ \ \textcolor{ForestGreen}{double}\ chullArea\ \textcolor{BrickRed}{=}\ \textbf{\textcolor{Black}{fabs}}\textcolor{BrickRed}{(}\textbf{\textcolor{Black}{area}}\textcolor{BrickRed}{(}\textbf{\textcolor{Black}{graham}}\textcolor{BrickRed}{(}p\textcolor{BrickRed}{)));} \\
\mbox{}\ \ \ \  \\
\mbox{}\ \ \ \ \textbf{\textcolor{Black}{printf}}\textcolor{BrickRed}{(}\texttt{\textcolor{Red}{"{}Tile\ \#\%d}}\texttt{\textcolor{CarnationPink}{\textbackslash{}n}}\texttt{\textcolor{Red}{"{}}}\textcolor{BrickRed}{,}\ tileNo\textcolor{BrickRed}{++);} \\
\mbox{}\ \ \ \ \textbf{\textcolor{Black}{printf}}\textcolor{BrickRed}{(}\texttt{\textcolor{Red}{"{}Wasted\ Space\ =\ \%.2f\ }}\texttt{\textcolor{CarnationPink}{\textbackslash{}\%\textbackslash{}n\textbackslash{}n}}\texttt{\textcolor{Red}{"{}}}\textcolor{BrickRed}{,}\ \ \textcolor{BrickRed}{(}chullArea\ \textcolor{BrickRed}{-}\ tileArea\textcolor{BrickRed}{)}\ \textcolor{BrickRed}{*}\ \textcolor{Purple}{100.0}\ \textcolor{BrickRed}{/}\ chullArea\textcolor{BrickRed}{);} \\
\mbox{}\ \ \  \\
\mbox{}\ \ \textcolor{Red}{\}} \\
\mbox{}\ \ \textbf{\textcolor{Blue}{return}}\ \textcolor{Purple}{0}\textcolor{BrickRed}{;} \\
\mbox{}\textcolor{Red}{\}} \\

} \normalfont\normalsize
%.tex

\subsection{Convex hull: Andrew's monotone chain}
\emph{Complejidad:} $ O(n \log_{2}{n}) $
% Generator: GNU source-highlight, by Lorenzo Bettini, http://www.gnu.org/software/src-highlite

{\ttfamily \raggedright {
\noindent
\mbox{}\textit{\textcolor{Brown}{//\ Implementation\ of\ Monotone\ Chain\ Convex\ Hull\ algorithm.}} \\
\mbox{}\textbf{\textcolor{RoyalBlue}{\#include}}\ \texttt{\textcolor{Red}{$<$algorithm$>$}} \\
\mbox{}\textbf{\textcolor{RoyalBlue}{\#include}}\ \texttt{\textcolor{Red}{$<$vector$>$}} \\
\mbox{}\textbf{\textcolor{Blue}{using}}\ \textbf{\textcolor{Blue}{namespace}}\ std\textcolor{BrickRed}{;} \\
\mbox{}\  \\
\mbox{}\textbf{\textcolor{Blue}{typedef}}\ \textcolor{ForestGreen}{long}\ \textcolor{ForestGreen}{long}\ CoordType\textcolor{BrickRed}{;} \\
\mbox{}\  \\
\mbox{}\textbf{\textcolor{Blue}{struct}}\ Point\ \textcolor{Red}{\{} \\
\mbox{}\ \ \ \ \ \ \ \ CoordType\ x\textcolor{BrickRed}{,}\ y\textcolor{BrickRed}{;} \\
\mbox{}\  \\
\mbox{}\ \ \ \ \ \ \ \ \textcolor{ForestGreen}{bool}\ \textbf{\textcolor{Blue}{operator}}\ \textcolor{BrickRed}{$<$(}\textbf{\textcolor{Blue}{const}}\ Point\ \textcolor{BrickRed}{\&}p\textcolor{BrickRed}{)}\ \textbf{\textcolor{Blue}{const}}\ \textcolor{Red}{\{} \\
\mbox{}\ \ \ \ \ \ \ \ \ \ \ \ \ \ \ \ \textbf{\textcolor{Blue}{return}}\ x\ \textcolor{BrickRed}{$<$}\ p\textcolor{BrickRed}{.}x\ \textcolor{BrickRed}{$|$$|$}\ \textcolor{BrickRed}{(}x\ \textcolor{BrickRed}{==}\ p\textcolor{BrickRed}{.}x\ \textcolor{BrickRed}{\&\&}\ y\ \textcolor{BrickRed}{$<$}\ p\textcolor{BrickRed}{.}y\textcolor{BrickRed}{);} \\
\mbox{}\ \ \ \ \ \ \ \ \textcolor{Red}{\}} \\
\mbox{}\textcolor{Red}{\}}\textcolor{BrickRed}{;} \\
\mbox{}\  \\
\mbox{}\textit{\textcolor{Brown}{//\ 2D\ cross\ product.}} \\
\mbox{}\textit{\textcolor{Brown}{//\ Return\ a\ positive\ value,\ if\ OAB\ makes\ a\ counter-clockwise\ turn,}} \\
\mbox{}\textit{\textcolor{Brown}{//\ negative\ for\ clockwise\ turn,\ and\ zero\ if\ the\ points\ are\ collinear.}} \\
\mbox{}CoordType\ \textbf{\textcolor{Black}{cross}}\textcolor{BrickRed}{(}\textbf{\textcolor{Blue}{const}}\ Point\ \textcolor{BrickRed}{\&}O\textcolor{BrickRed}{,}\ \textbf{\textcolor{Blue}{const}}\ Point\ \textcolor{BrickRed}{\&}A\textcolor{BrickRed}{,}\ \textbf{\textcolor{Blue}{const}}\ Point\ \textcolor{BrickRed}{\&}B\textcolor{BrickRed}{)} \\
\mbox{}\textcolor{Red}{\{} \\
\mbox{}\ \ \ \ \ \ \ \ \textbf{\textcolor{Blue}{return}}\ \textcolor{BrickRed}{(}A\textcolor{BrickRed}{.}x\ \textcolor{BrickRed}{-}\ O\textcolor{BrickRed}{.}x\textcolor{BrickRed}{)}\ \textcolor{BrickRed}{*}\ \textcolor{BrickRed}{(}B\textcolor{BrickRed}{.}y\ \textcolor{BrickRed}{-}\ O\textcolor{BrickRed}{.}y\textcolor{BrickRed}{)}\ \textcolor{BrickRed}{-}\ \textcolor{BrickRed}{(}A\textcolor{BrickRed}{.}y\ \textcolor{BrickRed}{-}\ O\textcolor{BrickRed}{.}y\textcolor{BrickRed}{)}\ \textcolor{BrickRed}{*}\ \textcolor{BrickRed}{(}B\textcolor{BrickRed}{.}x\ \textcolor{BrickRed}{-}\ O\textcolor{BrickRed}{.}x\textcolor{BrickRed}{);} \\
\mbox{}\textcolor{Red}{\}} \\
\mbox{}\  \\
\mbox{}\textit{\textcolor{Brown}{//\ Returns\ a\ list\ of\ points\ on\ the\ convex\ hull\ in\ counter-clockwise\ order.}} \\
\mbox{}\textit{\textcolor{Brown}{//\ Note:\ the\ last\ point\ in\ the\ returned\ list\ is\ the\ same\ as\ the\ first\ one.}} \\
\mbox{}vector\textcolor{BrickRed}{$<$}Point\textcolor{BrickRed}{$>$}\ \textbf{\textcolor{Black}{convexHull}}\textcolor{BrickRed}{(}vector\textcolor{BrickRed}{$<$}Point\textcolor{BrickRed}{$>$}\ P\textcolor{BrickRed}{)} \\
\mbox{}\textcolor{Red}{\{} \\
\mbox{}\ \ \ \ \ \ \ \ \textcolor{ForestGreen}{int}\ n\ \textcolor{BrickRed}{=}\ P\textcolor{BrickRed}{.}\textbf{\textcolor{Black}{size}}\textcolor{BrickRed}{(),}\ k\ \textcolor{BrickRed}{=}\ \textcolor{Purple}{0}\textcolor{BrickRed}{;} \\
\mbox{}\ \ \ \ \ \ \ \ vector\textcolor{BrickRed}{$<$}Point\textcolor{BrickRed}{$>$}\ \textbf{\textcolor{Black}{H}}\textcolor{BrickRed}{(}\textcolor{Purple}{2}\textcolor{BrickRed}{*}n\textcolor{BrickRed}{);} \\
\mbox{}\  \\
\mbox{}\ \ \ \ \ \ \ \ \textit{\textcolor{Brown}{//\ Sort\ points\ lexicographically}} \\
\mbox{}\ \ \ \ \ \ \ \ \textbf{\textcolor{Black}{sort}}\textcolor{BrickRed}{(}P\textcolor{BrickRed}{.}\textbf{\textcolor{Black}{begin}}\textcolor{BrickRed}{(),}\ P\textcolor{BrickRed}{.}\textbf{\textcolor{Black}{end}}\textcolor{BrickRed}{());} \\
\mbox{}\  \\
\mbox{}\ \ \ \ \ \ \ \ \textit{\textcolor{Brown}{//\ Build\ lower\ hull}} \\
\mbox{}\ \ \ \ \ \ \ \ \textbf{\textcolor{Blue}{for}}\ \textcolor{BrickRed}{(}\textcolor{ForestGreen}{int}\ i\ \textcolor{BrickRed}{=}\ \textcolor{Purple}{0}\textcolor{BrickRed}{;}\ i\ \textcolor{BrickRed}{$<$}\ n\textcolor{BrickRed}{;}\ i\textcolor{BrickRed}{++)}\ \textcolor{Red}{\{} \\
\mbox{}\ \ \ \ \ \ \ \ \ \ \ \ \ \ \ \ \textbf{\textcolor{Blue}{while}}\ \textcolor{BrickRed}{(}k\ \textcolor{BrickRed}{$>$=}\ \textcolor{Purple}{2}\ \textcolor{BrickRed}{\&\&}\ \textbf{\textcolor{Black}{cross}}\textcolor{BrickRed}{(}H\textcolor{BrickRed}{[}k\textcolor{BrickRed}{-}\textcolor{Purple}{2}\textcolor{BrickRed}{],}\ H\textcolor{BrickRed}{[}k\textcolor{BrickRed}{-}\textcolor{Purple}{1}\textcolor{BrickRed}{],}\ P\textcolor{BrickRed}{[}i\textcolor{BrickRed}{])}\ \textcolor{BrickRed}{$<$=}\ \textcolor{Purple}{0}\textcolor{BrickRed}{)}\ k\textcolor{BrickRed}{-\/-;} \\
\mbox{}\ \ \ \ \ \ \ \ \ \ \ \ \ \ \ \ H\textcolor{BrickRed}{[}k\textcolor{BrickRed}{++]}\ \textcolor{BrickRed}{=}\ P\textcolor{BrickRed}{[}i\textcolor{BrickRed}{];} \\
\mbox{}\ \ \ \ \ \ \ \ \textcolor{Red}{\}} \\
\mbox{}\  \\
\mbox{}\ \ \ \ \ \ \ \ \textit{\textcolor{Brown}{//\ Build\ upper\ hull}} \\
\mbox{}\ \ \ \ \ \ \ \ \textbf{\textcolor{Blue}{for}}\ \textcolor{BrickRed}{(}\textcolor{ForestGreen}{int}\ i\ \textcolor{BrickRed}{=}\ n\textcolor{BrickRed}{-}\textcolor{Purple}{2}\textcolor{BrickRed}{,}\ t\ \textcolor{BrickRed}{=}\ k\textcolor{BrickRed}{+}\textcolor{Purple}{1}\textcolor{BrickRed}{;}\ i\ \textcolor{BrickRed}{$>$=}\ \textcolor{Purple}{0}\textcolor{BrickRed}{;}\ i\textcolor{BrickRed}{-\/-)}\ \textcolor{Red}{\{} \\
\mbox{}\ \ \ \ \ \ \ \ \ \ \ \ \ \ \ \ \textbf{\textcolor{Blue}{while}}\ \textcolor{BrickRed}{(}k\ \textcolor{BrickRed}{$>$=}\ t\ \textcolor{BrickRed}{\&\&}\ \textbf{\textcolor{Black}{cross}}\textcolor{BrickRed}{(}H\textcolor{BrickRed}{[}k\textcolor{BrickRed}{-}\textcolor{Purple}{2}\textcolor{BrickRed}{],}\ H\textcolor{BrickRed}{[}k\textcolor{BrickRed}{-}\textcolor{Purple}{1}\textcolor{BrickRed}{],}\ P\textcolor{BrickRed}{[}i\textcolor{BrickRed}{])}\ \textcolor{BrickRed}{$<$=}\ \textcolor{Purple}{0}\textcolor{BrickRed}{)}\ k\textcolor{BrickRed}{-\/-;} \\
\mbox{}\ \ \ \ \ \ \ \ \ \ \ \ \ \ \ \ H\textcolor{BrickRed}{[}k\textcolor{BrickRed}{++]}\ \textcolor{BrickRed}{=}\ P\textcolor{BrickRed}{[}i\textcolor{BrickRed}{];} \\
\mbox{}\ \ \ \ \ \ \ \ \textcolor{Red}{\}} \\
\mbox{}\  \\
\mbox{}\ \ \ \ \ \ \ \ H\textcolor{BrickRed}{.}\textbf{\textcolor{Black}{resize}}\textcolor{BrickRed}{(}k\textcolor{BrickRed}{);} \\
\mbox{}\ \ \ \ \ \ \ \ \textbf{\textcolor{Blue}{return}}\ H\textcolor{BrickRed}{;} \\
\mbox{}\textcolor{Red}{\}} \\

} \normalfont\normalsize
%.tex

\subsection{Mínima distancia entre un punto y un segmento}
% Generator: GNU source-highlight, by Lorenzo Bettini, http://www.gnu.org/software/src-highlite

{\ttfamily \raggedright {
\noindent
\mbox{}\textbf{\textcolor{Blue}{struct}}\ point\textcolor{Red}{\{} \\
\mbox{}\ \ \textcolor{ForestGreen}{double}\ x\textcolor{BrickRed}{,}y\textcolor{BrickRed}{;} \\
\mbox{}\textcolor{Red}{\}}\textcolor{BrickRed}{;} \\
\mbox{} \\
\mbox{}\textbf{\textcolor{Blue}{inline}}\ \textcolor{ForestGreen}{double}\ \textbf{\textcolor{Black}{dist}}\textcolor{BrickRed}{(}\textbf{\textcolor{Blue}{const}}\ point\ \textcolor{BrickRed}{\&}a\textcolor{BrickRed}{,}\ \textbf{\textcolor{Blue}{const}}\ point\ \textcolor{BrickRed}{\&}b\textcolor{BrickRed}{)}\textcolor{Red}{\{} \\
\mbox{}\ \ \textbf{\textcolor{Blue}{return}}\ \textbf{\textcolor{Black}{sqrt}}\textcolor{BrickRed}{((}a\textcolor{BrickRed}{.}x\textcolor{BrickRed}{-}b\textcolor{BrickRed}{.}x\textcolor{BrickRed}{)*(}a\textcolor{BrickRed}{.}x\textcolor{BrickRed}{-}b\textcolor{BrickRed}{.}x\textcolor{BrickRed}{)}\ \textcolor{BrickRed}{+}\ \textcolor{BrickRed}{(}a\textcolor{BrickRed}{.}y\textcolor{BrickRed}{-}b\textcolor{BrickRed}{.}y\textcolor{BrickRed}{)*(}a\textcolor{BrickRed}{.}y\textcolor{BrickRed}{-}b\textcolor{BrickRed}{.}y\textcolor{BrickRed}{));} \\
\mbox{}\textcolor{Red}{\}} \\
\mbox{} \\
\mbox{}\textbf{\textcolor{Blue}{inline}}\ \textcolor{ForestGreen}{double}\ \textbf{\textcolor{Black}{distsqr}}\textcolor{BrickRed}{(}\textbf{\textcolor{Blue}{const}}\ point\ \textcolor{BrickRed}{\&}a\textcolor{BrickRed}{,}\ \textbf{\textcolor{Blue}{const}}\ point\ \textcolor{BrickRed}{\&}b\textcolor{BrickRed}{)}\textcolor{Red}{\{} \\
\mbox{}\ \ \textbf{\textcolor{Blue}{return}}\ \textcolor{BrickRed}{(}a\textcolor{BrickRed}{.}x\textcolor{BrickRed}{-}b\textcolor{BrickRed}{.}x\textcolor{BrickRed}{)*(}a\textcolor{BrickRed}{.}x\textcolor{BrickRed}{-}b\textcolor{BrickRed}{.}x\textcolor{BrickRed}{)}\ \textcolor{BrickRed}{+}\ \textcolor{BrickRed}{(}a\textcolor{BrickRed}{.}y\textcolor{BrickRed}{-}b\textcolor{BrickRed}{.}y\textcolor{BrickRed}{)*(}a\textcolor{BrickRed}{.}y\textcolor{BrickRed}{-}b\textcolor{BrickRed}{.}y\textcolor{BrickRed}{);} \\
\mbox{}\textcolor{Red}{\}} \\
\mbox{} \\
\mbox{}\textit{\textcolor{Brown}{/*}} \\
\mbox{}\textit{\textcolor{Brown}{\ \ Returns\ the\ closest\ distance\ between\ point\ pnt\ and\ the\ segment\ that\ goes\ from\ point\ a\ to\ b}} \\
\mbox{}\textit{\textcolor{Brown}{\ \ Idea\ by:\ }}\underline{\texttt{\textcolor{Blue}{http://local.wasp.uwa.edu.au/}}}\textit{\textcolor{Brown}{\textasciitilde{}pbourke/geometry/pointline/}} \\
\mbox{}\textit{\textcolor{Brown}{\ */}} \\
\mbox{}\textcolor{ForestGreen}{double}\ \textbf{\textcolor{Black}{distance$\_$point$\_$to$\_$segment}}\textcolor{BrickRed}{(}\textbf{\textcolor{Blue}{const}}\ point\ \textcolor{BrickRed}{\&}a\textcolor{BrickRed}{,}\ \textbf{\textcolor{Blue}{const}}\ point\ \textcolor{BrickRed}{\&}b\textcolor{BrickRed}{,}\ \textbf{\textcolor{Blue}{const}}\ point\ \textcolor{BrickRed}{\&}pnt\textcolor{BrickRed}{)}\textcolor{Red}{\{} \\
\mbox{}\ \ \textcolor{ForestGreen}{double}\ u\ \textcolor{BrickRed}{=}\ \textcolor{BrickRed}{((}pnt\textcolor{BrickRed}{.}x\ \textcolor{BrickRed}{-}\ a\textcolor{BrickRed}{.}x\textcolor{BrickRed}{)*(}b\textcolor{BrickRed}{.}x\ \textcolor{BrickRed}{-}\ a\textcolor{BrickRed}{.}x\textcolor{BrickRed}{)}\ \textcolor{BrickRed}{+}\ \textcolor{BrickRed}{(}pnt\textcolor{BrickRed}{.}y\ \textcolor{BrickRed}{-}\ a\textcolor{BrickRed}{.}y\textcolor{BrickRed}{)*(}b\textcolor{BrickRed}{.}y\ \textcolor{BrickRed}{-}\ a\textcolor{BrickRed}{.}y\textcolor{BrickRed}{))}\ \textcolor{BrickRed}{/}\ \textbf{\textcolor{Black}{distsqr}}\textcolor{BrickRed}{(}a\textcolor{BrickRed}{,}\ b\textcolor{BrickRed}{);} \\
\mbox{}\ \ point\ intersection\textcolor{BrickRed}{;} \\
\mbox{}\ \ intersection\textcolor{BrickRed}{.}x\ \textcolor{BrickRed}{=}\ a\textcolor{BrickRed}{.}x\ \textcolor{BrickRed}{+}\ u\textcolor{BrickRed}{*(}b\textcolor{BrickRed}{.}x\ \textcolor{BrickRed}{-}\ a\textcolor{BrickRed}{.}x\textcolor{BrickRed}{);} \\
\mbox{}\ \ intersection\textcolor{BrickRed}{.}y\ \textcolor{BrickRed}{=}\ a\textcolor{BrickRed}{.}y\ \textcolor{BrickRed}{+}\ u\textcolor{BrickRed}{*(}b\textcolor{BrickRed}{.}y\ \textcolor{BrickRed}{-}\ a\textcolor{BrickRed}{.}y\textcolor{BrickRed}{);} \\
\mbox{}\ \ \textbf{\textcolor{Blue}{if}}\ \textcolor{BrickRed}{(}u\ \textcolor{BrickRed}{$<$}\ \textcolor{Purple}{0.0}\ \textcolor{BrickRed}{$|$$|$}\ u\ \textcolor{BrickRed}{$>$}\ \textcolor{Purple}{1.0}\textcolor{BrickRed}{)}\textcolor{Red}{\{} \\
\mbox{}\ \ \ \ \textbf{\textcolor{Blue}{return}}\ \textbf{\textcolor{Black}{min}}\textcolor{BrickRed}{(}\textbf{\textcolor{Black}{dist}}\textcolor{BrickRed}{(}a\textcolor{BrickRed}{,}\ pnt\textcolor{BrickRed}{),}\ \textbf{\textcolor{Black}{dist}}\textcolor{BrickRed}{(}b\textcolor{BrickRed}{,}\ pnt\textcolor{BrickRed}{));} \\
\mbox{}\ \ \textcolor{Red}{\}} \\
\mbox{}\ \ \textbf{\textcolor{Blue}{return}}\ \textbf{\textcolor{Black}{dist}}\textcolor{BrickRed}{(}pnt\textcolor{BrickRed}{,}\ intersection\textcolor{BrickRed}{);} \\
\mbox{}\textcolor{Red}{\}} \\

} \normalfont\normalsize
%.tex

\subsection{Mínima distancia entre un punto y una recta}
% Generator: GNU source-highlight, by Lorenzo Bettini, http://www.gnu.org/software/src-highlite

{\ttfamily \raggedright {
\noindent
\mbox{}\textit{\textcolor{Brown}{/*}} \\
\mbox{}\textit{\textcolor{Brown}{\ \ Returns\ the\ closest\ distance\ between\ point\ pnt\ and\ the\ line\ that\ passes\ through\ points\ a\ and\ b}} \\
\mbox{}\textit{\textcolor{Brown}{\ \ Idea\ by:\ }}\underline{\texttt{\textcolor{Blue}{http://local.wasp.uwa.edu.au/}}}\textit{\textcolor{Brown}{\textasciitilde{}pbourke/geometry/pointline/}} \\
\mbox{}\textit{\textcolor{Brown}{\ */}} \\
\mbox{}\textcolor{ForestGreen}{double}\ \textbf{\textcolor{Black}{distance$\_$point$\_$to$\_$line}}\textcolor{BrickRed}{(}\textbf{\textcolor{Blue}{const}}\ point\ \textcolor{BrickRed}{\&}a\textcolor{BrickRed}{,}\ \textbf{\textcolor{Blue}{const}}\ point\ \textcolor{BrickRed}{\&}b\textcolor{BrickRed}{,}\ \textbf{\textcolor{Blue}{const}}\ point\ \textcolor{BrickRed}{\&}pnt\textcolor{BrickRed}{)}\textcolor{Red}{\{} \\
\mbox{}\ \ \textcolor{ForestGreen}{double}\ u\ \textcolor{BrickRed}{=}\ \textcolor{BrickRed}{((}pnt\textcolor{BrickRed}{.}x\ \textcolor{BrickRed}{-}\ a\textcolor{BrickRed}{.}x\textcolor{BrickRed}{)*(}b\textcolor{BrickRed}{.}x\ \textcolor{BrickRed}{-}\ a\textcolor{BrickRed}{.}x\textcolor{BrickRed}{)}\ \textcolor{BrickRed}{+}\ \textcolor{BrickRed}{(}pnt\textcolor{BrickRed}{.}y\ \textcolor{BrickRed}{-}\ a\textcolor{BrickRed}{.}y\textcolor{BrickRed}{)*(}b\textcolor{BrickRed}{.}y\ \textcolor{BrickRed}{-}\ a\textcolor{BrickRed}{.}y\textcolor{BrickRed}{))}\ \textcolor{BrickRed}{/}\ \textbf{\textcolor{Black}{distsqr}}\textcolor{BrickRed}{(}a\textcolor{BrickRed}{,}\ b\textcolor{BrickRed}{);} \\
\mbox{}\ \ point\ intersection\textcolor{BrickRed}{;} \\
\mbox{}\ \ intersection\textcolor{BrickRed}{.}x\ \textcolor{BrickRed}{=}\ a\textcolor{BrickRed}{.}x\ \textcolor{BrickRed}{+}\ u\textcolor{BrickRed}{*(}b\textcolor{BrickRed}{.}x\ \textcolor{BrickRed}{-}\ a\textcolor{BrickRed}{.}x\textcolor{BrickRed}{);} \\
\mbox{}\ \ intersection\textcolor{BrickRed}{.}y\ \textcolor{BrickRed}{=}\ a\textcolor{BrickRed}{.}y\ \textcolor{BrickRed}{+}\ u\textcolor{BrickRed}{*(}b\textcolor{BrickRed}{.}y\ \textcolor{BrickRed}{-}\ a\textcolor{BrickRed}{.}y\textcolor{BrickRed}{);} \\
\mbox{}\ \ \textbf{\textcolor{Blue}{return}}\ \textbf{\textcolor{Black}{dist}}\textcolor{BrickRed}{(}pnt\textcolor{BrickRed}{,}\ intersection\textcolor{BrickRed}{);} \\
\mbox{}\textcolor{Red}{\}} \\

} \normalfont\normalsize
%.tex
%---------------------------------------------------------------

\section{Java}
\subsection{Entrada desde entrada estándar}
Este primer método es muy fácil pero es mucho más ineficiente porque utiliza Scanner en vez de BufferedReader: \\
% Generator: GNU source-highlight, by Lorenzo Bettini, http://www.gnu.org/software/src-highlite

{\ttfamily \raggedright {
\noindent
\mbox{}\textbf{\textcolor{RoyalBlue}{import}}\ java\textcolor{BrickRed}{.}io\textcolor{BrickRed}{.*;} \\
\mbox{}\textbf{\textcolor{RoyalBlue}{import}}\ java\textcolor{BrickRed}{.}util\textcolor{BrickRed}{.*;} \\
\mbox{} \\
\mbox{}\textbf{\textcolor{Blue}{class}}\ Main\textcolor{Red}{\{} \\
\mbox{}\ \ \ \ \textbf{\textcolor{Blue}{public}}\ \textbf{\textcolor{Blue}{static}}\ \textcolor{ForestGreen}{void}\ \textbf{\textcolor{Black}{main}}\textcolor{BrickRed}{(}String\textcolor{BrickRed}{[]}\ args\textcolor{BrickRed}{)}\textcolor{Red}{\{} \\
\mbox{}\ \ \ \ \ \ \ \ Scanner\ sc\ \textcolor{BrickRed}{=}\ \textbf{\textcolor{Blue}{new}}\ \textbf{\textcolor{Black}{Scanner}}\textcolor{BrickRed}{(}System\textcolor{BrickRed}{.}in\textcolor{BrickRed}{);} \\
\mbox{}\ \ \ \ \ \ \ \ \textbf{\textcolor{Blue}{while}}\ \textcolor{BrickRed}{(}sc\textcolor{BrickRed}{.}\textbf{\textcolor{Black}{hasNextLine}}\textcolor{BrickRed}{())}\textcolor{Red}{\{} \\
\mbox{}\ \ \ \ \ \ \ \ \ \ \ \ String\ s\textcolor{BrickRed}{=}\ sc\textcolor{BrickRed}{.}\textbf{\textcolor{Black}{nextLine}}\textcolor{BrickRed}{();} \\
\mbox{}\ \ \ \ \ \ \ \ \ \ \ \ System\textcolor{BrickRed}{.}out\textcolor{BrickRed}{.}\textbf{\textcolor{Black}{println}}\textcolor{BrickRed}{(}\texttt{\textcolor{Red}{"{}Leí:\ "{}}}\ \textcolor{BrickRed}{+}\ s\textcolor{BrickRed}{);} \\
\mbox{}\ \ \ \ \ \ \ \ \textcolor{Red}{\}} \\
\mbox{}\ \ \ \ \textcolor{Red}{\}} \\
\mbox{}\textcolor{Red}{\}}
} \normalfont\normalsize
%.tex

\bigskip

Este segundo es más rápido: \\
% Generator: GNU source-highlight, by Lorenzo Bettini, http://www.gnu.org/software/src-highlite

{\ttfamily \raggedright {
\noindent
\mbox{}\textbf{\textcolor{RoyalBlue}{import}}\ java\textcolor{BrickRed}{.}util\textcolor{BrickRed}{.*;} \\
\mbox{}\textbf{\textcolor{RoyalBlue}{import}}\ java\textcolor{BrickRed}{.}io\textcolor{BrickRed}{.*;} \\
\mbox{}\textbf{\textcolor{RoyalBlue}{import}}\ java\textcolor{BrickRed}{.}math\textcolor{BrickRed}{.*;} \\
\mbox{}\  \\
\mbox{}\textbf{\textcolor{Blue}{class}}\ Main\ \textcolor{Red}{\{} \\
\mbox{}\ \ \ \ \textbf{\textcolor{Blue}{public}}\ \textbf{\textcolor{Blue}{static}}\ \textcolor{ForestGreen}{void}\ \textbf{\textcolor{Black}{main}}\textcolor{BrickRed}{(}String\textcolor{BrickRed}{[]}\ args\textcolor{BrickRed}{)}\ \textbf{\textcolor{Blue}{throws}}\ IOException\ \textcolor{Red}{\{} \\
\mbox{}\ \ \ \ \ \ \ \ BufferedReader\ reader\ \textcolor{BrickRed}{=}\ \textbf{\textcolor{Blue}{new}}\ \textbf{\textcolor{Black}{BufferedReader}}\textcolor{BrickRed}{(}\textbf{\textcolor{Blue}{new}}\ \textbf{\textcolor{Black}{InputStreamReader}}\textcolor{BrickRed}{(}System\textcolor{BrickRed}{.}in\textcolor{BrickRed}{));} \\
\mbox{}\ \ \ \ \ \ \ \ String\ line\ \textcolor{BrickRed}{=}\ reader\textcolor{BrickRed}{.}\textbf{\textcolor{Black}{readLine}}\textcolor{BrickRed}{();} \\
\mbox{}\ \ \ \ \ \ \ \ StringTokenizer\ tokenizer\ \textcolor{BrickRed}{=}\ \textbf{\textcolor{Blue}{new}}\ \textbf{\textcolor{Black}{StringTokenizer}}\textcolor{BrickRed}{(}line\textcolor{BrickRed}{);} \\
\mbox{}\ \ \ \ \ \ \ \ \textcolor{ForestGreen}{int}\ N\ \textcolor{BrickRed}{=}\ Integer\textcolor{BrickRed}{.}\textbf{\textcolor{Black}{valueOf}}\textcolor{BrickRed}{(}tokenizer\textcolor{BrickRed}{.}\textbf{\textcolor{Black}{nextToken}}\textcolor{BrickRed}{());} \\
\mbox{}\ \ \ \ \ \ \ \ \textbf{\textcolor{Blue}{while}}\ \textcolor{BrickRed}{(}N\textcolor{BrickRed}{-\/-}\ \textcolor{BrickRed}{$>$}\ \textcolor{Purple}{0}\textcolor{BrickRed}{)}\textcolor{Red}{\{} \\
\mbox{}\ \ \ \ \ \ \ \ \ \ \ \ String\ a\textcolor{BrickRed}{,}\ b\textcolor{BrickRed}{;} \\
\mbox{}\ \ \ \ \ \ \ \ \ \ \ \ a\ \textcolor{BrickRed}{=}\ reader\textcolor{BrickRed}{.}\textbf{\textcolor{Black}{readLine}}\textcolor{BrickRed}{();} \\
\mbox{}\ \ \ \ \ \ \ \ \ \ \ \ b\ \textcolor{BrickRed}{=}\ reader\textcolor{BrickRed}{.}\textbf{\textcolor{Black}{readLine}}\textcolor{BrickRed}{();} \\
\mbox{}\  \\
\mbox{}\ \ \ \ \ \ \ \ \ \ \ \ \textcolor{ForestGreen}{int}\ A\ \textcolor{BrickRed}{=}\ a\textcolor{BrickRed}{.}\textbf{\textcolor{Black}{length}}\textcolor{BrickRed}{(),}\ B\ \textcolor{BrickRed}{=}\ b\textcolor{BrickRed}{.}\textbf{\textcolor{Black}{length}}\textcolor{BrickRed}{();} \\
\mbox{}\ \ \ \ \ \ \ \ \ \ \ \ \textbf{\textcolor{Blue}{if}}\ \textcolor{BrickRed}{(}B\ \textcolor{BrickRed}{$>$}\ A\textcolor{BrickRed}{)}\textcolor{Red}{\{} \\
\mbox{}\ \ \ \ \ \ \ \ \ \ \ \ \ \ \ \ System\textcolor{BrickRed}{.}out\textcolor{BrickRed}{.}\textbf{\textcolor{Black}{println}}\textcolor{BrickRed}{(}\texttt{\textcolor{Red}{"{}0"{}}}\textcolor{BrickRed}{);} \\
\mbox{}\ \ \ \ \ \ \ \ \ \ \ \ \textcolor{Red}{\}}\textbf{\textcolor{Blue}{else}}\textcolor{Red}{\{} \\
\mbox{}\ \ \ \ \ \ \ \ \ \ \ \ \ \ \ \ BigInteger\ dp\textcolor{BrickRed}{[][]}\ \textcolor{BrickRed}{=}\ \textbf{\textcolor{Blue}{new}}\ BigInteger\textcolor{BrickRed}{[}\textcolor{Purple}{2}\textcolor{BrickRed}{][}A\textcolor{BrickRed}{];} \\
\mbox{}\ \ \ \ \ \ \ \ \ \ \ \ \ \ \ \ \textit{\textcolor{Brown}{/*}} \\
\mbox{}\textit{\textcolor{Brown}{dp[i][j]\ =\ cantidad\ de\ maneras\ diferentes}} \\
\mbox{}\textit{\textcolor{Brown}{en\ que\ puedo\ distribuir\ las\ primeras\ i}} \\
\mbox{}\textit{\textcolor{Brown}{letras\ de\ la\ subsecuencia\ (b)\ terminando}} \\
\mbox{}\textit{\textcolor{Brown}{en\ la\ letra\ j\ de\ la\ secuencia\ original\ (a)}} \\
\mbox{}\textit{\textcolor{Brown}{*/}} \\
\mbox{}\  \\
\mbox{}\ \ \ \ \ \ \ \ \ \ \ \ \ \ \ \ \textbf{\textcolor{Blue}{if}}\ \textcolor{BrickRed}{(}a\textcolor{BrickRed}{.}\textbf{\textcolor{Black}{charAt}}\textcolor{BrickRed}{(}\textcolor{Purple}{0}\textcolor{BrickRed}{)}\ \textcolor{BrickRed}{==}\ b\textcolor{BrickRed}{.}\textbf{\textcolor{Black}{charAt}}\textcolor{BrickRed}{(}\textcolor{Purple}{0}\textcolor{BrickRed}{))}\textcolor{Red}{\{} \\
\mbox{}\ \ \ \ \ \ \ \ \ \ \ \ \ \ \ \ \ \ \ \ dp\textcolor{BrickRed}{[}\textcolor{Purple}{0}\textcolor{BrickRed}{][}\textcolor{Purple}{0}\textcolor{BrickRed}{]}\ \textcolor{BrickRed}{=}\ BigInteger\textcolor{BrickRed}{.}ONE\textcolor{BrickRed}{;} \\
\mbox{}\ \ \ \ \ \ \ \ \ \ \ \ \ \ \ \ \textcolor{Red}{\}}\textbf{\textcolor{Blue}{else}}\textcolor{Red}{\{} \\
\mbox{}\ \ \ \ \ \ \ \ \ \ \ \ \ \ \ \ \ \ \ \ dp\textcolor{BrickRed}{[}\textcolor{Purple}{0}\textcolor{BrickRed}{][}\textcolor{Purple}{0}\textcolor{BrickRed}{]}\ \textcolor{BrickRed}{=}\ BigInteger\textcolor{BrickRed}{.}ZERO\textcolor{BrickRed}{;} \\
\mbox{}\ \ \ \ \ \ \ \ \ \ \ \ \ \ \ \ \textcolor{Red}{\}} \\
\mbox{}\ \ \ \ \ \ \ \ \ \ \ \ \ \ \ \ \textbf{\textcolor{Blue}{for}}\ \textcolor{BrickRed}{(}\textcolor{ForestGreen}{int}\ j\textcolor{BrickRed}{=}\textcolor{Purple}{1}\textcolor{BrickRed}{;}\ j\textcolor{BrickRed}{$<$}A\textcolor{BrickRed}{;}\ \textcolor{BrickRed}{++}j\textcolor{BrickRed}{)}\textcolor{Red}{\{} \\
\mbox{}\ \ \ \ \ \ \ \ \ \ \ \ \ \ \ \ \ \ \ \ dp\textcolor{BrickRed}{[}\textcolor{Purple}{0}\textcolor{BrickRed}{][}j\textcolor{BrickRed}{]}\ \textcolor{BrickRed}{=}\ dp\textcolor{BrickRed}{[}\textcolor{Purple}{0}\textcolor{BrickRed}{][}j\textcolor{BrickRed}{-}\textcolor{Purple}{1}\textcolor{BrickRed}{];} \\
\mbox{}\ \ \ \ \ \ \ \ \ \ \ \ \ \ \ \ \ \ \ \ \textbf{\textcolor{Blue}{if}}\ \textcolor{BrickRed}{(}a\textcolor{BrickRed}{.}\textbf{\textcolor{Black}{charAt}}\textcolor{BrickRed}{(}j\textcolor{BrickRed}{)}\ \textcolor{BrickRed}{==}\ b\textcolor{BrickRed}{.}\textbf{\textcolor{Black}{charAt}}\textcolor{BrickRed}{(}\textcolor{Purple}{0}\textcolor{BrickRed}{))}\textcolor{Red}{\{} \\
\mbox{}\ \ \ \ \ \ \ \ \ \ \ \ \ \ \ \ \ \ \ \ \ \ \ \ dp\textcolor{BrickRed}{[}\textcolor{Purple}{0}\textcolor{BrickRed}{][}j\textcolor{BrickRed}{]}\ \textcolor{BrickRed}{=}\ dp\textcolor{BrickRed}{[}\textcolor{Purple}{0}\textcolor{BrickRed}{][}j\textcolor{BrickRed}{].}\textbf{\textcolor{Black}{add}}\textcolor{BrickRed}{(}BigInteger\textcolor{BrickRed}{.}ONE\textcolor{BrickRed}{);} \\
\mbox{}\ \ \ \ \ \ \ \ \ \ \ \ \ \ \ \ \ \ \ \ \textcolor{Red}{\}} \\
\mbox{}\ \ \ \ \ \ \ \ \ \ \ \ \ \ \ \ \textcolor{Red}{\}} \\
\mbox{}\  \\
\mbox{}\ \ \ \ \ \ \ \ \ \ \ \ \ \ \ \ \textbf{\textcolor{Blue}{for}}\ \textcolor{BrickRed}{(}\textcolor{ForestGreen}{int}\ i\textcolor{BrickRed}{=}\textcolor{Purple}{1}\textcolor{BrickRed}{;}\ i\textcolor{BrickRed}{$<$}B\textcolor{BrickRed}{;}\ \textcolor{BrickRed}{++}i\textcolor{BrickRed}{)}\textcolor{Red}{\{} \\
\mbox{}\ \ \ \ \ \ \ \ \ \ \ \ \ \ \ \ \ \ \ \ dp\textcolor{BrickRed}{[}i\textcolor{BrickRed}{\%}\textcolor{Purple}{2}\textcolor{BrickRed}{][}\textcolor{Purple}{0}\textcolor{BrickRed}{]}\ \textcolor{BrickRed}{=}\ BigInteger\textcolor{BrickRed}{.}ZERO\textcolor{BrickRed}{;} \\
\mbox{}\ \ \ \ \ \ \ \ \ \ \ \ \ \ \ \ \ \ \ \ \textbf{\textcolor{Blue}{for}}\ \textcolor{BrickRed}{(}\textcolor{ForestGreen}{int}\ j\textcolor{BrickRed}{=}\textcolor{Purple}{1}\textcolor{BrickRed}{;}\ j\textcolor{BrickRed}{$<$}A\textcolor{BrickRed}{;}\ \textcolor{BrickRed}{++}j\textcolor{BrickRed}{)}\textcolor{Red}{\{} \\
\mbox{}\ \ \ \ \ \ \ \ \ \ \ \ \ \ \ \ \ \ \ \ \ \ \ \ dp\textcolor{BrickRed}{[}i\textcolor{BrickRed}{\%}\textcolor{Purple}{2}\textcolor{BrickRed}{][}j\textcolor{BrickRed}{]}\ \textcolor{BrickRed}{=}\ dp\textcolor{BrickRed}{[}i\textcolor{BrickRed}{\%}\textcolor{Purple}{2}\textcolor{BrickRed}{][}j\textcolor{BrickRed}{-}\textcolor{Purple}{1}\textcolor{BrickRed}{];} \\
\mbox{}\ \ \ \ \ \ \ \ \ \ \ \ \ \ \ \ \ \ \ \ \ \ \ \ \textbf{\textcolor{Blue}{if}}\ \textcolor{BrickRed}{(}a\textcolor{BrickRed}{.}\textbf{\textcolor{Black}{charAt}}\textcolor{BrickRed}{(}j\textcolor{BrickRed}{)}\ \textcolor{BrickRed}{==}\ b\textcolor{BrickRed}{.}\textbf{\textcolor{Black}{charAt}}\textcolor{BrickRed}{(}i\textcolor{BrickRed}{))}\textcolor{Red}{\{} \\
\mbox{}\ \ \ \ \ \ \ \ \ \ \ \ \ \ \ \ \ \ \ \ \ \ \ \ \ \ \ \ dp\textcolor{BrickRed}{[}i\textcolor{BrickRed}{\%}\textcolor{Purple}{2}\textcolor{BrickRed}{][}j\textcolor{BrickRed}{]}\ \textcolor{BrickRed}{=}\ dp\textcolor{BrickRed}{[}i\textcolor{BrickRed}{\%}\textcolor{Purple}{2}\textcolor{BrickRed}{][}j\textcolor{BrickRed}{].}\textbf{\textcolor{Black}{add}}\textcolor{BrickRed}{(}dp\textcolor{BrickRed}{[(}i\textcolor{BrickRed}{+}\textcolor{Purple}{1}\textcolor{BrickRed}{)\%}\textcolor{Purple}{2}\textcolor{BrickRed}{][}j\textcolor{BrickRed}{-}\textcolor{Purple}{1}\textcolor{BrickRed}{]);} \\
\mbox{}\ \ \ \ \ \ \ \ \ \ \ \ \ \ \ \ \ \ \ \ \ \ \ \ \textcolor{Red}{\}} \\
\mbox{}\ \ \ \ \ \ \ \ \ \ \ \ \ \ \ \ \ \ \ \ \textcolor{Red}{\}} \\
\mbox{}\ \ \ \ \ \ \ \ \ \ \ \ \ \ \ \ \textcolor{Red}{\}} \\
\mbox{}\ \ \ \ \ \ \ \ \ \ \ \ \ \ \ \ System\textcolor{BrickRed}{.}out\textcolor{BrickRed}{.}\textbf{\textcolor{Black}{println}}\textcolor{BrickRed}{(}dp\textcolor{BrickRed}{[(}B\textcolor{BrickRed}{-}\textcolor{Purple}{1}\textcolor{BrickRed}{)\%}\textcolor{Purple}{2}\textcolor{BrickRed}{][}A\textcolor{BrickRed}{-}\textcolor{Purple}{1}\textcolor{BrickRed}{].}\textbf{\textcolor{Black}{toString}}\textcolor{BrickRed}{());} \\
\mbox{}\ \ \ \ \ \ \ \ \ \ \ \ \textcolor{Red}{\}} \\
\mbox{}\ \ \ \ \ \ \ \ \textcolor{Red}{\}} \\
\mbox{}\ \ \ \ \textcolor{Red}{\}} \\
\mbox{}\textcolor{Red}{\}} \\

} \normalfont\normalsize
%.tex
\subsection{Entrada desde archivo}
% Generator: GNU source-highlight, by Lorenzo Bettini, http://www.gnu.org/software/src-highlite

{\ttfamily \raggedright {
\noindent
\mbox{}\textbf{\textcolor{RoyalBlue}{\#include}}\ \texttt{\textcolor{Red}{$<$iostream$>$}} \\
\mbox{}\textbf{\textcolor{RoyalBlue}{\#include}}\ \texttt{\textcolor{Red}{$<$fstream$>$}} \\
\mbox{} \\
\mbox{}\textbf{\textcolor{Blue}{using}}\ \textbf{\textcolor{Blue}{namespace}}\ std\textcolor{BrickRed}{;} \\
\mbox{} \\
\mbox{}\textcolor{ForestGreen}{int}\ \textbf{\textcolor{Black}{$\_$main}}\textcolor{BrickRed}{()}\textcolor{Red}{\{} \\
\mbox{}\ \ \textbf{\textcolor{Black}{freopen}}\textcolor{BrickRed}{(}\texttt{\textcolor{Red}{"{}entrada.in"{}}}\textcolor{BrickRed}{,}\ \texttt{\textcolor{Red}{"{}r"{}}}\textcolor{BrickRed}{,}\ stdin\textcolor{BrickRed}{);} \\
\mbox{}\ \ \textbf{\textcolor{Black}{freopen}}\textcolor{BrickRed}{(}\texttt{\textcolor{Red}{"{}entrada.out"{}}}\textcolor{BrickRed}{,}\ \texttt{\textcolor{Red}{"{}w"{}}}\textcolor{BrickRed}{,}\ stdout\textcolor{BrickRed}{);} \\
\mbox{} \\
\mbox{}\ \ string\ s\textcolor{BrickRed}{;} \\
\mbox{}\ \ \textbf{\textcolor{Blue}{while}}\ \textcolor{BrickRed}{(}cin\ \textcolor{BrickRed}{$>$$>$}\ s\textcolor{BrickRed}{)}\textcolor{Red}{\{} \\
\mbox{}\ \ \ \ cout\ \textcolor{BrickRed}{$<$$<$}\ \texttt{\textcolor{Red}{"{}Leí\ "{}}}\ \textcolor{BrickRed}{$<$$<$}\ s\ \textcolor{BrickRed}{$<$$<$}\ endl\textcolor{BrickRed}{;} \\
\mbox{}\ \ \textcolor{Red}{\}} \\
\mbox{}\ \ \textbf{\textcolor{Blue}{return}}\ \textcolor{Purple}{0}\textcolor{BrickRed}{;} \\
\mbox{}\textcolor{Red}{\}} \\
\mbox{} \\
\mbox{} \\
\mbox{}\textcolor{ForestGreen}{int}\ \textbf{\textcolor{Black}{main}}\textcolor{BrickRed}{()}\textcolor{Red}{\{} \\
\mbox{}\ \ ifstream\ \textbf{\textcolor{Black}{fin}}\textcolor{BrickRed}{(}\texttt{\textcolor{Red}{"{}entrada.in"{}}}\textcolor{BrickRed}{);} \\
\mbox{}\ \ ofstream\ \textbf{\textcolor{Black}{fout}}\textcolor{BrickRed}{(}\texttt{\textcolor{Red}{"{}entrada.out"{}}}\textcolor{BrickRed}{);} \\
\mbox{} \\
\mbox{}\ \ string\ s\textcolor{BrickRed}{;} \\
\mbox{}\ \ \textbf{\textcolor{Blue}{while}}\ \textcolor{BrickRed}{(}fin\ \textcolor{BrickRed}{$>$$>$}\ s\textcolor{BrickRed}{)}\textcolor{Red}{\{} \\
\mbox{}\ \ \ \ fout\ \textcolor{BrickRed}{$<$$<$}\ \texttt{\textcolor{Red}{"{}Leí\ "{}}}\ \textcolor{BrickRed}{$<$$<$}\ s\ \textcolor{BrickRed}{$<$$<$}\ endl\textcolor{BrickRed}{;} \\
\mbox{}\ \ \textcolor{Red}{\}} \\
\mbox{}\ \ \textbf{\textcolor{Blue}{return}}\ \textcolor{Purple}{0}\textcolor{BrickRed}{;} \\
\mbox{}\textcolor{Red}{\}} \\

} \normalfont\normalsize
%.tex

\section{C++}
\subsection{Entrada desde archivo}
% Generator: GNU source-highlight, by Lorenzo Bettini, http://www.gnu.org/software/src-highlite

{\ttfamily \raggedright {
\noindent
\mbox{}\textbf{\textcolor{RoyalBlue}{\#include}}\ \texttt{\textcolor{Red}{$<$iostream$>$}} \\
\mbox{}\textbf{\textcolor{RoyalBlue}{\#include}}\ \texttt{\textcolor{Red}{$<$fstream$>$}} \\
\mbox{} \\
\mbox{}\textbf{\textcolor{Blue}{using}}\ \textbf{\textcolor{Blue}{namespace}}\ std\textcolor{BrickRed}{;} \\
\mbox{} \\
\mbox{}\textcolor{ForestGreen}{int}\ \textbf{\textcolor{Black}{$\_$main}}\textcolor{BrickRed}{()}\textcolor{Red}{\{} \\
\mbox{}\ \ \textbf{\textcolor{Black}{freopen}}\textcolor{BrickRed}{(}\texttt{\textcolor{Red}{"{}entrada.in"{}}}\textcolor{BrickRed}{,}\ \texttt{\textcolor{Red}{"{}r"{}}}\textcolor{BrickRed}{,}\ stdin\textcolor{BrickRed}{);} \\
\mbox{}\ \ \textbf{\textcolor{Black}{freopen}}\textcolor{BrickRed}{(}\texttt{\textcolor{Red}{"{}entrada.out"{}}}\textcolor{BrickRed}{,}\ \texttt{\textcolor{Red}{"{}w"{}}}\textcolor{BrickRed}{,}\ stdout\textcolor{BrickRed}{);} \\
\mbox{} \\
\mbox{}\ \ string\ s\textcolor{BrickRed}{;} \\
\mbox{}\ \ \textbf{\textcolor{Blue}{while}}\ \textcolor{BrickRed}{(}cin\ \textcolor{BrickRed}{$>$$>$}\ s\textcolor{BrickRed}{)}\textcolor{Red}{\{} \\
\mbox{}\ \ \ \ cout\ \textcolor{BrickRed}{$<$$<$}\ \texttt{\textcolor{Red}{"{}Leí\ "{}}}\ \textcolor{BrickRed}{$<$$<$}\ s\ \textcolor{BrickRed}{$<$$<$}\ endl\textcolor{BrickRed}{;} \\
\mbox{}\ \ \textcolor{Red}{\}} \\
\mbox{}\ \ \textbf{\textcolor{Blue}{return}}\ \textcolor{Purple}{0}\textcolor{BrickRed}{;} \\
\mbox{}\textcolor{Red}{\}} \\
\mbox{} \\
\mbox{} \\
\mbox{}\textcolor{ForestGreen}{int}\ \textbf{\textcolor{Black}{main}}\textcolor{BrickRed}{()}\textcolor{Red}{\{} \\
\mbox{}\ \ ifstream\ \textbf{\textcolor{Black}{fin}}\textcolor{BrickRed}{(}\texttt{\textcolor{Red}{"{}entrada.in"{}}}\textcolor{BrickRed}{);} \\
\mbox{}\ \ ofstream\ \textbf{\textcolor{Black}{fout}}\textcolor{BrickRed}{(}\texttt{\textcolor{Red}{"{}entrada.out"{}}}\textcolor{BrickRed}{);} \\
\mbox{} \\
\mbox{}\ \ string\ s\textcolor{BrickRed}{;} \\
\mbox{}\ \ \textbf{\textcolor{Blue}{while}}\ \textcolor{BrickRed}{(}fin\ \textcolor{BrickRed}{$>$$>$}\ s\textcolor{BrickRed}{)}\textcolor{Red}{\{} \\
\mbox{}\ \ \ \ fout\ \textcolor{BrickRed}{$<$$<$}\ \texttt{\textcolor{Red}{"{}Leí\ "{}}}\ \textcolor{BrickRed}{$<$$<$}\ s\ \textcolor{BrickRed}{$<$$<$}\ endl\textcolor{BrickRed}{;} \\
\mbox{}\ \ \textcolor{Red}{\}} \\
\mbox{}\ \ \textbf{\textcolor{Blue}{return}}\ \textcolor{Purple}{0}\textcolor{BrickRed}{;} \\
\mbox{}\textcolor{Red}{\}} \\

} \normalfont\normalsize
%.tex

\subsection{Strings con caractéres especiales}
% Generator: GNU source-highlight, by Lorenzo Bettini, http://www.gnu.org/software/src-highlite

{\ttfamily \raggedright {
\noindent
\mbox{}\textbf{\textcolor{RoyalBlue}{\#include}}\ \texttt{\textcolor{Red}{$<$iostream$>$}} \\
\mbox{}\textbf{\textcolor{RoyalBlue}{\#include}}\ \texttt{\textcolor{Red}{$<$cassert$>$}} \\
\mbox{}\textbf{\textcolor{RoyalBlue}{\#include}}\ \texttt{\textcolor{Red}{$<$stdio.h$>$}} \\
\mbox{}\textbf{\textcolor{RoyalBlue}{\#include}}\ \texttt{\textcolor{Red}{$<$assert.h$>$}} \\
\mbox{}\textbf{\textcolor{RoyalBlue}{\#include}}\ \texttt{\textcolor{Red}{$<$wchar.h$>$}} \\
\mbox{}\textbf{\textcolor{RoyalBlue}{\#include}}\ \texttt{\textcolor{Red}{$<$wctype.h$>$}} \\
\mbox{}\textbf{\textcolor{RoyalBlue}{\#include}}\ \texttt{\textcolor{Red}{$<$locale.h$>$}} \\
\mbox{} \\
\mbox{}\textbf{\textcolor{Blue}{using}}\ \textbf{\textcolor{Blue}{namespace}}\ std\textcolor{BrickRed}{;} \\
\mbox{} \\
\mbox{}\textcolor{ForestGreen}{int}\ \textbf{\textcolor{Black}{main}}\textcolor{BrickRed}{()}\textcolor{Red}{\{} \\
\mbox{}\ \ \textbf{\textcolor{Black}{assert}}\textcolor{BrickRed}{(}\textbf{\textcolor{Black}{setlocale}}\textcolor{BrickRed}{(}LC$\_$ALL\textcolor{BrickRed}{,}\ \texttt{\textcolor{Red}{"{}en$\_$US.UTF-8"{}}}\textcolor{BrickRed}{)}\ \textcolor{BrickRed}{!=}\ NULL\textcolor{BrickRed}{);} \\
\mbox{}\ \ \textcolor{ForestGreen}{wchar$\_$t}\ c\textcolor{BrickRed}{;} \\
\mbox{} \\
\mbox{}\ \ wstring\ s\textcolor{BrickRed}{;} \\
\mbox{}\ \ \textbf{\textcolor{Blue}{while}}\ \textcolor{BrickRed}{(}\textbf{\textcolor{Black}{getline}}\textcolor{BrickRed}{(}wcin\textcolor{BrickRed}{,}\ s\textcolor{BrickRed}{))}\textcolor{Red}{\{} \\
\mbox{}\ \ \ \ wcout\ \textcolor{BrickRed}{$<$$<$}\ L\texttt{\textcolor{Red}{"{}Leí\ :\ "{}}}\ \textcolor{BrickRed}{$<$$<$}\ s\ \textcolor{BrickRed}{$<$$<$}\ endl\textcolor{BrickRed}{;} \\
\mbox{}\ \ \ \ \textbf{\textcolor{Blue}{for}}\ \textcolor{BrickRed}{(}\textcolor{ForestGreen}{int}\ i\textcolor{BrickRed}{=}\textcolor{Purple}{0}\textcolor{BrickRed}{;}\ i\textcolor{BrickRed}{$<$}s\textcolor{BrickRed}{.}\textbf{\textcolor{Black}{size}}\textcolor{BrickRed}{();}\ \textcolor{BrickRed}{++}i\textcolor{BrickRed}{)}\textcolor{Red}{\{} \\
\mbox{}\ \ \ \ \ \ c\ \textcolor{BrickRed}{=}\ s\textcolor{BrickRed}{[}i\textcolor{BrickRed}{];} \\
\mbox{}\ \ \ \ \ \ \textbf{\textcolor{Black}{wprintf}}\textcolor{BrickRed}{(}L\texttt{\textcolor{Red}{"{}\%lc\ \%lc}}\texttt{\textcolor{CarnationPink}{\textbackslash{}n}}\texttt{\textcolor{Red}{"{}}}\textcolor{BrickRed}{,}\ \textbf{\textcolor{Black}{towlower}}\textcolor{BrickRed}{(}s\textcolor{BrickRed}{[}i\textcolor{BrickRed}{]),}\ \textbf{\textcolor{Black}{towupper}}\textcolor{BrickRed}{(}s\textcolor{BrickRed}{[}i\textcolor{BrickRed}{]));} \\
\mbox{}\ \ \ \ \textcolor{Red}{\}} \\
\mbox{}\ \ \textcolor{Red}{\}} \\
\mbox{} \\
\mbox{}\ \ \textbf{\textcolor{Blue}{return}}\ \textcolor{Purple}{0}\textcolor{BrickRed}{;} \\
\mbox{}\textcolor{Red}{\}} \\
\mbox{} \\
\mbox{} \\

} \normalfont\normalsize
%.tex

\emph{Nota}: Como alternativa a la función getline, se pueden utilizar las funciones fgetws y fputws, y más adelante swscanf y wprintf:
% Generator: GNU source-highlight, by Lorenzo Bettini, http://www.gnu.org/software/src-highlite

{\ttfamily \raggedright {
\noindent
\mbox{}\textbf{\textcolor{RoyalBlue}{\#include}}\ \texttt{\textcolor{Red}{$<$iostream$>$}} \\
\mbox{}\textbf{\textcolor{RoyalBlue}{\#include}}\ \texttt{\textcolor{Red}{$<$cassert$>$}} \\
\mbox{}\textbf{\textcolor{RoyalBlue}{\#include}}\ \texttt{\textcolor{Red}{$<$stdio.h$>$}} \\
\mbox{}\textbf{\textcolor{RoyalBlue}{\#include}}\ \texttt{\textcolor{Red}{$<$assert.h$>$}} \\
\mbox{}\textbf{\textcolor{RoyalBlue}{\#include}}\ \texttt{\textcolor{Red}{$<$wchar.h$>$}} \\
\mbox{}\textbf{\textcolor{RoyalBlue}{\#include}}\ \texttt{\textcolor{Red}{$<$wctype.h$>$}} \\
\mbox{}\textbf{\textcolor{RoyalBlue}{\#include}}\ \texttt{\textcolor{Red}{$<$locale.h$>$}} \\
\mbox{} \\
\mbox{} \\
\mbox{}\textbf{\textcolor{Blue}{using}}\ \textbf{\textcolor{Blue}{namespace}}\ std\textcolor{BrickRed}{;} \\
\mbox{} \\
\mbox{}\textcolor{ForestGreen}{int}\ \textbf{\textcolor{Black}{main}}\textcolor{BrickRed}{()}\textcolor{Red}{\{} \\
\mbox{}\ \ \textbf{\textcolor{Black}{assert}}\textcolor{BrickRed}{(}\textbf{\textcolor{Black}{setlocale}}\textcolor{BrickRed}{(}LC$\_$ALL\textcolor{BrickRed}{,}\ \texttt{\textcolor{Red}{"{}en$\_$US.UTF-8"{}}}\textcolor{BrickRed}{)}\ \textcolor{BrickRed}{!=}\ NULL\textcolor{BrickRed}{);} \\
\mbox{}\ \ \textcolor{ForestGreen}{wchar$\_$t}\ in$\_$buf\textcolor{BrickRed}{[}\textcolor{Purple}{512}\textcolor{BrickRed}{],}\ out$\_$buf\textcolor{BrickRed}{[}\textcolor{Purple}{512}\textcolor{BrickRed}{];} \\
\mbox{}\ \ \textbf{\textcolor{Black}{swprintf}}\textcolor{BrickRed}{(}out$\_$buf\textcolor{BrickRed}{,}\ \textcolor{Purple}{512}\textcolor{BrickRed}{,}\ L\texttt{\textcolor{Red}{"{}¿Podrías\ escribir\ un\ número?,\ Por\ ejemplo\ \%d.\ ¡Gracias,\ pingüino\ español!}}\texttt{\textcolor{CarnationPink}{\textbackslash{}n}}\texttt{\textcolor{Red}{"{}}}\textcolor{BrickRed}{,}\ \textcolor{Purple}{3}\textcolor{BrickRed}{);} \\
\mbox{}\ \ \textbf{\textcolor{Black}{fputws}}\textcolor{BrickRed}{(}out$\_$buf\textcolor{BrickRed}{,}\ stdout\textcolor{BrickRed}{);} \\
\mbox{} \\
\mbox{}\ \ \textbf{\textcolor{Black}{fgetws}}\textcolor{BrickRed}{(}in$\_$buf\textcolor{BrickRed}{,}\ \textcolor{Purple}{512}\textcolor{BrickRed}{,}\ stdin\textcolor{BrickRed}{);} \\
\mbox{}\ \ \textcolor{ForestGreen}{int}\ n\textcolor{BrickRed}{;} \\
\mbox{}\ \ \textbf{\textcolor{Black}{swscanf}}\textcolor{BrickRed}{(}in$\_$buf\textcolor{BrickRed}{,}\ L\texttt{\textcolor{Red}{"{}\%d"{}}}\textcolor{BrickRed}{,}\ \textcolor{BrickRed}{\&}n\textcolor{BrickRed}{);} \\
\mbox{} \\
\mbox{}\ \ \textbf{\textcolor{Black}{swprintf}}\textcolor{BrickRed}{(}out$\_$buf\textcolor{BrickRed}{,}\ \textcolor{Purple}{512}\textcolor{BrickRed}{,}\ L\texttt{\textcolor{Red}{"{}Escribiste\ \%d,\ yo\ escribo\ ¿ÔÏàÚÑ\textasciitilde{}}}\texttt{\textcolor{CarnationPink}{\textbackslash{}n}}\texttt{\textcolor{Red}{"{}}}\textcolor{BrickRed}{,}\ n\textcolor{BrickRed}{);} \\
\mbox{}\ \ \textbf{\textcolor{Black}{fputws}}\textcolor{BrickRed}{(}out$\_$buf\textcolor{BrickRed}{,}\ stdout\textcolor{BrickRed}{);} \\
\mbox{} \\
\mbox{}\ \ \textbf{\textcolor{Blue}{return}}\ \textcolor{Purple}{0}\textcolor{BrickRed}{;} \\
\mbox{}\textcolor{Red}{\}} \\
\mbox{} \\
\mbox{} \\

} \normalfont\normalsize
%.tex

\end{document}