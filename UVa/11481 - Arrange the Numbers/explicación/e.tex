\documentclass[10pt,letterpaper]{article}

%---------------------------------------------------------------
\usepackage[utf8]{inputenc}
\usepackage[spanish]{babel}
\usepackage{listings}
\usepackage[usenames,dvipsnames]{color}
\usepackage{amsmath}
\usepackage{amsthm}
\usepackage{amssymb}
%\usepackage{color}
%---------------------------------------------------------------

\setlength{\topmargin}{-1.0in}
\setlength{\textheight}{9.5in} 
\setlength{\evensidemargin}{0.0in}
\setlength{\oddsidemargin}{0.0in}
\setlength{\textwidth}{6.5in} 

\begin{document}

La respuesta es 
$$
{m \choose k} \cdot \displaystyle \sum_{i=0}^{m-k} (-1)^{i}  {m - k \choose i} (n-k-i)!
$$

\bigskip

\textbf{Explicación:} Lo primero que hacemos es escoger los $k$ elementos que quedarán ``clavados'' en su sitio correcto. Esto se puede hacer de $ {m \choose k} $ maneras diferentes, pues cualquier subconjunto de $k$ elementos entre los primeros $m$ elementos sirve para este propósito. \\
\\
Ahora bien, por cada una de estas clavadas nos quedan $ n - k $ elementos por ubicar. Hay que hacerlo de manera que en las primeras $ m - k $ posiciones ningún elemento quede en su sitio, porque si sucede lo contrario entonces habrá mas elementos clavados en su sitio de los $k$ pedidos. Para contar cuántas permutaciones de $n-k$ elementos cumplen esto, utilizamos el principio de inclusión-exclusión\footnote {Bastante similar a la demostración del teorema 2, ``Número de permutaciones completas'', Sección 6.6: \textit{Aplicaciones del principio de inclusión-exclusión}, Capítulo 6: \textit{Técnicas avanzadas de recuento}, página 430, del libro \textbf{Matemáticas discretas y sus aplicaciones}, Kenneth Rosen, quinta edición, McGraw-Hill.}. El resultado es la sumatoria que multiplica a ${ m \choose k}$.\\

\begin{proof}{}
Decimos que una permutación tiene la propiedad $P_{i}$ si deja en su sitio el elemento $i$. El número que se busca es el número de permutaciones de $n - k$ elementos que no tiene la propiedad $P_{i}$ para $i=1, 2, \cdots , m - k$ y lo denotamos $ D = N(\bar{P_{1}} \bar{P_{2}} \cdots \bar{P}_{m-k}) $. Aplicando el principio de inclusión-exclusión tenemos que:\\
$$
D = N - \sum_{i}{N(P_{i})} + \sum_{i < j}{N(P_{i}P_{j})} - \sum_{i < j < k}{N(P_{i}P_{j}P_{k})} + \cdots + (-1)^{m-k}N(P_{1}P_{2} \cdots P_{m-k})
$$

Vemos que $$ N = (n-k)! $$ pues $N$ es la cantidad de permutaciones posibles.\\
Así mismo, $$ \sum_{i}{N(P_{i})} = { m - k \choose 1 } (n -k -1)! $$ Esto es, clavamos un sólo elemento de los primeros $m - k$ posibles, y el resto los permutamos de cualquier manera posible. Pero al restar estos elementos, restamos dos veces las permutaciones que tienen dos elementos fijos, así que las volvemos a sumar:
$$\sum_{i < j}{N(P_{i}P_{j})} = { m - k \choose 2 } (n -k -2)! $$
Esto es, clavamos dos elementos y el resto los permutamos de cualquier manera posible.\\
Y en general,
$$\sum{N(P_{i_{1}}P_{i_{2}} \cdots P_{i_{r}})} = { m - k \choose r } (n -k -r)! $$
Reemplazando estos términos en la ecuación original, obtenemos
$$
D = N - \sum_{i}{N(P_{i})} + \sum_{i < j}{N(P_{i}P_{j})} - \sum_{i < j < k}{N(P_{i}P_{j}P_{k})} + \cdots + (-1)^{m-k}N(P_{1}P_{2} \cdots P_{m-k}) $$
$$
= (n - k)! - { m - k \choose 1 } (n -k -1)! +  { m - k \choose 2 } (n -k -2)! - 
 \cdots + (-1)^{m-k}  { m - k \choose m - k } (n -k -(m - k))!
$$
$$
= \sum_{i=0}^{m-k} (-1)^{i}  {m - k \choose i} (n-k-i)!
$$
\end{proof}
\end{document}